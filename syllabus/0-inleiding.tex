\chapter*{Studiewijzer}%
\label{ch:studiewijzer}
\addcontentsline{toc}{chapter}{Studiewijzer}

\section{Doel en plaats van de cursus in het curriculum}%
\label{sec:doel-en-plaats}

Het doel van deze cursus is om de student voor te bereiden op de bachelorproef. Meer bepaald willen we op deze manier studenten ondersteunen bij:

\begin{itemize}
  \item Het zoeken naar een onderwerp en formuleren van een onderzoeksvraag;
  \item Kritisch opzoeken van informatie in de vakliteratuur (literatuuronderzoek);
  \item Het opzetten van een bibliografische databank en correct refereren naar bronnen;
  \item Geschikte onderzoeksmethoden toepassen;
  \item Gebruik maken van het {\LaTeX} tekstzetsysteem voor een professionele lay-out;
  \item Helder en correct communiceren over je onderzoek.
\end{itemize}

Om dit vak te kunnen volgen is geen specifieke voorkennis vereist. Een credit voor dit vak is vereist om de bachelorproef te kunnen opnemen.

\section{Leerdoelen en competenties}%
\label{sec:leerdoelen}

Zie ook de studiefiche.

\begin{itemize}
  \item Onderzoeksvraag formuleren
        \begin{itemize}
          \item De student kan de kenmerken van een goede onderzoeksvraag toelichten
          \item De student kan een onderzoeksvraag formuleren en aftoetsen aan de kenmerken van een goede onderzoeksvraag.
          \item De student kan een onderzoeksvraag opsplitsen in concrete deelvragen en bijhorende doelstellingen
          \item De student kan de verschillende soorten onderzoek toelichten en van elkaar onderscheiden
          \item De student kan de verschillende onderzoeksmethoden toelichten en van elkaar onderscheiden in functie van geschiktheid
        \end{itemize}

  \item Literatuurstudie
        \begin{itemize}
          \item De student kan de criteria van kwalitatieve vakliteratuur toelichten
          \item De student kan verschillende soorten bronnen van elkaar onderscheiden
          \item De student kan zijn zoekstrategie aanpassen in functie van de zoekresultaten
          \item De student kan vakliteratuur raadplegen ifv een thema
          \item De student kan uit vakliteratuur de essentie halen
          \item De student kan kenmerken van een gestructureerde tekst toelichten
          \item De student kan een gestructureerde tekst schrijven
          \item De student kan het belang van correct refereren toelichten
          \item De student kan correct refereren
        \end{itemize}

  \item Onderzoeksmethoden
        \begin{itemize}
          \item De student kan een onderzoeksmethode selecteren in functie van de probleemstelling
          \item De student kan een onderzoeksmethode uitvoeren in functie van de probleemstelling
          \item De student kan methoden om kwantitatieve data te verzamelen van elkaar onderscheiden en toepassen
          \item De student kan methoden om kwalitatieve data te verzamelen van elkaar onderscheiden
        \end{itemize}

  \item Rapporteren
        \begin{itemize}
          \item De student kan een document op een correct gestructureerde manier opmaken en voorzien van referenties m.b.v.\ een tekstzetsysteem.
        \end{itemize}
\end{itemize}

\section{Leerinhoud}%
\label{sec:leerinhoud}

Dit opleidingsonderdeel is onderverdeeld in twee grote luiken. In het IT-luik ligt de focus op het vakspecifieke, bv.\ zoeken naar een geschikt onderwerp, informatie opzoeken in de vakliteratuur, enz. Het taalluik biedt ondersteuning in heldere communicatie: teksten analyseren, een professionele schrijfstijl hanteren, typische taalfouten vermijden, enz.

In dit document vind je vooral de inhoud van het IT-luik.

Module~\ref{ch:voorbereiding} helpt bij het opzetten van een werkomgeving, meer bepaald het tekstzetsysteem {\LaTeX} en een versiebeheersysteem.

Module~\ref{ch:onderzoeksvraag} geeft tips bij het kiezen van een onderwerp en uitschrijven van een onderwerp.

Module~\ref{ch:literatuuronderzoek} gaat gedetailleerd in op het voeren van een literatuuronderzoek. In de daarop volgende Module~\ref{ch:bibliografie} leren we hoe je de gevonden referenties gestructureerd kan bijhouden in een bibliografische databank zodat je een correct vormgegeven bibliografie in een {\LaTeX}-document kan invoegen.

Module~\ref{ch:onderzoeksmethoden} bespreekt een aantal onderzoeksmethoden die typisch gebruikt worden in een bachelorproef, o.a.\ de vergelijkende studie, of het opzetten van experimenten.

Module~\ref{ch:resultaten-rapporteren}, tenslotte bevat een aantal tips in verband met de eerder technische aspecten van het schrijven van een scriptie, en enkele {\LaTeX}-specifieke richtlijnen.

\section{Leermateriaal}%
\label{sec:leermateriaal}

Al het leermateriaal voor dit opleidingsonderdeel (deze cursus, leerpaden, \ldots) wordt via Chamilo ter beschikking gesteld.

\section{Werkvormen}%
\label{sec:werkvormen}

In dit opleidingsonderdeel voorzien we twaalf contactmomenten van elk twee uur, afwisselend rond enerzijds IT en anderzijds talen. De onderlinge volgorde kan verschillen van klasgroep tot klasgroep. In de tabel hieronder vind je de onderwerpen van elk luik die in de opeenvolgende lessen aan bod komen.

Je vindt een overzicht in Tabel~\ref{tab:planning}.

\begin{table}
  \centering\begin{tabular}{cll}
    \toprule
    \textbf{Module} & \textbf{IT-luik}               & \textbf{Taalluik}                   \\
    \midrule
    1               & Voorbereiding, {\LaTeX}        & Structuur van professionele teksten \\
    2               & Een onderzoeksvraag formuleren & Professionele schrijfstijl          \\
    3               & Literatuurstudie               & Typische taalfouten                 \\
    4               & Correct refereren              & Bronnen kritisch benaderen          \\
    5               & Onderzoeksmethoden             & Een samenvatting schrijven          \\
    6               & Rapporteren                    & Een goede presentatie geven         \\
  \end{tabular}
  \caption{\label{tab:planning}Onderwerpen van de contactmomenten voor dit opleidingsonderdeel.}
\end{table}

\section{Werk- en leeraanwijzingen}%
\label{sec:werk-en-leeraanwijzingen}

TODO

\section{Studiebegeleiding en planning}%
\label{sec:studiebegeleiding-en-planning}

TODO

\section{Evaluatie}%
\label{sec:evaluatie}

De evaluatie voor dit opleidingsonderdeel gebeurt op basis van een paper, door elke student individueel te schrijven of in groepen van twee studenten. In deze paper werk je een onderwerp uit dat geschikt kan zijn om een bachelorproef rond te schrijven. Je formuleert een gepasste onderzoeksvraag, voert een initiële literatuurstudie over het onderzoeksdomein en beschrijft hoe je van plan bent om je onderzoek aan te pakken. 

Let wel: je bent in je laatste jaar niet verplicht om je aan dit onderwerp te houden voor je bachelorproef. Je zal immers nog veel vakkennis (en maturiteit) opdoen in de specialisatievakken, wat tot nieuwe inspiratie kan leiden. Door binnen dit vak alvast eens de oefening te maken, hopen we echter dat je een veel beter beeld hebt van wat we verwachten van de bachelorproef en dat je voldoende praktische vaardigheden meeneemt waarmee je aan de slag kan.

Het resultaat wordt beoordeeld door enerzijds de IT-lector en anderzijds de taallector, elk vanuit hun expertise. Zo heeft de IT-lector aandacht voor de kwaliteit van de gerefereerde bronnen, correct gebruik van {\LaTeX}, enz., terwijl de taallector kijkt naar de structuur, schrijfstijl en correct taalgebruik.

De criteria waarop beoordeeld wordt, zullen op Chamilo gepubliceerd worden in de vorm van een evaluatiekaart gebaseerd op Rubrics.
