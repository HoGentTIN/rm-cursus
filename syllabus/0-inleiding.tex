\chapter*{Inleiding}
\label{ch:inleiding}

\section{Studiewijzer}
\label{sec:studiewijzer}

\subsection{Doel en plaats van de cursus in het curriculum}
\label{sec:doel-en-plaats}

\lipsum[3-4]

\subsection{Leerdoelen en competenties}
\label{sec:leerdoelen}

\lipsum[5-6]

\subsection{Leerinhoud}
\label{sec:leerinhoud}

\lipsum[7-9]

\subsection{Leermateriaal}
\label{sec:leermateriaal}

\lipsum[10-12]

\subsection{Werkvormen}
\label{sec:werkvormen}

\lipsum[13]

\subsection{Werk- en leeraanwijzingen}
\label{sec:werk-en-leeraanwijzingen}

\lipsum[14]

\subsection{Studiebegeleiding en planning}
\label{sec:studiebegeleiding-en-planning}

\lipsum[15]

\subsection{Evaluatie}
\label{sec:evaluatie}

\lipsum[16-18]


\subsection{Structuur van deze cursus}
\label{sec:structuur}

De rest van deze cursus is als volgt gestructureerd:

Hoofdstuk~\ref{ch:voorbereiding} helpt bij het opzetten van een werkomgeving, meer bepaald {\LaTeX} en een versiebeheersysteem.

Hoofdstuk~\ref{ch:onderwerp} geeft tips bij het kiezen van een onderwerp en uitschrijven van een onderwerp.

Hoofdstuk~\ref{ch:literatuuronderzoek} gaat gedetailleerd in op het voeren van een literatuuronderzoek, het bijhouden van referenties in een bibliografische databank en hoe dit correct in een {\LaTeX}-document te verwerken.

Hoofdstuk~\ref{ch:onderzoeksmethoden} bespreekt een aantal onderzoeksmethoden die typisch gebruikt worden in een bachelorproef, o.a. de vergelijkende studie, of het opzetten van experimenten.

Hoofdstuk~\ref{ch:schrijven}, tenslotte bevat een aantal tips in verband met het schrijven van de tekst, en enkele {\LaTeX}-specifieke richtlijnen.
