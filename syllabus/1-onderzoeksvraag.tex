\chapter{Een onderzoeksvraag formuleren}%
\label{ch:onderzoeksvraag}

In dit hoofdstuk worden enkele suggesties gegeven voor het zoeken naar een onderwerp dat geschikt is voor de bachelorproef toegepaste informatica.

De doelstelling van de bachelorproef is om aan te tonen dat je in staat bent om een onderbouwd advies te geven over het toepassen van actuele ict-technologieën in een bedrijfscontext.

Dat houdt in dat je relevante informatie kan verzamelen en kritisch beschouwen. Deze informatie zoek je enerzijds op in de vakliteratuur, waarbij je erop let dat je bronnen gebruikt die betrouwbaar zijn en relevant voor jouw specifiek onderzoek. Anderzijds ga je in gesprek met belanghebbenden, zoals de opdrachtgever, de eindgebruiker, enz., bijvoorbeeld om requirements te verzamelen. Daarnaast zal je ook zelf nieuwe kennis moeten genereren, bijvoorbeeld door het uitvoeren van een experiment of het uitwerken van een prototype, proefopstelling of proof of concept. Op basis daarvan rapporteer je over je bevindingen en formuleer je een advies.

Alles start met een goede onderzoeksvraag. Vanuit de opleiding krijgen we van externen regelmatig aanbiedingen van onderwerpen die geschikt zijn voor een bachelorproef, maar dat aanbod is niet groot genoeg om alle studenten van een onderwerp te voorzien. Langs de andere kant is het zelf uitwerken van een onderwerp een interessante kans om je te verdiepen in een onderwerp waar je na je afstuderen graag mee zou verder gaan.

\section{Soorten onderzoek}%
\label{sec:soorten-onderzoek}

Een bachelorproef hoeft geen academische masterthesis proberen na te bootsen. De professionele bachelor heeft zijn eigen waarde het is dus perfect mogelijk om te excelleren binnen de specifieke eigenheid van dit profiel. Het grote verschil bestaat in de eerste plaats uit het soort onderzoek dat gevoerd wordt.

Bij \emph{fundamenteel onderzoek}, dat typisch aan universiteiten uitgevoerd wordt, ligt de nadruk op het uitbreiden van onze kennis in het algemeen. Binnen computerwetenschappen kan het bijvoorbeeld gaan over het ontwikkelen van nieuwe en/of efficiëntere algoritmen voor probleemoplossing. Of de resultaten van fundamenteel onderzoek ook onmiddellijk toepasbaar zijn is van secundair belang.

\emph{Toegepast onderzoek}, daarentegen, probeert een antwoord te formuleren op een concrete vraag uit het werkveld. De onderzoeker zal dan aan de hand van de kennis die er op dit moment beschikbaar is (en gepubliceerd door vakexperten), proberen die vraag te beantwoorden. Dat kan bijvoorbeeld gaan over hoe een nieuwe technologie concreet in een bedrijfscontext kan toegepast worden, een keuze maken tussen verschillende alternatieve producten of technologieën, een vooronderzoek voorafgaand aan het ontwikkelen van een applicatie, enz. Bij toegepast onderzoek is de doelgroep ook heel specifiek, bijvoorbeeld één enkel bedrijf. We merken dan ook dat de onderwerpen die aangebracht zijn door bedrijven ook het best uitgewerkt zijn en het vaakst leiden tot een goede bachelorproef.

Op hogescholen wordt typisch aan toegepast onderzoek gedaan. Voorbeelden daarvan zijn marktonderzoek, het verzamelen van requirements en ontwerpen van een architectuur voor een ict-oplossing, een vergelijkende studie uitvoeren om een keuze te maken tussen verschillende alternatieve technologieën, het opzetten van een proof-of-concept om een nieuwe technologie te testen, enz.

Toegepast onderzoek is niet ondergeschikt aan fundamenteel onderzoek, maar is complementair. Resultaten van fundamenteel onderzoek zijn immers vaak niet onmiddellijk toepasbaar in de praktijk. Deze zijn bijvoorbeeld gebaseerd op ``toy problems'' die niet representatief zijn voor of de echte wereld, of een te grote vereenvoudiging ervan. Daarom is het belangrijk dat er ook toegepast onderzoek gebeurt, zodat de resultaten van fundamenteel onderzoek wel degelijk in de praktijk kunnen toegepast worden.

De professionele bachelor toegepaste informatica is uniek gepositioneerd om als vakspecialist de vertaalslag te maken tussen theorie en praktijk.

\section{Het onderzoeksdomein kiezen}%
\label{sec:het_onderzoeksdomein_kiezen}

Een eerste stap is het kiezen van je onderzoeksdomein. Dit is iets waar eigenlijk niemand je mee kan helpen. Kies een domein waar je zelf in geïnteresseerd bent, zodat je voldoende motivatie kan opbrengen om je je hier in te verdiepen. Je gekozen specialisatierichting in het laatste jaar is wellicht een goed startpunt. In welk soort job zou je na je afstuderen graag starten? Met welke technologieën, platformen, \ldots\ zou je graag werken?

Ga op zoek te naar de actualiteit binnen je gekozen vakdomein. Belangrijk is je tijd te nemen om je ``onder te dompelen'' in de actualiteit. Dit lukt niet op een avond. Het is efficiënter om hier gedurende een aantal weken regelmatig wat tijd in te steken. Op de duur zou je in principe de belangrijkste thema's moeten herkennen waar men op dit moment vooral mee bezig is en dat zou inspiratie kunnen geven voor je onderwerp.

In Hoofdstuk~\ref{ch:literatuuronderzoek} vind je enkele concrete tips en startpunten voor het zoeken van relevante informatie.

\section{Onderzoeksvraag en -doelstellingen formuleren}%
\label{sec:onderzoeksvraag_formuleren}

Eens je voeling krijgt met de actualiteit van een onderwerp, leer je typisch ook de belangrijkste problemen en discussiepunten kennen. Die kunnen aanleiding geven tot het formuleren van je hoofdonderzoeksvraag, die je verder kan opsplitsen in concretere deelonderzoeksvragen.

Een goede onderzoeksvraag voor een bachelorproef start vanuit een reële of in elk geval realistische \textit{bedrijfscasus}, een concreet probleem waar een specifiek bedrijf mee worstelt. Dat zorgt meteen ook voor een duidelijke \textit{afbakening} van het onderwerp. Een te vage of algemene onderzoeksvraag leidt nooit tot een goede bachelorproef.

Bijvoorbeeld, ``Wat is de beste agile methodologie voor bedrijven?'' is in dat opzicht \textit{geen} goede onderzoeksvraag. Er zijn verschillende agile methodologieën, en daar is allicht een goede reden voor. Er is geen duidelijke winnaar die in elke situatie het meest geschikt is. Als dat wel zo was, dan zou intussen, meer dan twintig jaar na de publicatie van het Agile Manifesto \autocite{BeckEtAl2001}, al duidelijk zijn wat de beste methodologie is. Een geschikte methodologie hangt af van veel factoren: de de bedrijfscultuur, het soort bedrijf, het soort it-oplossingen dat ontwikkeld wordt, de grootte van het bedrijf, enz. Als je je onderzoeksvraag te algemeen formuleert, zal je nooit tot een sluitende conclusie kunnen komen. Dit hebben we in het verleden al te vaak ondervonden. De student schrijft dan in de conclusie ``de beste oplossing hangt af van je voorkeur en je specifieke situatie.'' Met andere woorden, er \textit{is} geen conclusie. \textbf{Start dus vanuit een concrete casus}, bv.\ bedrijf XYZ worstelt met het tijdig en binnen budget opleveren van software en wil een agile methodologie gaan toepassen. Wat is voor hun specifieke situatie de meest geschikte methodologie? Daarmee is het onderwerp duidelijk afgebakend en de meerwaarde ervan is meteen zichtbaar.

Op goede onderzoeksvraag bestaat er nu nog geen sluitend antwoord. Vragen als ``Wat is data mining?'', ``Welke PHP frameworks zijn er?'', ``Wat is de geschiedenis van de Cloud?'' zijn dus niet geschikt, want het antwoord is snel te vinden door even te zoeken op Wikipedia of Google, of door relevante vakliteratuur te raadplegen.

Het is ook belangrijk om meteen na te denken over het verwachte resultaat, de \textit{onderzoeksdoelstelling}. Onder welke omstandigheden kan je spreken van een succesvolle bachelorproef? Wat is het concrete eindresultaat dat je wil bereiken?

Voor het formuleren van je onderzoeksdoelstelling (of -doelstellingen) kan je gebruik maken van het S.M.A.R.T.-principe \autocite{UchelenJungjohann2003}:

\begin{description}
  \item[Specifiek] Is de doelstelling eenduidig? Is het duidelijk wie er baat bij heeft? Bij een bachelorproef is dat vaak één specifiek bedrijf! Als je onderwerp voor ``bedrijven'' in het algemeen geschreven is, of zelfs voor de samenleving als geheel, dan moet je het onderwerp nog beter afbakenen.
  \item[Meetbaar] Onder welke (meetbare/observeerbare) voorwaarden of vorm is het doel bereikt? Is het doel om een proof-of-concept of prototype af te leveren? Een analyse voor een te ontwikkelen applicatie? Een vergelijking tussen mogelijke alternatieve oplossingen met een onderbouwd advies? Enz. Door de doelstelling zo concreet mogelijk te formuleren, wordt het ook makkelijker om effectief te starten met je onderzoek. Het is immers duidelijker welke informatie je nodig hebt om de onderzoeksvraag te beantwoorden.
  \item[Acceptabel] Zijn deze doelen acceptabel voor de doelgroep? Zal de doelgroep de voorgestelde oplossing ook daadwerkelijk kunnen gebruiken?
  \item[Realistisch] Is het doel haalbaar? Is het onderwerp voldoende afgebakend? Is de moeilijkheidsgraad van het probleem op het niveau van de professionele bachelor?
  \item[Tijdsgebonden] Wanneer (in de tijd) moet het doel bereikt zijn? Voor een bachelorproef is dit eenduidig vastgelegd: de deadline voor indienen.
\end{description}

\section{Onderwerp uitschrijven}%
\label{sec:onderwerp_uitschrijven}

Eens je een onderzoeksvraag hebt, kan je je onderwerp uitschrijven om het in te dienen ter goedkeuring. Dat betekent dat je ook al wat literatuuronderzoek gaat uitvoeren. Voor aanwijzingen over de aanpak hiervan, zie Hoofdstuk~\ref{ch:literatuuronderzoek}.

Wat je leest over het onderwerp, moet je dan structureren en formuleren in je eigen woorden in een doorlopende tekst. Een goed hulpmiddel om je gedachten over een onderwerp te ordenen om er later een gestructureerde tekst rond te schrijven is het opzetten van een mindmap. Er bestaan hiervoor verschillende tools die je kosteloos kan gebruiken, zoals bijvoorbeeld XMind\footnote{\url{https://www.xmind.net/}} of FreeMind\footnote{\url{http://freemind.sourceforge.net/}}.

In dit stadium is het nog niet de bedoeling een volledig uitgewerkte literatuurstudie uit te schrijven, maar een aantal referenties worden wel verwacht. Je promotor moet na het lezen van je voorstel begrijpen wat de context van je onderzoek is en waarom er een probleem is dat om een oplossing vraagt. Je moet dit kunnen aantonen aan de hand van gezaghebbende vakliteratuur.

Denk ook na over een titel voor je bachelorproef. Die hoeft nog niet definitief te zijn, maar een goede titel maakt duidelijk welke richting je precies wil uitgaan. Een concrete titel geeft je begeleiders het vertrouwen dat je weet wat je precies wil gaan doen en dat dit een realistische doelstelling is. Probeer te letten op het volgende bij het formuleren van een titel:

\begin{itemize}
  \item Formuleer de titel niet als een vraag.
  \item Vermijd obscuur vakjargon en gebruik zeker geen afkortingen. De titel moet ook begrijpbaar zijn voor iemand buiten jouw specifieke vakgebied.
  \item Enkel je vakdomein benoemen is onvoldoende want veel te vaag. Je titel moet concreet zijn en duidelijk maken wat je precies wil onderzoeken. Dat betekent dus ook dat een titel gerust lang mag zijn. Bijvoorbeeld, ``Cloud computing'' is een algemene term die veel ladingen dekt, en die dus niets zegt. ``De selectie van een open source Infrastructure as a Service platform voor het opzetten van een testomgeving voor webontwikkeling'' is een stuk concreter.
\end{itemize}

Bij het beoordelen van een onderwerp, wordt o.a.\ rekening gehouden met volgende criteria:

\begin{itemize}
  \item Er is een concrete, \textbf{duidelijk afgebakende onderzoeksvraag} en onderzoeksdoelstelling.
  \item Het voorgestelde onderzoek vertrekt vanuit een \textbf{concrete use case/een concreet probleem} vanuit het werkveld. 
  \item De \textbf{doelgroep} van het voorgestelde onderzoek is specifiek en duidelijk. 
  \item Het voorstel is \textbf{vernieuwend} en heeft een duidelijke \textbf{meerwaarde} voor een specifieke doelgroep uit het werkveld.
  \item Het onderzoek kan enkel worden uitgevoerd door iemand met een bachelor toegepaste informatica. Er is dus een duidelijke \textbf{technologische component} aanwezig.
  \item Het voorgestelde onderzoek vraagt om een \textbf{eigen bijdrage}. Er is een duidelijke onderzoekscomponent. Er worden geen a priorikeuzes gemaakt/alle alternatieven worden mee overwogen. 
  \item De \textbf{methodologie} is uitgewerkt en verantwoord. De onderzoekstechnieken zijn passend voor de centrale onderzoeksvraag. 
\end{itemize}

\section{Struikelblokken}%
\label{sec:onderwerp-struikelblokken}

Volgende zaken geven ons het idee dat je onderwerp nog onvoldoende is uitgewerkt of dat er nog belangrijke struikelblokken zijn die een succesvolle bachelorproef in de weg staan:

% TODO: onduidelijke doelgroep, of voor het grote publiek
\begin{itemize}
  \item Er is geen concrete reële of realistische \textbf{bedrijfscasus} gekoppeld aan het onderwerp. Onderzoek op bachelorniveau is toegepast, m.a.w.~we proberen concrete, reële problemen op te lossen. Dat moet ook in je onderwerp naar voor komen.

  \item Het is niet duidelijk voor \textbf{welke doelgroep} het onderwerp een meerwaarde kan bieden. In toegepast onderzoek is dat vaak één specifiek bedrijf.
  \item (Eén van) de onderzoeksdoelstelling(en) is \textbf{speculatie over de toekomst} (bv. ``wat zal de impact zijn van het Internet of Things op het dagelijks leven?''). De conclusie van een bachelorproef moet aantoonbaar zijn, toekomstvoorspellingen zijn dat nooit.

  \item Het resultaat van het onderzoek is in grote mate \textbf{afhankelijk van externe factoren}. Als je bijvoorbeeld een enquête zal voeren, is het belangrijk te beseffen dat het niet makkelijk is om voldoende respondenten te vinden. Je zal dus al meteen moeten aangeven op welke manier je van plan bent om een voldoende grote steekproef te nemen.
  
  Ook als je plan is om interviews af te nemen bij experten, bedrijven, enz.\ is het belangrijk om al voor de start van je onderzoek de nodige contacten te leggen. Als het immers niet lukt om tijdig de nodige personen te kunnen spreken, komt het resultaat van je bachelorproef in gevaar.

  Als je voor je onderzoek gebruik moet maken van specifieke hardware, commerciële diensten, enz., dan is het belangrijk om je er van te verzekeren dat je daar toegang toe kan krijgen, hetzij via een geïnteresseerd bedrijf, via een demo-licentie of op eigen kosten.

  \item Het onderwerp vertoont onvoldoende \textbf{technische diepgang}. De professionele bachelor toegepaste informatica is een technisch profiel en we verwachten dat je ook aantoont technisch sterk te staan. Een onderwerp als ``wat is de meest gebruiksvriendelijke mobiele applicatie voor thuisbankieren voor senioren?'' voldoet hier bijvoorbeeld \textit{niet} aan. Dit soort applicaties is wordt door banken ontwikkeld en als student zal je nooit toegang krijgen tot de interne werking. Je zal je wellicht moeten beperken tot interviews en/of enquêtes bij de doelgroep. Een student marketing zou dit soort onderwerp ook kunnen uitvoeren. Op deze manier kan je dus niet aantonen wat je als informaticus in je mars hebt.

  \item De \textbf{referentielijst} is te kort of bestaat uit ongeschikte bronnen. Neem de nodige tijd om een initiële literatuurstudie te voeren en ga verder kijken dan het eerste het beste blogartikel. In Hoofdstuk~\ref{ch:literatuuronderzoek} vind je aanwijzingen om dit aan te pakken.

  \item Het onderwerp ligt \textbf{buiten de scope van een bacheloropleiding of van de student}. Kies een onderwerp dat qua niveau past bij een bacheloropleiding. Een typisch voorbeeld van een te moeilijk onderwerp is \textit{Quantum Computing}, dit is minstens voor een masteropleiding. Een dergelijk onderwerp kan in het oog springen, maar hiermee loop je de kans om in je eigen voeten te schieten qua moeilijkheid.
\end{itemize}

\section{Samenvatting}%
\label{sec:onderwerp_samenvatting}

\begin{itemize}
  \item Neem de tijd om een onderwerp te zoeken dat je ligt.
  \item Bij het uitschrijven van een onderwerp ben je best zo \emph{concreet} mogelijk. Denk eraan dat toegepast onderzoek een oplossing zoekt voor één specifieke situatie. Onderwerpen die te algemeen geformuleerd zijn worden niet aanvaard.
  \item Vermijd struikelblokken als het ontbreken van een concrete casus, speculatie, afhankelijkheid van externe factoren en een te oppervlakkige literatuurstudie.
\end{itemize}
