\chapter*{Voorwoord}%
\label{ch:voorwoord}

Deze gids is tot stand gekomen vanuit de begeleiding van de bachelorproef van de opleiding toegepaste informatica aan Hogeschool Gent en werd oorspronkelijk gepubliceerd via Github\footnote{\url{https://github.com/HoGentTIN/bachproef-gids/}}. De bedoeling was om onze studenten wat houvast te geven in hoe ze hieraan in de eerste plaats moeten beginnen, en bevat heel wat tips i.v.m.\ methodologie, werken met {\LaTeX} voor een professionele opmaak, enz.

In het nieuwe curriculum, dat ingevoerd werd vanaf het academiejaar 2020--2021, is er een nieuw opleidingsonderdeel ingevoerd waar we hier expliciet tijd voor kunnen maken. De bachelorproefgids is dan verder uitgewerkt als de cursus voor dat nieuwe vak.

De doelstelling van de bachelorproef binnen onze opleiding is aan te tonen dat je in staat bent om een ``klant'' (binnen de eigen organisatie of extern) \textbf{advies te geven over het in een bedrijfscontext toepassen van een actuele ICT-technologie}. We beogen geen academisch wetenschappelijke studie, maar wel een toegepast onderzoek met voldoende technische diepgang dat een meerwaarde biedt in het gekozen vakgebied.

Dat houdt in dat je aan de slag gaat met een concreet en actueel probleem uit het werkveld. Misschien ben je nog geen expert in dit onderwerp, dus je moet je inwerken door relevante, objectieve en gezaghebbende informatie over dat onderwerp te verzamelen en kritisch te beschouwen. Enerzijds gaat het dan over het opzoeken van bestaande kennis die te vinden is in de vakliteratuur, via experten of belanghebbenden. Anderzijds verwachten we dat tijdens het onderzoek ook nieuwe kennis wordt gecreëerd, bijvoorbeeld via zelf opgezette experimenten, een proof-of-concept of prototype, interviews, enz. Al die informatie moet vervolgens gestructureerd en geanalyseerd worden, in het geval van kwantitatieve gegevens op een statistisch verantwoorde manier. Op basis daarvan kan je een oplossing uitwerken en je advies formuleren voor de klant of opdrachtgever. Dit alles verwerk je in een scriptie die je indient en verdedigt voor een jury.

Hopelijk biedt deze cursus voldoende houvast om je bachelorproef met succes aan te vatten!

\bigskip
Revisie: \today