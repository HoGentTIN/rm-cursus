\chapter*{Voorwoord}%
\label{ch:voorwoord}

Deze gids is tot stand gekomen vanuit de begeleiding van de bachelorproef van de opleiding toegepaste informatica aan Hogeschool Gent en werd oorspronkelijk gepubliceerd via Github\footnote{\url{https://github.com/HoGentTIN/bachproef-gids/}}. De bedoeling was om onze studenten wat houvast te geven in hoe ze hieraan in de eerste plaats moeten beginnen, en bevat heel wat tips i.v.m.\ methodologie, werken met {\LaTeX} voor een professionele opmaak, enz.

In het nieuwe curriculum, dat ingevoerd werd vanaf het academiejaar 2020--2021, is er een nieuw opleidingsonderdeel ingevoerd waar we hier expliciet tijd voor kunnen maken. De bachelorproefgids is dan verder uitgewerkt als de cursus voor dat nieuwe vak.

HOGENT heeft via Chamilo een cursus over informatievaardigheden gepubliceerd\footnote{\href{https://chamilo.hogent.be/index.php?go=CourseViewer&application=Chamilo\%5CApplication\%5CWeblcms&course=22068}{Kritisch denken en onderzoekscompetenties voor de bachelorstudent}} die je zeker ook eens moet doornemen. Het is immers niet de bedoeling om in deze gids dezelfde inhoud te herhalen.

De doelstelling van de bachelorproef binnen onze opleiding is aan te tonen dat je in staat bent om een ``klant'' (binnen de eigen organisatie of extern) \textbf{advies te geven over het in een bedrijfscontext toepassen van een actuele ICT-technologie}. We beogen geen academisch wetenschappelijke studie, maar wel een toegepast onderzoek met voldoende technische diepgang dat een meerwaarde biedt in het gekozen vakgebied.

Dat houdt in dat je aan de slag gaat met een concreet en actueel probleem uit het werkveld. Misschien ben je nog geen expert in dit onderwerp, dus je moet je inwerken door relevante, objectieve en gezaghebbende informatie over dat onderwerp te verzamelen en kritisch te beschouwen. Enerzijds gaat het dan over het opzoeken van bestaande kennis die te vinden is in de vakliteratuur, via experten of belanghebbenden. Anderzijds verwachten we dat tijdens het onderzoek ook nieuwe kennis wordt gecreëerd, bijvoorbeeld via zelf opgezette experimenten, een proof-of-concept of prototype, interviews, enz. Al die informatie moet vervolgens gestructureerd en geanalyseerd worden, in het geval van kwantitatieve gegevens op een statistisch verantwoorde manier. Op basis daarvan kan je een oplossing uitwerken en je advies formuleren voor de klant of opdrachtgever. Dit alles verwerk je in een scriptie die je indient en verdedigt voor een jury.

\textbf{Voor veel studenten is de bachelorproef een enorme uitdaging.} De stage lijkt vaak een stuk haalbaarder: je werkt onder supervisie van een mentor uit het stagebedrijf die je dagelijks opvolgt, de stage-opdracht is in principe duidelijk omlijnd, en vaak zijn keuzes qua gebruik van tools en technologieën al op voorhand beslist.

Bij de bachelorproef is dat allemaal anders: je moet (meestal) zelf een onderwerp kiezen, je moet zelf op zoek naar relevante informatie, je wordt niet dagelijks opgevolgd, je moet zelf keuzes maken (en kunnen motiveren!), je moet de resultaten samenvatten in een goed gestructureerd en professioneel geschreven rapport (de scriptie of bachelorproef), en je moet een verdediging voorbereiden. Dat is een hele uitdaging, maar het is ook een kans om jezelf te bewijzen en jezelf te ontwikkelen.

De intentie van alle lectoren die bij dit vak betrokken zijn is om je voldoende houvast te bieden om je bachelorproef met succes aan te vatten! Wij wensen jullie daar dan ook van harte veel succes mee!

\bigskip
Revisie: \today