\chapter{Werken met \LaTeX{}}%
\label{ch:voorbereiding}

In dit hoofdstuk behandelen we het opstarten van het werk aan een bachelorproef. Je vindt er enkele aanbevelingen over te gebruiken tools en het onderzoeksproces.

\section{Gebruik van {\LaTeX}}%
\label{sec:gebruik-van-latex}

De meeste studenten zijn gewend om opgemaakte tekst met een klassieke tekstverwerker (typisch MS Word) te schrijven. Voor je bachelorproef is het aangewezen om hier van af te stappen.

Word, zeker met het standaardsjabloon, geeft een layout die niet geschikt is voor publicatie. Eens de lengte en complexiteit van een Word-document toenemen (en bij een eindwerk is dat zeker het geval), krijg je te maken met inconsistenties in de layout van je tekst, paginanummering en slecht gepositioneerde afbeeldingen.

Wanneer je tekst kopieert vanuit een ander (voorbereidend) document of vanuit een website, wordt de oorspronkelijke layout overgenomen. Als die niet consistent is met deze van je hoofddocument, moet je alles gaan aanpassen. Dit is een bijzonder tijdrovend werk met een grote kans op fouten.

Een klassieke tekstverwerker die gebaseerd is op het WYSIWYG-principe\footnote{\emph{What You See Is What You Get}, zoals je wellicht weet}, laat je toe om tot in de puntjes te bepalen waar tekst op het papier terecht komt, maar eigenlijk is deze grote vrijheid in dit geval een nadeel. Een strakke en professionele vormgeving is een specialiteit die een grote aandacht voor vaak pietluttige  details vraagt. Als informaticus hebben wij niet de nodige kennis om dit te realiseren. Wanneer je een significant deel van de tijd bezig bent met het vormgeven van je document, word je bovendien afgeleid van de kern van de zaak: de inhoud van de tekst!

Een ander nadeel van de klassieke tekstverwerker is het binaire bestandsformaat. Dit maakt het onmogelijk om een document in een versiebeheersysteem te steken (zie Sectie~\ref{sec:versiebeheersysteem}). Al gauw gaan er verschillende versies van het document naast elkaar leven: `bachproef 3.docx', `bachproef 5 30 maart.docx', `final draft.docx', `final draft na feedback.docx', `final final draft.docx' \dots. Je hebt versies op je laptop, op dropbox, op je vaste pc en in je mailbox. Op de duur is het overzicht zoek, vergeet je stukken tekst over te nemen of maak je andere fouten.

Voor het opmaken van een lange tekst met een professionele en strakke vormgeving is {\LaTeX} een aanrader. Dit is \emph{tekstzetsysteem} met een markuptaal (zoals HTML) die ontworpen is voor het op papier zetten van tekst met een opmaak volgens de regels van de kunst. Je schrijft je tekst in de {\LaTeX} markuptaal, en een `compiler' zet deze om in een PDF. Een bijkomend voordel van {\LaTeX} is dat het tekstgebaseerd is, en je het dus kan opslaan in een versiebeheersysteem.

Toegegeven, {\LaTeX} heeft wel degelijk enkele nadelen. Er is een niet te onderschatten leercurve, en zolang je vasthoudt aan de gewoonten die je overgehouden hebt aan het werken met een tekstverwerker, doet {\LaTeX} niet altijd wat je verwacht. Maar je moet de meeste inspanning leveren in het begin, wanneer je {\LaTeX} nog onder de knie moet krijgen. Eens je het basissjabloon met succes kan compileren naar PDF, kan je stelselmatig inhoud toevoegen zonder dat je je nog zorgen hoeft te maken over de vormgeving. Die zal altijd strak en consistent blijven. Zaken zoals hoofdstuk- en paginanummering, verwijzen naar andere hoofdstukken, figuren of bronnen worden automatisch bijgehouden en aangepast wanneer nodig.

Bij het schrijven van een bachelorproef met een tekstverwerker heb je het mees\-te werk op het einde, om alle onvolkomenheden, inconsistenties en fouten in de vormgeving weg te werken. Op dat moment heb je daar meestal niet meer voldoende tijd voor, want de deadline nadert. Het gevolg is vrijwel altijd een document dat onvoldoende afgewerkt is en er heel onprofessioneel uitziet voor de lezer. Dit past niet niet bij een werkstuk dat dient als afsluiter van een \textit{professionele} bacheloropleiding.

In de rest van deze gids gaan we er van uit dat je {\LaTeX} gebruikt. Het is niet de bedoeling dat dit een {\LaTeX} handleiding wordt, daarvoor zijn er voldoende andere bronnen beschikbaar~\parencite{Oetiker2015}.

\section{Installatie}%
\label{sec:latex-installatie}

Om met {\LaTeX} te kunnen werken heb je het volgende nodig:

\begin{itemize}
  \item Een {\LaTeX} compiler. Afhankelijk van je besturingssysteem zijn de volgende goede keuzes:
  \begin{itemize}
    \item Windows: MiKTeX\footnote{\url{https://miktex.org}} of TeX Live\footnote{\url{https://tug.org/texlive/}}.
    \item macOS: MacTeX\footnote{\url{https://tug.org/mactex/}}.
    \item Linux: TeX Live.
  \end{itemize}
  \item Een {\LaTeX} IDE of editor met goede ondersteuning. Wij raden een van deze aan (omdat ze op alle besturingssystemen goed werken):
  \begin{itemize}
    \item TeXstudio\footnote{\url{https://www.texstudio.org}}.
    \item VS Code met de LaTeX Workshop extensie\footnote{\url{https://marketplace.visualstudio.com/items?itemName=James-Yu.latex-workshop}}.
  \end{itemize}
  \item Een tool voor het bijhouden van bibliografische referenties. Wij raden JabRef\footnote{\url{https://www.jabref.org}} aan (voor alle besturingssystemen).
  \item Een {\LaTeX}-sjabloon. Vanuit de opleiding bieden we enkele sjablonen aan:
  \begin{itemize}
    \item Voor de bachelorproef\footnote{\url{https://github.com/HoGentTIN/latex-hogent-bachproef}}
    \item Voor een korte tekst of artikel\footnote{\url{https://github.com/HoGentTIN/latex-hogent-article}} (dit sjabloon gebruik je voor het uitwerken van bachelorproefvoorstel voor Research Methods)
    \item Voor een presentatie\footnote{\url{https://github.com/HoGentTIN/latex-hogent-beamer}}
  \end{itemize}
  \item Lettertypes voor de HOGENT-huisstijl. Deze kan je downloaden en installeren via de Github-repos met de {\LaTeX}-sjablonen\footnote{Bv.\ \url{https://github.com/HoGentTIN/latex-hogent-bachproef/tree/main/fonts}}.
  \begin{itemize}
    \item Montserrat (officieel hoofdlettertype van de HOGENT huisstijl)
    \item Fira Code (monogespatieerde tekst)
    \item Fira Math (wiskundige formules)
  \end{itemize}
\end{itemize}

Lettertypes installeren kan je op zowel Windows, macOS als Linux door te dubbelklikken op het .otf-bestand en op de knop "Installeren" te klikken in het preview-venster. Je kan de bestanden ook rechtstreeks kopiëren naar de directory met lettertypes:

\begin{itemize}
  \item Windows: \verb+C:\Windows\Fonts+
  \item macOS: \verb+/Library/Fonts+ of \verb+~/Library/Fonts+
  \item Linux: \verb+/usr/share/fonts+ of \verb+~/.local/share/fonts+
\end{itemize}

\subsection{Installatie op Windows}%
\label{ssec:installatie-op-windows}

Als je de nodige software wil installeren op Windows, dan raden we aan om gebruik te maken van een package manager, bv.\ Chocolatey\footnote{\url{https://chocolatey.org}}:

\begin{minted}[linenos,breaklines,frame=single]{console}
choco install miktex
choco install texstudio
choco install jabref
\end{minted}

\subsection{Installatie op macOS}%
\label{ssec:installatie-op-macos}

Ook op macOS gebruik je best een package manager zoals Homebrew\footnote{\url{https://brew.sh}}:

\begin{minted}[linenos,breaklines,frame=single]{console}
brew install --cask mactex
brew install --cask texstudio
brew install --cask jabref
\end{minted}

\subsection{Installatie op Linux}%
\label{ssec:installatie-op-linux}

Linux-gebruikers zijn het meestal gewend om de package manager voor hun specifieke distributie te gebruiken. Op Debian/Ubuntu kan je bijvoorbeeld het volgende commando uitvoeren:

\begin{minted}[linenos,breaklines,frame=single]{console}
sudo apt install texlive-base texlive-latex-base texlive-latex-recommended texlive-bibtex-extra texlive-pictures texlive-fonts-recommended jabref texstudio
\end{minted}


\section{{\TeX}studio configureren}%
\label{sec:texstudio-configureren}

Zoals eerder vermeld raden we {\TeX}studio of VS Code aan als {\LaTeX}-editor. In deze sectie gaan we {\TeX}studio configureren. De configuratie van VS Code is iets complexer en zullen we hier niet behandelen. Kies na opstarten in het menu voor \textit{Options} en vervolgens voor \textit{Configure {\TeX}studio}.

De belangrijkste opties zijn:

\begin{itemize}
  \item Build:
    \begin{itemize}
      \item Default compiler: \textit{XeLaTeX}
      \item Default Bibliography tool: \textit{Biber}
    \end{itemize}
  \item Commands:
    \begin{itemize}
      \item xelatex:
      
        \begin{minted}[autogobble,linenos,breaklines,frame=single]{console}
          xelatex -shell-escape -synctex=1 -interaction=nonstopmode -file-line-error %
        \end{minted}

    \end{itemize}
  \item Editor:
    \begin{itemize}
      \item Indentation mode: Indent and Unindent Automatically
      \item Replace Indentation Tab by Spaces: Aanvinken
      \item Replace Tab in Text by spaces: Aanvinken
      \item Replace Double Quotes: English Quotes: ‘‘’’
    \end{itemize}
  \item Language Checking:
    \begin{itemize}
      \item Default language: selecteer "nl\_NL-Dutch" of een gelijkaardige optie voor Nederlands zodat je gebruik kan maken van spellingcontrole. Is deze optie niet beschikbaar? Je kan een woordenboekbestand voor OpenOffice of LibreOffice downloaden en installeren. Het dialoogvenster heeft links die je meteen naar de juiste pagina leiden.
    \end{itemize}
\end{itemize}

\section{Versiebeheersysteem}%
\label{sec:versiebeheersysteem}

Aan een informaticus hoeven hopelijk de voordelen van een versiebeheersysteem niet uitgelegd te worden? Gebruik altijd een versiebeheersysteem zoals Git om je werk bij te houden. Creëer ook een Github repository. Dit is enerzijds een goed backupsysteem (mits je regelmatig synchroniseert met Github), en anderzijds laat het je toe om je werk te delen met je promotor. Eén van de eigenschappen van een versiebeheer is dat het bij uitstek ontworpen is om wijzigingen in \emph{tekstbestanden} op te volgen. Binaire bestandsformaten zoals documenten van een klassieke tekstverwerker zijn hiervoor niet geschikt, wat een extra motivatie is voor het gebruik van \LaTeX{}.

Volgende zaken horen zeker thuis in je repository:

\begin{itemize}
  \item \LaTeX{}-broncode van de bachelorproef.
  \item in te voegen afbeeldingen.
  \item broncode van zelf geschreven scripts, benchmarks, experimenten, proof-of-concepts, enz. Dit maakt dat je experimenten makkelijker te reproduceren en te valideren zijn door derden.
  \item ruwe resultaten experimenten (in tekstformaat, bv. CSV), transcripties van interviews, enz.
  \item losse nota's, ideeën, enz. Gebruik hiervoor Markdown\footnote{\url{https://guides.github.com/features/mastering-markdown/}} en vermijd Word-do\-cu\-men\-ten.
\end{itemize}

Kortom, \emph{alle} artefacten die resulteren uit je onderzoek horen thuis in de repository. Voor werkdocumenten waar je opgemaakte tekst wenst, maar waarvoor \LaTeX{} overkill is gebruik je Markdown.

Wat hoort \emph{niet} thuis in je repository:

\begin{itemize}
  \item Hulpbestanden aangemaakt bij het compileren van \LaTeX{}. Je kan ervoor zorgen dat deze niet in een repository opgenomen worden door een \texttt{.gitignore}-bestand aan te maken\footnote{Bijvoorbeeld \url{https://github.com/github/gitignore/blob/master/TeX.gitignore}}. Het \LaTeX{}-sjabloon voor de bachelorproef is al goed ingesteld.
  \item Grote (binaire) bestanden zoals ISO's, virtuele machines (bv.\ OVA), enz.
  \item PDFs van de artikels/ebooks die je gelezen hebt (dit wordt beschouwd als ``herdistribueren'' en mag niet onder de auteurswetgeving).
  \item Binaire bestanden die vaak veranderen, bv. Word-documenten.
  \item Bestanden die automatisch gegenereerd worden uit code in Git, bv.\ gecompileerde code.
\end{itemize}

Een versiebeheersysteem wordt pas echt nuttig als je het goed gebruikt. Commit dus zo vaak mogelijk, schrijf duidelijke commit-boodschappen en synchroniseer regelmatig met Github!

Als je een van de HOGENT-sjablonen op Github wilt gebruiken, dan kan je deze best downloaden als ZIP via de groene knop rechtsboven. De repository klonen of een fork aanmaken is in dit geval niet aan te raden. Daarmee neem je immers ook de hele historiek van het sjabloon over en die is niet relevant voor jouw werk. Initialiseer vervolgens zelf een repository met \texttt{git init} en voeg de bestanden van het sjabloon toe met \texttt{git add} en \texttt{git commit}. Vervolgens kan je de repository koppelen aan een nieuwe repository op Github met \texttt{git remote add origin URL} en de inhoud van je lokale repository uploaden met \texttt{git push -u origin master}.

\section{Samenvatting}%
\label{sec:voorbereiding-samenvatting}

De kernpunten van dit hoofdstuk zijn:

\begin{itemize}
  \item Om een strakke, professionele opmaak te bekomen, schrijf je je scriptie beter in {\LaTeX} dan met een klassieke tekstverwerker;
  \item Gebruik een \emph{reference manager} voor het bijhouden van een bibliografische databank (JabRef is aanbevolen).
  \item Gebruik een versiebeheersysteem om al je werk in op te slaan (Git is aanbevolen);
\end{itemize}
