\chapter{Literatuuronderzoek}
\label{ch:literatuuronderzoek}

De eerste fase in elk onderzoek is typisch om een overzicht te schrijven van de huidige stand van zaken in het onderzoeksdomein. Daarvoor is het nodig om je te verdiepen in \emph{alles} wat over het onderwerp reeds geschreven is. Dit is het literatuuronderzoek. In elk verslag over onderzoek is het essentieel dat je elke bewering die je doet ook kan aantonen. Dat kan ofwel op basis van data die je zelf op een methodologisch correcte manier verzameld en geanalyseerd hebt (maar dat komt verder in deze gids aan bod), ofwel aan de hand van refererenties naar \emph{gezaghebbende} publicaties.

Dit hoofdstuk gaat dieper in op dit onderwerp: wat bereik je precies met een literatuurstudie, hoe kan je er aan beginnen, hoe kan je de bronnen die je vindt gestructureerd bijhouden en op welke manier gebruik je die dan in je tekst?

HOGENT heeft via Chamilo een cursus over informatievaardigheden gepubliceerd\footnote{\url{https://chamilo.hogent.be/index.php?go=CourseViewer\&application=Chamilo\%5CApplication\%5CWeblcms\&course=22068\&tool=LearningPath\&browser=Table\&tool_action=ComplexDisplay\&publication=980981}} die je zeker eens moet doornemen. Het is immers niet de bedoeling om in deze gids dezelfde inhoud te herhalen. Dit hoofdstuk spitst zich in de eerste plaats toe op literatuuronderzoek over een ICT-gerelateerd onderwerp, en het correct opmaken van een referentielijst met {\LaTeX} en JabRef (zie Sectie~\ref{sec:bibliografische-databank}).

\section{Doel van het literatuuronderzoek}
\label{sec:doel-literatuuronderzoek}

Het belangrijkste doel van een literatuuronderzoek is om vertrouwd te worden met het onderzoeksdomein en die kennis ook door te geven aan de lezers van je bachelorproef. Je gaat dus zoveel mogelijk informatie verzamelen en lezen over het onderwerp zodat je er eigenlijk alles over weet dat er op dit moment over te weten valt. Het is dan de bedoeling om alle voor je onderzoek relevante kennis die je op deze manier hebt opgedaan ook op een gestructureerde manier en in je eigen woorden samen te vatten in een doorlopende tekst. Dit vormt meestal het eerste hoofdstuk (of de eerste hoofdstukken) van je bachelorproef.

Aan de hand van de literatuurstudie geef je de lezer de nodige achtergrond om het onderwerp te begrijpen. Het is een inleiding op het onderwerp, en bespreekt de huidige stand van zaken. Je vermeldt wat de experten in het domein er over te zeggen hebben en welk onderzoek er in het verleden al over gedaan is (met uiteraard vermelding van de belangrijkste conclusies). Uit de literatuurstudie moet ook naar voor komen dat er nog hiaten in onze kennis zijn, dat er een probleem is dat om een oplossing vraagt. En dat is uiteraard precies het onderwerp van je bachelorproef.

\section{Soorten bronnen}
\label{sec:soorten-bronnen}

In het kader van een onderzoek kunnen we informatiebronnen ondeverdelen in deze drie categorieën:

\begin{description}
  \item[Primaire] kennis die je zelf vergaart tijdens je onderzoek. Bijvoorbeeld metingen uit experimenten, resultaten van enquêtes, transcripties van interviews, enz.
  \item[Secundaire] publicatie van kennis, onderzoek, enz.~door anderen. Bijvoorbeeld artikels in vaktijdschriften, boek, presentatie op een conferentie, enz.
  \item[Tertiaire] zoekindexen en encyclopedieën. Bijvoorbeeld Google Scholar, Web of Science, Elsevier ScienceDirect, Arxiv.org, Wikipedia, about.com, Webopedia, enz.
\end{description}

Wanneer in een tekst verwezen wordt naar de literatuur, dan gaat het telkens over \emph{secundaire} bronnen (ook \emph{publicaties} genoemd). Dat betekent ook dat primaire of tertiaire bronnen \emph{niet} kunnen. Er mag dus bijvoorbeeld nooit verwezen worden naar een Wikipedia-artikel, woordenboeken, enz. Tertiaire bronnen zijn wel goede \textit{startpunten} van de zoektocht naar geschikte publicaties (Zie Sectie~\ref{sec:op_zoek_naar_relevante_informatie}). Je kan ook niet verwijzen naar het verslag van een interview dat je gevoerd hebt, omdat dat ook niet gepubliceerd is en dus niet toegankelijk voor de lezer.

\subsection{Publicatievormen}
\label{sub:publicatievormen}

Kennis wordt doorgegeven en gepubliceerd in verschillende vormen. In deze sectie lijsten we de belangrijkste op en bespreken de betrouwbaarheid en objectiviteit van elk.

\paragraph{Artikel in wetenschappelijk tijdschrift}

Van een artikel, dat in een wetenschappelijk tijdschrift (of Eng. \emph{journal}) gepubliceerd wordt, mag je uitgaan dat er eerst een rigoreus verificatieproces aan vooraf gegaan is. Ingezonden artikels worden door de redacteurs van het tijdschrift doorgegeven aan andere experten in het vakgebied die de verantwoordelijkheid hebben de inhoud ervan op een onafhankelijke en anonieme manier te verifiëren. Dit noemt men in het Engels \emph{peer review}. Dit proces duurt typisch verschillende maanden en kan zelfs uitlopen tot een paar jaar. Publicaties in wetenschappelijke tijdschriften worden algemeen beschouwd als de meest betrouwbare en als er over jouw bachelorproefonderwerp te vinden zijn, is het ten zeerste aan te raden die te lezen. Nadelen is dat het niveau (vooral op vlak van wiskunde) typisch vrij hoog ligt en dus niet altijd even toegankelijk voor de gemiddelde bachelorstudent.

\paragraph{Artikel in een vaktijdschrift}

Vaktijdschriften zijn gericht op een professioneel publiek, dus dit zijn geen academische, wetenschappelijke teksten. Er gaat ook geen peer review-proces aan de publicatie van artikels vooraf. Typisch beslist een redacteur of een artikel al dan niet gepubliceerd wordt. Vaktijdschriften zijn ook meer en meer enkel online beschikbaar en we rekenen hieronder ook portaalsites rond een bepaald thema of vakgebied, zoals dZone\footnote{\url{https://dzone.com/}}, Informit\footnote{\url{https://www.informit.com/}}, enz.

Het is dan belangrijk te weten wie de auteur van het artikel is. Is dit een erkend vakexpert of een journalist? In het eerste geval is het artikel zeker bruikbaar als bron, maar in het andere moet je er toch eens kritisch naar kijken. Er is immers zelden garantie dat een journalist voldoende expertise in het onderwerp van zijn artikel heeft. Journalisten hebben ook andere drijfveren dan vakexperten en willen vooral dat hun artikel door zoveel mogelijk mensen gelezen wordt. Soms zullen ze de zaken dan wat sensationeler voorstellen dan ze eigenlijk zijn, of slaan ze de bal mis als het gaat over technische details.

\paragraph{Presentatie op een conferentie}

Zowel de wetenschappelijke als de professionele gemeenschap organiseren wereldwijd conferenties om met elkaar te overleggen, om nieuwe resultaten te presenteren en kennis door te geven. Typisch wordt er enkele maanden voor de start een oproep gedaan om onderwerpen voor presentaties voor te stellen. Bij een wetenschappelijke conferentie wordt dan meestal gevraagd om een uitgeschreven artikel in te dienen dat wordt beoordeeld via een peer-reviewproces dat meestal wel een stuk minder zwaar is dan voor een journal. Voor vakconferenties en voor sommige wetenschappelijke conferenties is een paragraaf tekst met een samenvatting van de inhoud (abstract) voldoende. Bij vakconferenties zal meestal een panel dat door de organisator is samengesteld de inzendingen beoordelen en de presentaties selecteren.

Na afloop van een conferentie wordt de inhoud van de geselecteerde presentaties gebundeld en gepubliceerd. Bij een wetenschappelijke conferentie is dat een (e-)boek dat bestaat uit de ingezonden artikels, wat men in het Engels de \emph{proceedings} noemt. Bij een vakconferentie gaat het meestal enkel over de gebundelde presentatieslides, of zetten de sprekers zelf hun slides op een website om presentaties te delen, zoals SlideShare of Speaker Deck. Meer en meer vakconferenties nemen sommige of alle presentaties op en maken die beschikbaar via bv.~Youtube of Vimeo, of via een website in eigen beheer.

Het loont de moeite om uit te zoeken welke conferenties er doorgaan rond het vakgebied waarbinnen jouw gekozen onderwerp past en zo mogelijk de presentaties te bekijken die er zijn doorgegaan.

Qua betrouwbaarheid benadert het niveau van artikels in \emph{conference proceedings} die van wetenschappelijke artikels in journals. Dit geldt minder voor vakconferenties, omdat er geen peer-review proces aan vooraf gaat. Je kan er wel de belangrijkste experts over een bepaald vakdomein leren kennen en veel bijleren over recente ontwikkelingen binnen het vakgebied. Weet dat onderzoekers veelal het onderzoek van hun conferentie op een licht aangepaste manier opnieuw publiceren in een hoger aangeschreven ``A1 tijdschrift''.

\paragraph{Thesis}

Thesissen zijn ook vaak interessant als informatiebronnen. De diepgang hangt hier grotendeels af van de opleiding waarvoor de thesis geschreven is: doctoraatsthesis (PhD thesis), masterthesis of bachelorproef. Deze teksten zijn geschreven onder begeleiding van een promotor die borg staat voor de kwaliteit van de inhoud. Als je een gepubliceerde thesis vindt, dan is die dus in principe nagelezen door een expert. Doctoraatsthesissen staan wat dat betreft ongeveer op het niveau van wetenschappelijke artikels. Meestal is het ook zo dat een of meerdere onderdelen van een doctoraatsthesis ook als artikels in een wetenschappelijk tijdschrift zijn gepubliceerd.

\paragraph{Boek of handleiding}

Ook bij boeken is het belangrijk om te weten wie de auteur is en in hoeverre die de autoriteit heeft om over een onderwerp te schrijven. In principe kan iedereen immers een boek uitgeven, en er is geen formele peer-review, dus geen onafhankelijke validatie van de inhoud.

% TODO: iets over handleidingen?

\paragraph{White paper}

Een \emph{white paper} is een rapport over een bepaald onderwerp dat als bedoeling heeft de lezer voldoende achtergrondinformatie te geven over dat onderwerp om het te begrijpen, beslissingen te nemen of een probleem op te lossen. In ons vakdomein worden white papers typisch uitgegeven door bedrijven die een product verkopen gerelateerd aan het behandelde onderwerp. Een bedrijf dat antivirussoftware verkoopt kan bijvoorbeeld white papers uitgeven over het beveiligen van computers, hoe wachtwoorden gekraakt worden, enz.

Het is belangrijk om te beseffen dat white papers meestal niet objectief zijn. De auteurs hebben iets te verkopen, dus het is voor hen belangrijk om het onderwerp op een zodanige manier te presenteren dat het aankopen van hun producten of diensten interessant lijkt. Een leverancier van beveiligingssoftware heeft er bijvoorbeeld belang bij om het probleem van cybercriminaliteit erger voor te stellen dan het in werkelijkheid is. De lezer die ongerust wordt over de toestand is meer geneigd om beveiligingssoftware te kopen.

Lees een white paper dan ook met een zeer kritische blik en tracht ook objectieve informatie uit andere bronnen te vinden.

\paragraph{Blogartikel}

Een blog is meestal (een deel van) een persoonlijke website waar de auteur regelmatig artikels publiceert rond een bepaald onderwerp en haar/zijn kennis deelt met anderen in hetzelfde vakgebied. Over ICT-gerelateerde onderwerpen zijn er duizenden blogs, en de kans is groot dat de belangrijkste experten binnen je gekozen onderwerp er één bijhouden.

Ook hier is het belangrijk om te achterhalen wie de auteur is en welke autoriteit die heeft over het onderwerp. Wanneer je een artikel vindt over bijvoorbeeld ``continuous delivery'' van Martin Fowler, een wereldwijd erkend expert en spreker rond software-ontwikkeling, dan is dit heel bruikbaar als bron. Een artikel over hetzelfde onderwerp door een andere auteur die bijvoorbeeld na wat zoeken op LinkedIn een marketeer blijkt te zijn, neem je best niet op in je literatuurlijst.

\subsection{De kwaliteit van bronnen beoordelen}
\label{sub:de-kwaliteit-van-bronnen-beoordelen}

Uit de voorgaande sectie zou je al moeten opgevallen zijn dat de kwaliteit van bronnen niet altijd makkelijk te evalueren is. Veel hangt af van wie de auteur is en welke autoriteit die heeft binnen het vakgebied.

Een hulpmiddel bij het beoordelen van een bron is de \emph{CRAP-test} \autocite{Gratz2015}:

\begin{description}
  \item[Currency] of actualiteit: is de bron voldoende recent voor het onderwerp?
  \item[Reliability/Relevance] of betrouwbaarheid/relevantie: is de inhoud goed onderbouwd? Wordt er naar bronnen verwezen? Is de inhoud relevant voor jouw onderzoek?
  \item[Authority] of autoriteit: is de auteur een autoriteit over het onderwerp? Gaat het over een persoon of een organisatie?
  \item[Point of view] of objectiviteit: wat is de intentie van de auteur? Wat wil die bereiken?
\end{description}

Bij het beoordelen van een bron is het dan uiteraard noodzakelijk om te weten te komen wie de auteur is en wanneer die geschreven en gepubliceerd is. Van vele websites is dit jammer genoeg niet mogelijk en wordt deze informatie niet gegeven. Dit soort bronnen hoort niet thuis in een literatuurlijst!

\section{Samenvatting}
\label{sec:literatuuronderzoek_samenvatting}

\begin{itemize}
  \item Het doel van de literatuurstudie is om jezelf en de lezer van je bachelorproef voldoende context te geven om het onderwerp ten gronde te begrijpen.
  \item Er zijn drie soorten bronnen: primaire (resultaten van eigen onderzoek), secundaire (publicaties in wetenschappelijke of vakliteratuur) en tertiaire (encyclopedieën en zoekindexen).
  \item In een bibliografie horen enkel \emph{secundaire} bronnen thuis.
\end{itemize}
