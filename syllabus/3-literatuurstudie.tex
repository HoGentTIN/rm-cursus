\chapter{Literatuuronderzoek}%
\label{ch:literatuuronderzoek}

De eerste fase in elk onderzoek is typisch om een overzicht te schrijven van de huidige stand van zaken in het onderzoeksdomein. Daarvoor is het nodig om je te verdiepen in \emph{alles} wat over het onderwerp reeds geschreven is. Dit is het literatuuronderzoek. In elk verslag over onderzoek is het essentieel dat je elke bewering die je doet ook kan aantonen.

Dit hoofdstuk spitst zich in de eerste plaats toe op literatuuronderzoek over een ICT-gerelateerd onderwerp. Wat bereik je precies met een literatuurstudie, hoe kan je er aan beginnen, hoe kan je de bronnen die je vindt gestructureerd bijhouden en op welke manier gebruik je die dan in je tekst? 

In het volgende hoofstuk gaan we in op het correct opmaken van een referentielijst met {\LaTeX} en JabRef (zie Hoofdstuk~\ref{ch:bibliografie}).

\section{Doel van het literatuuronderzoek}%
\label{sec:doel-literatuuronderzoek}

Het belangrijkste doel van een literatuuronderzoek is om vertrouwd te worden met het onderzoeksdomein en die kennis ook door te geven aan de lezers van je bachelorproef. Je gaat dus zoveel mogelijk informatie verzamelen en lezen over het onderwerp zodat je er eigenlijk alles over weet dat er op dit moment over te weten valt. Het is dan de bedoeling om alle voor je onderzoek relevante kennis die je op deze manier hebt opgedaan ook op een gestructureerde manier en in je eigen woorden samen te vatten in een doorlopende tekst. Dit vormt meestal het eerste hoofdstuk (of de eerste hoofdstukken) van je bachelorproef.

De literaturstudie vormt een inleiding op het onderwerp, en bespreekt de huidige stand van zaken. Je introduceert alle relevante vaktermen, vermeldt wat de experten in het domein er over te zeggen hebben en welk onderzoek er in het verleden al eventueel over gedaan is (met uiteraard vermelding van de belangrijkste conclusies). Uit de literatuurstudie moet ook naar voor komen dat er nog hiaten in onze kennis zijn, dat er een probleem is dat om een oplossing vraagt. En dat is uiteraard precies het onderwerp van je bachelorproef.

Op het einde van dit hoofdstuk zou de lezer in staat moeten zijn om de volledige inhoud van je bachelorproef te begrijpen, zonder nog een externe bron te moeten raadplegen. Ga er daarbij van uit dat je bachelorproef gelezen wordt door een ict-professional. Die persoon heeft niet noodzakelijk een achtergrond in het specifieke onderwerp van je bachelorproef, maar heeft wel algemene vakkennis én is typisch geïnteresseerd in de technische details.

\section{Soorten bronnen}%
\label{sec:soorten-bronnen}

In het kader van een onderzoek kunnen we informatiebronnen onderverdelen in deze drie categorieën:

\begin{description}
  \item[Primaire] kennis die je zelf vergaart tijdens je onderzoek. Bijvoorbeeld metingen uit experimenten, resultaten van enquêtes, transcripties van interviews, enz.
  \item[Secundaire] publicatie van kennis, onderzoek, enz.~door anderen. Bijvoorbeeld artikels in vaktijdschriften, boek, presentatie op een conferentie, enz.
  \item[Tertiaire] zoekindexen en encyclopedieën. Bijvoorbeeld Google Scholar, Web of Science, Elsevier ScienceDirect, Arxiv.org, Wikipedia, about.com, Webopedia, Techopedia, Investopedia, enz.
\end{description}

Wanneer in een tekst verwezen wordt naar de literatuur, dan gaat het telkens over \emph{secundaire} bronnen (ook \emph{publicaties} genoemd). Dat betekent ook dat primaire of tertiaire bronnen \emph{niet} kunnen. Er mag dus bijvoorbeeld nooit verwezen worden naar een Wikipedia-artikel, woordenboeken, enz. Tertiaire bronnen zijn wel goede \textit{startpunten} van de zoektocht naar geschikte publicaties (Zie Sectie~\ref{sec:op_zoek_naar_relevante_informatie}).

Een bron moet ook toegankelijk zijn voor de lezer, m.a.w.\ deze moet \textit{gepubliceerd} zijn. Daarom kan je ook \textit{niet} verwijzen naar:

\begin{itemize}
  \item Het verslag van een interview dat je gevoerd hebt, een notitie die je gemaakt hebt tijdens een vergadering, een notitie die je gemaakt hebt tijdens een brainstormsessie, enz.
  \item Een e-mail die je ontvangen hebt, een chatbericht, een sms-bericht, een tweet, een Facebook-post, enz.
  \item De syllabus of slides van een cursus die je tijdens je opleiding gevolgd hebt
  \item Een intern document van het bedrijf dat als opdrachtgever van je onderzoek fungeert.
\end{itemize}

\subsection{De kwaliteit van bronnen beoordelen}%
\label{sub:de-kwaliteit-van-bronnen-beoordelen}

Het is belangrijk om jouw tekst te onderbouwen met bronnen die betrouwbaar en gezaghebbend zijn. Dit is niet altijd makkelijk te evalueren. Veel hangt af van wie de auteur is en welke autoriteit die heeft binnen het vakgebied, wanneer de bron gepubliceerd is, wat de bedoeling was van de publicatie, enz.

Een hulpmiddel bij het beoordelen van een bron is de \emph{CRAAP-test} \autocite{Blakeslee2004}:

\begin{description}
  \item[Currency] of actualiteit: is de bron voldoende recent voor het onderwerp? Let op! Het is niet mogelijk om hier een exact getal op te plakken. Afhankelijk van het onderwerp kan een bron van één jaar geleden al verouderd zijn terwijl een publicatie van twintig jaar geleden nog steeds relevant is!

  \item[Relevance] of relevantie: is de inhoud relevant voor jouw onderzoek? Bevat het informatie die je nodig hebt? Let ook op het doelpubliek. Is het een vakliteratuur, specifiek bestemd voor ict-professionals (dit is waar je naar op zoek bent!), of slechts een populariserende tekst, geschreven voor een groot publiek (deze vermijd je best!)?

  \item[Authority] of autoriteit: is de auteur een autoriteit over het onderwerp? Is het een vakexpert, marketeer, journalist, \ldots? Gaat het over een persoon of een organisatie? In het laatste geval: is de organisatie een overheidsinstantie, een commercieel bedrijf, een non-profit organisatie, een vakorganisatie, \ldots?

  \item[Accuracy] of correctheid: is de bron betrouwbaar? Verwijst de tekst zelf naar bronnen? Zijn de beweringen die gemaakt worden verifieerbaar en goed onderbouwd? Is ook het taalgebruik en de schrijfstijl correct en zakelijk, professioneel?

  \item[Purpose] of doel: wat is de intentie van de tekst? Informeren, aanleren, verkopen, vermaken, onderzoeken, \ldots Wat wil de auteur bereiken? Is de inhoud neutraal of is er een duidelijke mening?
\end{description}

Bij het beoordelen van een bron is het dan noodzakelijk om te weten te komen wie de auteur is en wanneer die geschreven en gepubliceerd is. Van vele websites is dit jammer genoeg niet mogelijk en wordt deze informatie niet gegeven. Dit soort bronnen hoort niet thuis in een literatuurlijst!

\subsection{Publicatievormen}%
\label{sub:publicatievormen}

Kennis wordt doorgegeven en gepubliceerd in verschillende vormen. In deze sectie lijsten we de belangrijkste op en bespreken de betrouwbaarheid en objectiviteit van elk.

Een goede literatuustudie maakt gebruik van een gevarieerde mix van bronnen. Beperk je dus zeker niet tot de eerste de beste zoekresultaten die je via Google gevonden hebt, maar ga actief op zoek naar kwalitatieve vakliteratuur.Over het algemeen getuigt een bronnenlijst met enkel of vooral blogartikels of webpagina's van een zekere gemakzucht!

In de ICT moet je ook rekening houden met het feit dat de meest actuele literatuur in het Engels geschreven is. Enkel Nederlandstalige werken in een bronnenlijst is dus ook een indicatie van een onzorgvuldige aanpak.

\paragraph{Artikel in wetenschappelijk tijdschrift}

Van een artikel, dat in een wetenschappelijk tijdschrift (of Eng. \emph{journal}) gepubliceerd wordt, mag je uitgaan dat er eerst een rigoreus verificatieproces aan vooraf gegaan is. Ingezonden artikels worden door de redacteurs van het tijdschrift doorgegeven aan andere experten in het vakgebied die de verantwoordelijkheid hebben de inhoud ervan op een onafhankelijke en anonieme manier te verifiëren. Dit noemt men in het Engels \emph{peer review}. Dit proces duurt typisch verschillende maanden en kan zelfs uitlopen tot een paar jaar. Publicaties in wetenschappelijke tijdschriften worden algemeen beschouwd als de meest betrouwbare en als er over jouw bachelorproefonderwerp te vinden zijn, is het ten zeerste aan te raden die te lezen. Nadelen is dat het niveau (vooral op vlak van wiskunde) typisch vrij hoog ligt en dus niet altijd even toegankelijk voor de gemiddelde bachelorstudent. Ook is er over sommige, erg praktijkgerichte onderwerpen, geen wetenschappelijke literatuur te vinden.

\paragraph{Artikel in een vaktijdschrift}

Vaktijdschriften zijn gericht op een professioneel publiek, dus dit zijn geen academische, wetenschappelijke teksten. Er gaat ook geen peer review-proces aan de publicatie van artikels vooraf. Typisch beslist een redacteur of een artikel al dan niet gepubliceerd wordt. Vaktijdschriften zijn ook meer en meer enkel online beschikbaar en we rekenen hieronder ook portaalsites rond een bepaald thema of vakgebied, zoals dZone\footnote{\url{https://dzone.com/}}, Informit\footnote{\url{https://www.informit.com/}}, enz.

Het is dan belangrijk te weten wie de auteur van het artikel is. Is dit een erkend vakexpert of een journalist? In het eerste geval is het artikel zeker bruikbaar als bron, maar in het andere moet je er toch eens kritisch naar kijken. Er is immers zelden garantie dat een journalist voldoende expertise in het onderwerp van zijn artikel heeft. Journalisten hebben ook andere drijfveren dan vakexperten en kunnen onder druk staan om een artikel zo te schrijven dat het door zoveel mogelijk mensen gelezen wordt. Soms zullen ze de zaken dan wat sensationeler voorstellen dan ze eigenlijk zijn, of slaan ze de bal mis als het gaat over technische details.

\paragraph{Kranten- of tijdschriftartikel}

Artikels die in kranten of algemene tijdschriften verschijnen zijn in principe minder geschikt als bron voor een bacheloproef in de ict. Ze zijn niet bestemd voor een professioneel publiek, dus ontbreekt de nodige technische diepgang. 

Dit soort artikels is vaak geschreven door journalisten die niet noodzakelijk over de nodige expertise beschikken. Journalisten hebben ook niet de gewoonte om hun bronnen expliciet te vermelden. Het is dan ook belangrijk om kritisch te zijn bij het lezen van dergelijke artikels!

Wanneer je in je tekst een vermelding maakt van een gebeurtenis waar in de actualiteit veel aandacht is voor geweest (bv.\ een cyberaanval, het verschijnen van een revolutionaire nieuwe technologie, product of applicatie, \ldots), dan kan je uiteraard wel verwijzen naar kranten- of tijdschriftartikels die daarover gaan. Je kan dan zelf een kritische beschouwing toevoegen waar het artikel eventueel tekort schoot in het geven van een volledig beeld van de zaak, of waar ze zelfs de bal misgeslagen hebben.

\paragraph{Presentatie op een conferentie}

Zowel de wetenschappelijke als de professionele gemeenschap organiseren wereldwijd conferenties om met elkaar te overleggen, om nieuwe resultaten te presenteren en kennis door te geven. Typisch wordt er enkele maanden voor de start een oproep gedaan om onderwerpen voor presentaties voor te stellen. Bij een wetenschappelijke conferentie wordt dan meestal gevraagd om een uitgeschreven artikel in te dienen dat wordt beoordeeld via een peer-reviewproces dat meestal wel een stuk minder zwaar is dan voor een journal. Voor vakconferenties en voor sommige wetenschappelijke conferenties is een paragraaf tekst met een samenvatting van de inhoud (abstract) voldoende. Bij vakconferenties zal meestal een panel dat door de organisator is samengesteld de inzendingen beoordelen en de presentaties selecteren.

Na afloop van een conferentie wordt de inhoud van de geselecteerde presentaties gebundeld en gepubliceerd. Bij een wetenschappelijke conferentie is dat een {(e-)boek} dat bestaat uit de ingezonden artikels, wat men in het Engels de \emph{proceedings} noemt. Bij een vakconferentie gaat het meestal enkel over de gebundelde presentatieslides, of zetten de sprekers zelf hun slides op een website om presentaties te delen, zoals SlideShare of Speaker Deck. Meer en meer vakconferenties nemen sommige of alle presentaties op en maken die beschikbaar via bv.~Youtube of Vimeo, of via een website in eigen beheer.

Het loont de moeite om uit te zoeken welke conferenties er doorgaan rond het vakgebied waarbinnen jouw gekozen onderwerp past en zo mogelijk de presentaties te bekijken die er zijn doorgegaan.

Qua betrouwbaarheid benadert het niveau van artikels in \emph{conference pro\-ceed\-ings} die van wetenschappelijke artikels in journals. Dit geldt minder voor vakconferenties, omdat er geen peer-review proces aan vooraf gaat. Je kan er wel de belangrijkste experts over een bepaald vakdomein leren kennen en veel bijleren over recente ontwikkelingen binnen het vakgebied.

\paragraph{Thesis, eindwerk}

Thesissen of eindwerken zijn ook vaak interessant als informatiebronnen. De diepgang hangt hier grotendeels af van de opleiding waarvoor de thesis geschreven is: doctoraatsthesis (PhD thesis), masterthesis of bachelorproef. Deze teksten zijn geschreven onder begeleiding van een promotor die borg staat voor de kwaliteit van de inhoud. Als je een gepubliceerde thesis vindt, dan is die dus in principe nagelezen door een expert.

Doctoraatsthesissen staan wat dat betreft ongeveer op het niveau van wetenschappelijke artikels. Meestal is het ook zo dat een of meerdere onderdelen van een doctoraatsthesis ook als artikels in een wetenschappelijk tijdschrift zijn gepubliceerd.

Een masterthesis is geschreven door een student tijdens een masteropleiding, onder begeleiding van een professor of assistent. Je mag er dus van uitgaan dat de inhoud nagekeken is en dat deze voldoende kwaliteit heeft om te worden gepubliceerd. Het moeilijkheidsniveau van een masterthesis ligt meestal een stuk lager dan dat van een doctoraatsthesis.

Misschien bouwt jouw bachelorproefonderzoek verder op een onderwerp dat al eerder is behandeld door een bachelorstudent. Het is sowieso interessant om eens op zoek te gaan of je bachelorproeven van de vorige jaren kan terugvinden over jouw onderwerp. Ook een bachelorproef wordt geschreven onder begeleiding van een promotor, en ook hier geldt dat je er van mag uitgaan dat de inhoud nagekeken is en voldoende kwaliteit heeft om te worden gepubliceerd. Aan HOGENT is het zo dat bachelorproeven die 15/20 of meer halen, worden gepubliceerd via de bibliotheek. Via de bibliotheek kan je dus ook op zoek gaan naar bachelorproeven over jouw onderwerp.

\paragraph{Boek of handleiding}

Over de meeste onderwerpen waarover een bachelorproef toegepaste informatica geschreven wordt, is er wel een boek geschreven. Het is ook belangrijk om de moeite te doen om een boek te lezen over een onderwerp dat je wil behandelen. En boek bevat immers al een gestructureerde samenvatting van de kennis die er over dat onderwerp bestaat. Je vindt er ook vaak definities van vaktermen die je kan gebruiken in je eigen tekst. Soms bevat een boek ook een bronnenlijst die je kan gebruiken om je eigen tekst te onderbouwen.

Ook bij boeken is het belangrijk om te weten wie de auteur is en in hoeverre die de autoriteit heeft om over een onderwerp te schrijven. In principe kan iedereen immers een boek uitgeven, en er is geen formele peer-review, dus geen onafhankelijke validatie van de inhoud.

Bepaalde uitgeverijen zoals O'Reilly, Apress, Pearson, enz.\ staan bekend voor het uitgeven van technische boeken. Deze uitgeverijen hebben een reputatie opgebouwd door alleen boeken uit te geven die voldoen aan een bepaalde kwaliteitsnorm. Ze hebben ook een uitgebreid netwerk van auteurs die zeer bekend zijn in hun vakgebied. Als je een boek van een dergelijke uitgeverij vindt, dan kan je er van uitgaan dat het een degelijk boek is.

Als je tijdens je onderzoek gebruik maakt van specifieke software, of je onderwerp is een vergelijkende studie tussen verschillende applicaties, dan is het belangrijk om de handleidingen van die software te raadplegen, én er ook naar te refereren. Handleidingen worden tegenwoordig meestal online gepubliceerd, maar soms kan je ze ook vinden in de vorm van een boek.

\paragraph{White paper}

Een \emph{white paper} is een rapport over een bepaald onderwerp dat als bedoeling heeft de lezer voldoende achtergrondinformatie te geven over dat onderwerp om het te begrijpen, beslissingen te nemen of een probleem op te lossen. In ons vakdomein worden white papers typisch uitgegeven door bedrijven die een product verkopen gerelateerd aan het behandelde onderwerp. Een bedrijf dat antivirussoftware verkoopt kan bijvoorbeeld white papers uitgeven over het beveiligen van computers, hoe wachtwoorden gekraakt worden, enz.

Het is belangrijk om te beseffen dat white papers meestal niet objectief zijn. De auteurs hebben iets te verkopen, dus het is voor hen belangrijk om het onderwerp op een zodanige manier te presenteren dat het aankopen van hun producten of diensten interessant lijkt. Een leverancier van beveiligingssoftware heeft er bijvoorbeeld belang bij om het probleem van cybercriminaliteit erger voor te stellen dan het in werkelijkheid is. De lezer die ongerust wordt over de toestand is meer geneigd om beveiligingssoftware te kopen.

Lees een white paper dan ook met een zeer kritische blik en tracht ook objectieve informatie uit andere bronnen te vinden.

\paragraph{Blogartikel}

Een blog is meestal (een deel van) een persoonlijke website waar de auteur regelmatig artikels publiceert rond een bepaald onderwerp en haar/zijn kennis deelt met anderen in hetzelfde vakgebied. Over ICT-gerelateerde onderwerpen zijn er duizenden blogs, en de kans is groot dat de belangrijkste experten binnen je gekozen onderwerp er één bijhouden.

Ook bedrijven hebben soms een technische blog waar ze artikels publiceren over problemen of uitdagingen die ze zijn tegengekomen en hoe ze deze aangepakt hebben. Denk bv.\ aan de google Developers Blog\footnote{\url{https://developers.googleblog.com}}, Twitter Engineering Blog\footnote{\url{https://blog.twitter.com/engineering/}}, Netflix Tech Blog\footnote{\url{https://netflixtechblog.com}}, AWS Blog\footnote{\url{https://aws.amazon.com/blogs/}}, \dots

Ook hier is het belangrijk om te achterhalen wie de auteur is en welke autoriteit die heeft over het onderwerp. Wanneer je een artikel vindt over bijvoorbeeld ``continuous delivery'' van Martin Fowler, een wereldwijd erkend expert en spreker rond software-architectuur, dan is dit heel bruikbaar als bron. Een artikel over hetzelfde onderwerp door een andere auteur die bijvoorbeeld na wat zoeken op LinkedIn een marketeer blijkt te zijn, neem je best niet op in je literatuurlijst.

\section{Goede startpunten voor ICT vakliteratuur}%
\label{sec:startpunten-ict-vakliteratuur}

In deze sectie geven we een aantal goede startpunten voor het zoeken naar ICT-gerelateerde vakliteratuur.

\subsection{De HOGENT bib}%
\label{ssec:hogent-bib}

De HOGENT bib heeft als taak om kwalitatieve informatiebronnen beschikbaar te maken voor onderzoekers, docenten en studenten. Ga dus zeker naar de website\footnote{\url{https://www.hogent.be/student/bibliotheken/}} kijken of ga persoonlijk langs en vraag de medewerkers om raad.

Je kan eerst en vooral zoeken in de catalogus van de bibliotheek zelf. De kans is eerlijk gezegd relatief klein dat je op deze manier een relevant en voldoende recent boek vindt, maar wat je op deze manier wél kan vinden is bachelorproeven over jouw onderwerp die al eerder geschreven zijn. Deze proeven zijn een goede bron van inspiratie en kunnen je helpen om je eigen onderzoeksvraag af te bakenen of om er op verder te bouwen.

Verder kan je ook zoeken naar vaktijdschrijften of externe databanken voor vakliteratuur. De bib heeft een abonnement op een aantal van deze databanken, en je kan er als student gratis toegang toe krijgen. Verderop geven we enkele goede databanken voor ICT-gerelateerde vakliteratuur.

Een abonnementen tot zo'n databank wordt automatisch geactiveerd als je op het campusnetwerk zit. Dat betekent dat je vanop de campus in principe nergens hoeft in te loggen om toegang te krijgen en dat je publicaties meteen kan downloaden als PDF, EPUB, enz.

Als je thuis of een andere locatie aan het werk bent, dan kan je VPN aanzetten zodat het lijkt alsof je op de campus zit. De website van de IT-helpdesk geeft instructies over hoe je de VPN-client moet installeren.

Een andere mogelijkheid is om met je HOGENT-emailadres en wachtwoord in te loggen op de website van Academic Software\footnote{\url{https://www.academicsoftware.eu/}}, en in het aanbod te zoeken naar ``BIB bronnen HOGENT''. Selecteer de HOME USE-versie en klik op Starten. Er zal een nieuw tabblad geopend worden met wat lijkt op een webbrowser binnen jouw webbrowser met de hoofdpagina van de HOGENT-bib. Dit is effectief een virtuele omgeving die vanop de campus wordt opgestart en van waar je dus de volledige toegang hebt tot alle databanken. Je kan ook via deze weg publicaties downloaden naar je eigen pc. Deze werkwijze is iets omslachtiger in gebruik dan VPN, maar je hoeft er niets voor te installeren.

\subsection{Startpunten voor ``wetenschappelijke'' literatuur}%
\label{ssec:startpunten-wetenschappelijke-literatuur}

De professionele bacheloropleiding toegepaste informatica is vooral een technische en praktijkgerichte opleiding en niet zozeer een academische. Toch mag je als ict'er academische, wetenschappelijke bronnen niet uit de weg gaan. Nu zal je voor het ene vakgebied (zoals data engineering) ongetwijfeld meer academische literatuur vinden dan het andere (zoals mobiele applicatieontwikkeling), maar doe in elk geval een poging!

Als je op zoek bent naar wetenschappelijke literatuur, dan kan je beginnen met de volgende startpunten:

\paragraph{Google Scholar}

Google Scholar\footnote{\url{https://scholar.google.com/}} is een zoekmachine die zich richt op wetenschappelijke literatuur. Je kan er zoeken naar artikels die verschenen zijn in wetenschappelijke journals of gepresenteerd op conferenties (InProceedings) boeken, proefschriften, enz. Als je vanop de campus of via VPN werkt, dan word in de zoekresultaten meteen getoond of de publicatie beschikbaar is via een van de abonnementen van de HOGENT bib. Je ziet dit aan de links met de tekst ``[Fulltext@HOGENT]''.

\paragraph{Elicit}

Elicit\footnote{\url{https://www.elicit.be/}} is een AI assistent die je kan helpen bij het zoeken naar wetenschappelijke literatuur. Je kan er vragen stellen in natuurlijke taal, en Elicit zal je een lijst van relevante publicaties tonen. Elicit is vooral handig als je een onderwerp wil verkennen en niet precies weet waar je moet beginnen.

\paragraph{Elsevier ScienceDirect}

Elsevier is één van de grootste uitgeverijen van wetenschappelijke publicaties. ScienceDirect\footnote{\url{https://www.sciencedirect.com}} is een zoekmachine in hun catalogus van journals, conference proceedings, enz. Ook hier krijg je meteen toegang tot de volledige tekst vanaf de campus of via VPN. ScienceDirect bevat een grote hoeveelheid publicaties over ict en computerwetenschappen.

\paragraph{Springer Link}

Springer is een andere grote uitgeverij van wetenschappelijke publicaties. Springer Link\footnote{\url{https://link.springer.com/}} is de zoekmachine voor hun catalogus. Deze is bij uitstek interessant voor informatici omdat ook de boeken van de uitgeverij Apress langs deze weg toegankelijk zijn. Daarover later meer.

\paragraph{Andere (commerciële) databanken}

Er bestaan nog andere databanken, maar deze zijn ofwel veel breder dan enkel ict, of de HOGENT bib heeft er geen abonnement voor:

\begin{itemize}
  \item Web of Science\footnote{\url{https://www.webofscience.com/}} is zo'n brede databank (mét abonnement).
  \item Via IEEE Explore\footnote{\url{https://ieeexplore.ieee.org}} van het Institute of Electrical and Electronics Engineers (IEEE, een vakorganisatie) kan je ook enorm veel artikels over computerwetenschappen vinden. Jammer genoeg heeft HOGENT hier \textit{geen} abonnement voor en zijn vele bronnen niet beschikbaar voor ons.
  \item ACM Digital Library\footnote{\url{https://dl.acm.org}} van de Association for Computing Machinery, is eveneens een relevante databank die je ongetwijfeld zal tegenkomen in zoekresultaten, maar waar we geen toegang toe hebben.
\end{itemize}

Via de website van de HOGENT bib kan je nog meer databanken vinden: klik op de homepagina van de bib op ``Databanken'', en dan onder de kop ``Overzicht databanken per departement'' op ``IT en Digitale Innovatie'' en ``Toegepaste informatica''. Let er wel op dat deze soms zeer breed zijn, niet echt gericht op ict of minder geschikt als bron voor een bachelorproef (bv.\ LexisNexis, een databank van kranten en nieuwsmagazines).

\paragraph{Open Access bronnen}

De hierboven genoemde databanken zijn in principe niet gratis (hoewel je als student wel gratis toegang krijgt als de HOGENT bib een abonnement heeft). Als je op zoek bent naar open access publicaties, dan kan je beginnen met de volgende startpunten:

\begin{itemize}
  \item arXiv\footnote{\url{https://arxiv.org}}: een online archief waar veel auteurs in de exacte wetenschappen (o.a.\ computerwetenschappen) hun artikels publiceren voordat ze ze indienen bij een journal, dus vóór het peer-reviewproces.
  \item CiteSeerX\footnote{\url{https://citeseerx.ist.psu.edu}}: een publieke zoekmachine en digitaal archief voor academische artikels, vooral in computerwetenschappen, opgezet door Pennsylvania State University.
  \item Internet Archive Scholar\footnote{\url{https://scholar.archive.org}}: een zoekmachine naar academische literatuur onderhouden door het Internet Archive.
\end{itemize}

Open acces betekent dus dat iedereen vrij toegang heeft tot de publicatie, zonder abonnement. Recentelijk zijn de grote uitgeverijen als Elsevier en Springer onder vuur te komen liggen omwille van hun verdienmodel. Enerzijds betalen universiteiten en hogescholen hoge bedragen voor hun abonnementen, maar anderzijds moeten ook onderzoekers die een artikel willen publiceren daarvoor betalen. Uitgeverijen organiseren weliswaar het administratieve aspect van het peer-reviewproces, maar uiteindelijk besteden ze dat uit aan academici, vakexperten die hier niet voor vergoed worden. Op die manier slaagden uitgeverijen er in om hun winstmarges enorm te laten toenemen. 

Vele onderzoekers worden gesubsidieerd door overheden en dus onrechtstreeks ook door de belastingbetalers. Zij vinden dat het resultaat van hun onderzoek ook toegankelijk moet zijn voor diegenen die het mogelijk maakten en dat de grote uitgevers onredelijk grote winsten maken zonder zelf een significante meerwaarde te creëren.

Enkele activisten gingen zelfs een stuk verder dan openlijk protesteren tegen deze gang van zaken. In 2010 schreef Aaron Schwarz een script waarmee systematisch artikels gedownload werden van JSTOR, een online archief met betalende toegang, via een gastaccount dat hij van het MIT ter beschikking had gekregen. Zijn arrestatie en de daarop volgende dreiging van een monsterboete en een 35-jarige gevangenisstraf dreven hem in 2013 tot zelfmoord. In 2011 startte de Kazachstaanse Alexandra Elbakyan de site Sci-Hub, een digitaal archief met zoekmachine voor artikels die in principe achter een ``paywall'' zitten. Op dit moment zouden er een 90-tal miljoen bestanden in het archief zitten. In de academische wereld kan het initiatief op vrij grote steun rekenen, maar uiteraard leidde dit ook tot rechtzaken, o.a.\ in de VS en India. In sommige landen, waaronder België, moeten ISP's de DNS-naam van de site blokkeren en bezoekers van de site omleiden naar een webpagina met een melding van de politie.

In elk geval, open access is een groeiende trend en meer en meer academische instituten kiezen er expliciet voor om enkel te publiceren in journals met open access tot de gepubliceerde artikels.

\subsection{Startpunten voor technische vakliteratuur}%
\label{ssec:technische-vakliteratuur}

Niet voor elk mogelijk onderzoeksdomein van een bacheloproef bestaan er ook effectief \textit{wetenschappelijke} publicaties. Maar wat je zeker wél zou moeten vinden is \textit{vakliteratuur}, geschreven voor en door ict-specialisten. Je zou op zijn minst één of enkele boeken over het onderwerp van je bachelorproef moeten zoeken en lezen!

Hier vind je enkele nuttige startpunten:

\paragraph{Springer Link}

Springer Link hebben we al eerder vernoemd in het kader van wetenschappelijke literatuur. Maar via deze weg kan je ook de catalogus van uitgeverij Apress raadplegen, waar je vele ict-gerelateerde boeken kan terugvinden. Vanop de campus of via VPN kan je deze als PDF of EPUB gratis downloaden.

\paragraph{EBSCOhost}

Packt Publishing is een gelijkaardige uitgeverij met een grote hoeveelheid boeken over ict. Via EBSCOhost\footnote{\url{https://search.ebscohost.com/login.aspx?authtype=ip,uid&custid=s4061053&groupid=main&profile=ehost&defaultdb=nlebk} (zal enkel werken op de campus of via VPN)} (waarop de HOGENT bib een abonnement heeft) kan je deze eveneens downloaden als PDF of EPUB.

\paragraph{Vakconferenties}

Steeds vaker worden lezingen op vakconferenties opgenomen en publiek toegankelijk gemaakt via Youtube, Vimeo, Slideshare, enz. Ga dus op zoek naar de namen van conferenties binnen jouw specifieke vakgebied (bv.\ Configuration Management Camp, FOSDEM, Google IO, Monitorama, OSCON, Velocity, WWDC, \ldots) en kijk eens of je presentaties terugvindt die relevant zijn voor jouw onderzoek.

\paragraph{Vaktechnische websites}

Enkele websites streven er naar om online communities te vormen onder ict professionals en het publiceren van technische artikels over actuele onderwerpen. Deze zijn typisch geschreven door vakspecialisten en worden onderworpen aan een peer-re\-view\-pro\-ces voor publicatie. Op die manier nemen ze de plaats in van ``klassieke'' gedrukte vaktijdschriften.

DZone\footnote{\url{https://dzone.com/}} en InfoQ\footnote{\url{https://www.infoq.com}} zijn voorbeelden van dit soort websites. Je vindt er aparte rubrieken voor de grote deeldomeinen van ict, bv.\ software-architectuur en -ont\-wik\-ke\-ling, databases, netwerken, DevOps, cybersecurity, \ldots

\paragraph{Online leeromgevingen en cursussen}

Als het onderwerp van je bachelorproef gaat over een domein waar je nog niet zo veel van weet, dan is het misschien nuttig om er een cursus over te volgen. Tegenwoordig kan je over de meest uiteenlopende onderwerpen (in het bijzonder over ict) een online cursuss of MOOC (massive open online course) volgen. Vaak zijn die betalend, maar sommige zijn gratis of kan je als student van HOGENT gratis volgen. Enkele ideeën:

\begin{itemize}
  \item Cisco Networking Academy\footnote{\url{https://www.netacad.com}}: via de opleiding heb je misschien al een account. Je kan langs deze weg ook inschrijven voor andere cursussen dan deze die binnen de opleiding gebruikt worden.
  \item Coursera\footnote{\url{https://www.coursera.org}}, een MOOC provider die samenwerkt met zowel universiteiten als grote it-bedrijven met zowel gratis als betalende cursussen.
  \item The Cyber Mentor\footnote{\url{https://www.thecybermentor.com}}: cybersecurity, ethical hacking
  \item edX\footnote{\url{https://www.edx.org}}, een MOOC provider gecreëerd door MIT en Harvard met zowel gratis als betalende cursussen.
  \item LinkedIn Learning\footnote{\url{https://www.linkedin.com/learning/}}: gratis toegang via Academic Software.
  \item Microsoft Learn\footnote{\url{https://learn.microsoft.com/}} (het vroegere Technet)
\end{itemize}

\section{Samenvatting}%
\label{sec:literatuuronderzoek_samenvatting}

\begin{itemize}
  \item Het doel van de literatuurstudie is om jezelf en de lezer van je bachelorproef voldoende context te geven om het onderwerp ten gronde te begrijpen.
  \item Er zijn drie soorten bronnen: primaire (resultaten van eigen onderzoek), secundaire (publicaties in wetenschappelijke of vakliteratuur) en tertiaire (encyclopedieën en zoekindexen). In een bibliografie horen enkel \emph{secundaire} bronnen thuis.
  \item Het is belangrijk om bronnen kritisch te beoordelen. Gebruik de CRAAP-test!
  \item Gebruik niet enkel Google als startpunt voor je zoektocht, maar gebruik zoekmachines specifiek voor wetenschappelijke en vakliteratuur. Zorg voor variatie in je bronnen! Dus niet enkel blogs of webpagina's, maar zeker ook boeken, technische of wetenschappelijke artikels, enz.
\end{itemize}
