\chapter*{Preface}
\label{ch:preface}

This guide orginated from the guidance process of the bachelor theses in the applied computer sciences department of HOGent and was originally published in Dutch via  Github\footnote{\url{https://github.com/HoGentTIN/bachproef-gids/}}. The idea was to give the students useful advice regarding methodology and working with {\LaTeX} for a professional layout, but also, to get them started on time.

In the new curriculum, that was introduced from the academic year 2020--2021, a new course i.e. ``Research methods'' was introduced mainly to support the bachelor's thesis writing process. This bachelor's thesis guide is then further developed and used as the course text for the new subject.

The objective of the bachelor's thesis within the study program is to demonstrate that you are able to serve a ``customer'' (within their own organization or externally) \textbf{with advice about the application of current ICT technology in a business context. } We are not aiming for an academic scientific study, but we target applied research with sufficient technical depth offering added value in the chosen field.

This means that you get to work with a concrete and current problem from the field. You may not be an expert on this topic (yet), forcing you to gather relevant, objective, and authoritative information on that topic. On the one hand, this involves looking up existing knowledge that can be found in the professional literature, via experts or stakeholders. On the other hand, we expect that new knowledge will also be created during the research, for example through self-designed experiments, a proof-of-concept or prototype, interviews, etc. All this information should then be structured and analysed. In the case of quantitative data this should happen in a statistically sound way. Based on this, you can work out a solution and formulate your advice for the customer or client. You compile all this information in a thesis that you submit and defend before a jury.

For most students, the bachelor's thesis seems to pose a greater challenge than their internship. During an internship, you will be working on a clearly defined task with predefined goals, under daily supervision of a mentor. Most choices regarding technologies, tools, environment $\ldots$ have already been made.

For your bachelor's thesis, on the other hand, you are the one to define the scope, set the goal and decide on the required tools to reach said goal. On top of that, you need to be able \textbf{to motivate any choices you have made, write your research down in a professional report and be able to defend your work in front of a jury}. All of this adds a certain level of challenge compared the the internship, but it also enables you to grow as an individual and prove your expertise in a domain of your choosing.

Hopefully this course offers sufficient guidance to successfully start working on your bachelor's thesis!

\bigskip
Revision: \today