\chapter{Formulating a research question}
\label{ch:researchquestion}

This chapter provides some suggestions on searching for a topic. The website of the HOGENT library also has some general advice about this\footnote{\url{https://bib.hogent.be/how-to/onderwerp-formuleren/inleiding/}}. This guide is specifically aimed at the bachelor applied Informatics though.

We regularly receive offers from external parties on subjects that are suitable for a bachelor's thesis, but the offer is not large enough to provide all students with a subject. On the other hand, developing a subject yourself is an interesting opportunity to delve into a subject that you would like to continue with after graduation.

\section{Types of research}
\label{sec:researchtypes}

A bachelor's thesis does not have to try to imitate an academic master's thesis. The professional bachelor has a different value, so it is perfectly possible to excel within the specific uniqueness of this profile. The main difference lies mainly in the type of research that is conducted.

In \emph{fundamental research}, which is typically carried out at universities, the emphasis is on expanding our knowledge in general. For example, within computer science it can be about developing new and/or more efficient algorithms for problem solving. Whether the results of fundamental research are immediately applicable is of secondary importance.

\emph{Applied research}, on the other hand, tries to formulate an answer to a concrete question from the field. The researcher will then try to answer that question on the basis of the knowledge that is currently available (and published by experts on the subject). This may concern, how a new technology can be applied concretely in a business context, making a choice between various alternative products or technologies, a preliminary research prior to developing an application, etc. In applied research, the target group is also very specific, for example a single company. We also notice that the topics introduced by companies are also the best developed and most often lead to a good bachelor's thesis.

\section{Research domain choice}
\label{sec:research}

The first step is to choose your research domain. This is something that no one can really help you with. Choose a domain that you are interested in, so that you can gain sufficient motivation to delve into it. Your chosen specialization in the final year may be a good starting point. What kind of job would you like to start after graduation? Which technologies, platforms, \ldots would you like to work with?

Search for current problems/challenges in your chosen field. It is important to take your time to ``immerse'' yourself in current events. This cannot be done overnight. It is more efficient to put some time into this regularly over a number of weeks (e.g.\ an hour every day). In the long run you should be able to recognize the most important themes that people are mainly working on at the moment and that could provide inspiration for your subject.

In Section~\ref{sec:searchingforrelevantinformation} you will find some concrete tips and starting points on searching for relevant information. 

\section{Formulating research questions and targets}
\label{sec:formulatingresearchquestion}

Once you get a feel for recent changes on a topic, you typically also get to know the most important problems and points of discussion. These can lead to the formulation of your main research question, which you can further divide into more concrete research sub-questions.

A good research question for a bachelor's thesis starts from a real or at least realistic \textit{company case}, a concrete problem that a specific company struggles with. This immediately ensures a clear \textit{delimitation} of the subject. A research question that is too vague or general will never lead to a good bachelor's thesis.

For example, ``What is the best agile methodology for companies?'' is \textit{not} a good research question in that regard. There are several agile methodologies, and there's probably a good reason for that. There is no clear winner that is best suited for every business. If that were the case, then twenty years after the publication of the Agile Manifesto \autocite{BeckEtAl2001}, it would already be clear what the best methodology is. A suitable methodology depends on many factors: the company culture, the type of company, the type of IT solutions being developed, the size of the company, etc. If you formulate your research question too general, you will never be able to come to a conclusive conclusion. We have experienced this too often in the past. The student then writes in the conclusion ``The best solution depends on your preference and your specific situation.'' In other words, there is no conclusion. So start from a concrete case, e.g. company XYZ struggles with the delivery of software on time and within budget and wants to apply an agile methodology. What is the most appropriate methodology for their specific situation? This clearly defines the subject and its added value is immediately visible.

There is no conclusive answer to a good research question yet. Questions such as ``What is data mining?'', ``Which PHP frameworks are there?'', ``What is the history of the Cloud?'' are therefore not suitable, because the answer can be found quickly by search on Wikipedia or Google, or by checking relevant professional literature.

It is also important to immediately think about the expected result, the \textit{research objective}. Under what circumstances can you speak of a successful bachelor's thesis? What is the concrete end result that you want to achieve?

You can use the S.M.A.R.T. principle of \autocite{UchelenJungjohann2003} to formulate your research objective(s):

\begin{description}
    \item[Specific] Is the objective unambiguous? Is it clear who will benefit?
    \item[Measurable] Under which (measurable/observable) conditions or form has the goal been achieved? Is the goal to deliver a proof-of-concept or prototype? An analysis for an application to be developed? A comparison between possible alternative solutions with clear advice? etc.
    \item[Acceptable] Are these goals acceptable to the target audience? Will the target group actually be able to use the proposed solution?
    \item[Realistic] Is the goal achievable? Is the subject sufficiently defined? Is the difficulty of the problem at the professional bachelor level?
    \item[Time-bound] When (in time) should the goal be reached? This is clearly laid down for a bachelor's thesis: the deadline for submission. 
\end{description}

\section{Writing a research proposal}
\label{sec:researchproposal}

Once you have a research question, you can write out your topic to submit it for approval. This means that you will also be conducting some literature research. For instructions on how to approach this, see Chapter~\ref{ch:literaturestudy}.

What you read about the subject, you have to structure and formulate in your own words in a continuous text. A good tool to organize your thoughts on a topic and to write a structured text around it later is to set up a mind map. There are various tools that you can use for free, such as XMind\footnote{\url{https://www.xmind.net/}} or FreeMind\footnote{\url{http://freemind.sourceforge.net/ }}.

% TODO: voorbeeld mindmap?

It is not the intention at this stage to write a fully detailed literature study, but a number of references are expected. After reading your proposal, your supervisor should understand the context of your research and why there is a problem that needs a solution. You must be able to demonstrate this on the basis of authoritative professional literature.

Also think about a title for your bachelor's thesis. It doesn't have to be final yet, but a good title makes it clear exactly which direction you want to go. A concrete title gives your supervisor(s) the confidence that you know exactly what you want to do and that this is a realistic goal. Try to keep the following in mind when formulating a title:

\begin{itemize}
    \item Do not formulate the title as a question.
    \item Avoid obscure jargon and do not use abbreviations. The title should also be understandable to someone outside your specific field.
    \item Just naming your professional domain is not enough because it is too vague. Your title should be specific and make it clear what exactly you want to research. That also means that a title can be long. For example, ``Cloud computing'' is a general term that covers a broad range of topics, and thus means nothing. ``The selection of an open source Infrastructure as a Service platform for setting up a test environment for web development'' is a lot more concrete.  
\end{itemize}

When assessing a subject, the following criteria are taken into account:

\begin{itemize}
 \item There is a concrete, \textbf{clearly defined research question}, research objective.
 \item The proposal is \textbf{innovative} and has a clear \textbf{added value} for a specific target group from the ICT field.
 \item The \textbf{methodology} is clearly justified, research techniques are suitable for answering research question
 \item There is a clear \textbf{personal contribution} and \textbf{technical complexity}.
\end{itemize}

\section{Common mistakes}
\label{sec:subjectcommonmistakes}

The following things provide us with the idea that your subject has not yet been sufficiently worked out in depth or that there still are important mistakes that stand in the way of a successful bachelor's thesis:

\begin{itemize}
  \item There is no concrete real or realistic \textbf{company case} associated with the topic. Research at bachelor level is applied, in other words we try to solve concrete, real problems. That should also be reflected in your subject.
  \item (One of) the research objective(s) is \textbf{speculation about the future} (e.g.  ``what will be the impact of the Internet of Things on everyday life?''). The conclusion of a bachelor's thesis must be verifiable, future predictions never are.
  \item The result of the research is \textbf{depending on external factors} to a large extent. For example, if you are going to conduct a survey, it is important to realize that it is not easy to find enough respondents. You will therefore immediately have to indicate how you intend to take a sufficiently large sample. Even if your plan is to conduct interviews with experts, companies, etc., it is important to make the necessary contacts before the start of your research. After all, if it is not possible to speak to the necessary people in time, the result of your bachelor's thesis will be in danger.
  \item The subject shows insufficient \textbf{technical complexity}. The professional bachelor's degree in applied computer science is a technical profile and we expect you to demonstrate that you are technically strong. For example, a topic such as ``what is the most user-friendly mobile application for home banking for seniors?'' \textit{not} meets this requirement. These kinds of applications are developed by banks and as a student you will never have access to the internal workings. You may have to limit yourself to interviews and/or surveys with the target group. In this way you cannot demonstrate what you have to offer as a computer scientist.
  \item The \textbf{reference list} is too short or consists of unsuitable sources. Take the time to do an initial literature review and look beyond the first blog entry. In chapter~\ref{ch:literaturestudy} you will find instructions on how to approach this.
  \item The subject is \textbf{outside the scope of a bachelor's program or of the student}. Choose a subject that corresponds to a bachelor's degree in terms of level. A typical example of a subject that is too difficult is \textit{Quantum Computing}, at least for a bachelor's degree. Such a subject may be eye-catching, but you are taking the risk of shooting yourself in the foot in terms of difficulty.     
\end{itemize}

\section{Summary}
\label{sec:subjectsummary}

\begin{itemize}
 \item Take your time to find a topic that really interests you.
 \item When writing up a proposal, it is best to be as \emph{concrete} as possible.
 \item Avoid common mistakes such as the lack of a concrete case, speculation, dependence on external factors or a literature study that is too superficial.
\end{itemize}
