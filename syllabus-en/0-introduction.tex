\chapter*{Study Guide}
\label{ch:studyguide}
\addcontentsline{toc}{chapter}{Study Guide}

\section{Target and positioning of the course in the curriculum}
\label{sec:targetandplace}

The target of this course is to prepare the student for writing a bachelor thesis. More specifically, we want to support students with:

\begin{itemize}
  \item Searching for and formulating a research question;
  \item Searching for information in professional literature (literature study);
  \item The setup of a bibliographic database and the correct referencing of sources; 
  \item Applying appropriate research methods;
  \item Making use of the {\LaTeX} typesetting system for a professional layout;
  \item Clear and correct communication about research
\end{itemize}

There is no previous knowledge required to follow the course. A credit for the subject is a formal prerequisite for writing the bachelor thesis. 

\section{Study targets and competences}
\label{sec:studytargets}

Please check the study details on the Chamilo platform.

\begin{itemize}
  \item Formulating a research question
        \begin{itemize}           
          \item The student is able to explain the characteristics of a good research question
          \item The student is able to formulate a research question and defend why it is a good research question
          \item The student can split a research question into specific sub-questions and related objectives
          \item The student is able to explain and distinguish the different types of research
          \item The student is able to explain the different research methods and distinguish their suitability for a specific problem
        \end{itemize}

  \item Literature study
        \begin{itemize}               
          \item The student is able to explain the criteria of qualitative professional literature
          \item The student can distinguish between different types of sources
          \item The student can adjust his/her search strategy in function of the search results
          \item The student can consult professional literature in relation to a theme
          \item The student can extract the main concepts from professional literature
          \item The student can explain characteristics of a structured text
          \item The student can write a structured text
          \item The student can explain the importance of correct referencing
          \item The student can reference literature correctly
        \end{itemize}

  \item Research methods
        \begin{itemize}                  
          \item The student can select a research method in function of the problem at hand
          \item The student is able to carry out a research method in function of the problem 
          \item The student is able to distinguish between methods to collect quantitative data and is able to apply them correctly
          \item The student is able to distinguish between qualitative data collection methods 
        \end{itemize}

  \item Reporting
        \begin{itemize}        
           \item The student is able to format a document in a correctly structured manner and provide references by means of text typesetting software.
        \end{itemize}
\end{itemize}

\section{Content}
\label{sec:studycontent}
This course is subdivided into two major parts. In the IT part, the focus is on the technical part, e.g. searching for a suitable subject, looking up information in professional literature, etc. The language part offers support in clear communication: analyzing texts, adopting a professional writing style, avoiding typical language errors, etc. .

In this document you will mainly find the content of the IT part.

Module~\ref{ch:preparation} helps to set up a working environment, in particular the text typesetting system {\LaTeX} and a version control system.

Module~\ref{ch:researchquestion} provides tips on topic choice and how to work out a topic.

Module~\ref{ch:literaturestudy} provides details on how to perform a literature study. The subsequent module~\ref{ch:bibliography} teaches how to keep references centralized in a bibliographic database facilitating the insertion of a correct bibliography in a {\LaTeX} document.

Module~\ref{ch:researchmethods} elaborates on a set of research methods typically used in a bachelor thesis, i.e. \ the comparison between different methods, or the setup of experiments. 

Finally, module~\ref{ch:reportingresults} provides information regarding the more technical aspects of writing a thesis, and some {\LaTeX} specific guidelines.

\section{Study material}
\label{sec:studymaterial}

All the learning material for this course is provided to the students using the Chamilo platform. 

\section{Teaching approaches}
\label{sec:teachingapproach}

In this course, 12 contact moments of two hours are budgetted, alternating between IT and language classes. The order may vary depending on the student group. In the table below you will find more details on the content of the course in the different classes. 

An overview is provided in Table~\ref{tab:planning}.

\begin{table}
  \centering\begin{tabular}{cll}
    \toprule
    \textbf{Module} & \textbf{IT part}               & \textbf{Language part}                   \\
    \midrule
    1               & Preparation, {\LaTeX}        & Professional text structure \\
    2               & Formulating a research question & Professional writing     \\
    3               & Literature study               & Common language mistakes                 \\
    4               & Referencing correctly              & Evaluating sources   \\
    5               & Research methods             & Writing a summary          \\
    6               & Reporting                    & Presentation skills       \\
  \end{tabular}
  \caption{\label{tab:planning}Course content for the different classes of this course.}
\end{table}

\section{Work and study guidelines}
\label{sec:workstudyguidelines}

TODO

\section{Study counselling and planning}
\label{sec:studycounsellingplanning}

TODO

\section{Evaluation}
\label{sec:evaluation}
The evaluation for this course is based on a paper, written by each student individually or in groups of 2 students. In this paper you will develop a topic that may be suitable for writing a bachelor's thesis. You formulate an appropriate research question, conduct an initial literature study on the research domain and describe your research approach.

There is however no strict requirement that you should pick this subject for your bachelor's thesis. Hopefully you will gain expert knowledge and maturity in the specialization courses yet to be followed, possibly providing some extra inspiration. However, by performing the exercise within this course, you will have a better idea of what to expect from the bachelor's thesis. The process is supposed to train some necessary skills needed to write your bachelor's thesis. 

The result is assessed by both the IT lecturer and the language lecturer, each focussing on their expertise. E.g. the IT lecturer pays attention to the quality of the referenced sources, correct use of {\LaTeX}, etc., while the language lecturer looks at the structure, writing style and correct language use.

The assessment criteria will be published on Chamilo in the form of a Rubrics evaluation card.