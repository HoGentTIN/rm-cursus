\chapter{Preparation, working with \LaTeX{}}
\label{ch:preparation}

In this chapter we treat the initial work, starting to write a bachelor's thesis. You will find some recommendations on useful tools and the research process.

\section{{\LaTeX} use}
\label{sec:latex-use}

Most students are used to writing formatted text in a classic word processor environment (typically MS Word). For your bachelor's thesis it is advisable to abandon this habit.

Word, especially with the default template, produces a layout that is not suitable for publication. Once the length and complexity of a Word document increases (and it certainly does with a thesis), you will have to deal with inconsistencies in the layout of your text, page numbering and poorly positioned images.

When you copy text from another (preparatory) document or from a website, the original layout is copied. If it's not consistent with your main document, you'll have to adjust everything. This is a very time consuming and error prone job.

A classic word processor based on the WYSIWYG principle\footnote{What You See Is What You Get}, allows you to fine-tune the positioning of the text in the paper, but actually this great freedom will turn out as a disadvantage in this case. A sleek and professional design is a specialty that requires great attention to often finicky details. As a computer scientist, we do not have the necessary knowledge to realize this. Moreover, when you spend a significant amount of time designing your document, you are distracted from the core goal: the content of the text!

Another disadvantage of the classic word processor is the binary file format. This makes it impossible to put a document into a version control system (see Section~\ref{sec:versioncontrolsystem}). Soon different versions of the document start to live side by side: `bach ​​test 3.docx', `bach ​​test 5 March 30.docx', `final draft.docx', `final draft after feedback.docx', `final final draft. docx' \dots. You have versions on your laptop, on dropbox, on your fixed PC and in your mailbox. In the long run, the overview is lost, you forget to copy pieces of text or you make other mistakes.

{\LaTeX} is recommended for formatting a long text with a professional and sleek design. As you know, this is a \emph{text typesetting system} with a markup language (like HTML) that specializes in formatting text on paper. You write source code in {\LaTeX} markup, a `compiler' generates a PDF. {\LaTeX} is text based, so you can put this in a version control system.

Admittedly, {\LaTeX} does have some drawbacks. There's a learning curve that shouldn't be underestimated, and as long as you stick to the habits you've picked up from working with a word processor, {\LaTeX} won't always do what you expect. But most effort is required in the beginning to master {\LaTeX}. When writing a bachelor's thesis with a word processor, you have most work at the end, to get rid of all the imperfections, inconsistencies and errors in the design. At that point, you usually don't have enough time for that anymore, because the deadline is approaching. The result is almost always a document that is insufficiently finished and looks very unprofessional to the reader. This is not appropriate for a paper that serves as the conclusion of a \textit{professional} bachelor's degree.

For the rest of this guide, we'll assume you're using {\LaTeX}. Yet, this is not intended to be a {\LaTeX} manual, there are plenty of other sources available for that~\parencite{Oetiker2015}.

You can find a {\LaTeX} template for preparing the bachelor's thesis in the Github repository \url{https://github.com/HoGentTIN/bachproef-latex-template}. Don't forget to change the language to English. You can download the template via the green button at the top right. Cloning the repository or creating a fork is not recommended. After all, doing this will provide you with the entire history of the template, which is not relevant to your work.


\section{Bibliographic database}
\label{sec:bibliografic-database}

A mandatory part of a bachelor's thesis is a literature study. This is to familiarize yourself with the research domain (see Chapter~\ref{ch:literaturestudy}). It is important to keep track of what you read, so that you can refer to your sources when writing the introduction. Referring to sources and setting up a bibliography must be done in a strictly defined manner. This is something that you don't have to do manually, there are several software packages that largely automate this: bibliographic databases.

A bibliographic database allows you to keep structured metadata about the works read: title, author, year, and such, but also (clickable) URLs, PDFs of articles, notes, etc.

There are several possibilities, but JabRef\footnote{\url{http://www.jabref.org/}} is perhaps the most interesting for our purposes. JabRef is an open source bibliographic database written in Java and ideally suited for working with {\LaTeX}. It uses the same file format as Bib{\LaTeX}, the bibliography system built into {\LaTeX}.

The way in which a bibliography and references in the text should be written is fixed. However, there are different formatting styles. One of these is the so-called \textbf{APA style} which comes from the American Psychological Association (APA). It is the main standard for publications in the field of social sciences. The official guide to the APA style is very extensive and can therefore be inconvenient to use. The {\LaTeX} template for the bachelor's thesis already has the APA style built-in, so you don't have to worry about the correct formatting at all. This happens automatically, provided you keep the bibliographic data (eg title, author, year of publication) correctly in Jabref. More about that in Section~\ref{sec:publicationsjabref}.

\textbf{Please note!!} The use of the APA style is mandatory throughout HOGENT. This applies to the layout of the bibliography as well as references to sources in the text.

\section{Version control system}
\label{sec:versioncontrolsystem}

Hopefully the advantages of a version control system do not need to be explained to a computer scientist? Always use a version control system like Git to keep track of your work. Also create a Github repository. This is a good backup system (provided you sync regularly with Github). It also allows you to easily share your work with your promoter. One of the properties of version control is that it is ideally designed to track changes in \emph{text files}. Binary file formats such as documents from a classic word processor are not suitable for this, which is an extra motivation for using \LaTeX{}.

The following things definitely belong in your repository:

\begin{itemize}
    \item \LaTeX{} source code of the bachelor's thesis.
    \item images to insert.
    \item source code of self-written scripts, benchmarks, experiments, proof-of-concepts, etc. This makes your experiments easier to reproduce and validate by third parties.
    \item raw experiment results (in text format, e.g. CSV), transcripts of interviews, etc.
    \item single notes, ideas, etc. Use Markdown for this and avoid Word-do\-cu\-ments.  
\end{itemize}

In short, \emph{all} artifacts resulting from your research belong in the repository. For work documents where you want formatted text, but where \LaTeX{} is overkill, use Markdown\footnote{\url{https://guides.github.com/features/mastering-markdown/}}.

What does \emph{not} belong in your repository:

\begin{itemize}
    \item Help files created when compiling \LaTeX{}. You can prevent these from being included in a repository by creating a \texttt{.gitignore} file\footnote{For example \url{https://github.com/github/gitignore/blob/master/TeX. gitignore}}. The \LaTeX{} template for the bachelor's thesis is already set up correctly.
    \item Large (binary) files such as ISOs, virtual machines (e.g.\ .ova), etc.
    \item PDFs of the articles/ebooks you have read (this is considered ``redistribution'' and is not allowed under copyright law).
    \item Binary files that change often, eg Word documents.
    \item Files automatically generated from code in Git, e.g. compiled code.    
\end{itemize}

A version control system only becomes really useful if you use it properly. So commit as often as possible, write clear commit messages and sync regularly with Github!

\section{Summary}
\label{sec:preparation-summary}

The key points of this chapter are:

\begin{itemize}
    \item To obtain a clean, professional layout, it is better to write your thesis in {\LaTeX} compared to a classic word processor;
    \item Use a \emph{reference manager} to maintain a bibliographic database (JabRef is recommended).
    \item Use a version control system to store all your work (Git is recommended);
\end{itemize}
