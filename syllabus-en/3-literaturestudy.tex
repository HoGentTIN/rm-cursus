\chapter{Literature study}
\label{ch:literaturestudy}

The first stage in any research is typically to write an overview of the current state of the art in the research domain. It is necessary to delve into \emph{everything} that is related to the subject. This is the literature review. In every report on research, it is essential that you can support any statement you pose. This can be done either on the basis of data that you have collected and analyzed yourself in a methodologically correct manner (but that will be discussed later in this guide), or on the basis of references to \emph{reputable} publications.

This chapter takes a closer look at this topic: what exactly can you achieve with a literature study, how can you start working on it, and how to use the sources that you found in your text?

HOGENT has published a course on information skills via Chamilo\footnote{\url{https://chamilo.hogent.be/index.php?go=CourseViewer\&application=Chamilo\%5CApplication\%5CWeblcms\&course=22068\&tool=LearningPath \&browser=Table\&tool_action=ComplexDisplay\&publication=980981}} that you should definitely read. After all, it is not the intention to repeat the same content in this guide. In the first place, this chapter focuses on literature research on an ICT-related topic, and the correct preparation of a reference list with {\LaTeX} and JabRef (see Section~\ref{ch:bibliography}).

\section{Aim of the literature review}
\label{sec:aimliteraturereview}

The main purpose of a literature review is to become familiar with the research domain and to pass that knowledge on to the readers of your bachelor's thesis. So you are going to collect and read as much information as possible about the subject in order to know everything there is to know about it at the moment. It is then the intention to summarize all the knowledge that you have gained in this way that is relevant to your research in a structured manner and in your own words in a continuous text. This usually forms the first chapter (or the first chapters) of your bachelor thesis.

Based on the literature study, you provide the reader with the necessary background to understand the subject. It is an introduction to the subject, and discusses the current state of the art. You state what the experts in the field have to say about it and what research has already been carried out in the past (with the main conclusions). The literature study should also show that there are still gaps in our knowledge, that there is a problem that requires a solution. And that is of course exactly the subject of your bachelor's thesis.

\section{Types of sources}
\label{sec:typessources}

In the context of research, we can divide information sources into these three categories:

\begin{description}
   \item[Primary] knowledge that you acquire yourself during your research. For example measurements from experiments, results of surveys, transcripts of interviews, etc.
   \item[Secondary] publication of knowledge, research, etc.~by others. For example, articles in professional journals, books, presentations at conferences, etc.
   \item[Tertiary] search indexes and encyclopedias. For example, Google Scholar, Web of Science, Elsevier ScienceDirect, Arxiv.org, Wikipedia, about.com, Webopedia, etc.
\end{description}

Whenever reference is made to literature in a text, it always refers to \emph{secondary} sources (also called \emph{publications}). That also means that primary or tertiary sources \emph{cannot} be referenced. For example, reference should never be made to a Wikipedia article, dictionaries, etc. Tertiary sources are good \textit{starting points} for the search for suitable publications (See Section~\ref{sec:searchingforrelevantinformation}). You also cannot refer to the report of an interview that you have conducted, because that is also not published and therefore not accessible to the reader.

\subsection{Publication formats}
\label{sub:publicationformats}

Knowledge is passed on and published in various forms. In this section, we list the most important and discuss the reliability and objectivity of each form.

\paragraph{Article in a scientific journal}

An article that is published in a scientific journal can be assumed to have been preceded by a rigorous verification process. Articles submitted are passed on by the magazine's editors to other experts in the field who are responsible for independently and anonymously verifying their content. This is called \emph{peer review}. This process typically takes several months and can even take up to a few years. Publications in scientific journals are generally considered to be the most reliable and if there are any closely related to your bachelor thesis topic, it is highly recommended that you read them. The disadvantages are that the level (especially in mathematics) is typically quite high and therefore not always accessible to the average bachelor's student.

\paragraph{Article in a professional magazine}

Professional magazines are aimed at a professional audience, so these are not academic, scientific texts. There is also no peer review process prior to the publication of articles. Typically, an editor decides whether or not an article is published. Professional journals are also gradually becoming available only online and we also include portal sites around a specific theme or field, such as dZone\footnote{\url{https://dzone.com/}}, Informit\footnote{\url{https: //www.informit.com/}}, etc.

It is important to know who the author of the article is. Is he/she a recognized subject expert or a journalist? In the first case, the article is certainly useful as a source, but in the other you have to look at it critically. After all, there is rarely any guarantee that a journalist has sufficient expertise in the subject of his article. Journalists also have different motivations than subject experts and above all want their article to be read by as many people as possible. Sometimes they will make things a bit more sensational than they actually are, or they will miss the point when it comes to technicalities.

\paragraph{Presentation at a conference}

Both the scientific and professional community organize conferences worldwide to discuss, present new results and pass on knowledge. Typically, a call is made several months before the start to propose topics for presentations. At a scientific conference, people are then usually asked to submit a written article that is assessed through a peer-review process, which is usually a lot less demanding than for a journal. For professional conferences and for some scientific conferences, a paragraph of text with a summary of the content (abstract) is sufficient. At industry conferences, a panel composed by the organizer will usually review the submissions and select the presentations.

After a conference, the content of the selected presentations is bundled and published. At a scientific conference, this is an (e-)book consisting of the submitted articles, which is called the \emph{proceedings}. At a professional conference, it usually only concerns the bundled presentation slides, or the speakers themselves put their slides on a website to share presentations, such as SlideShare or Speaker Deck. Nowadays, professional conferences are recording some or all of the presentations and making them available via, for example, ~Youtube or Vimeo, or via an in-house website.

It pays to find out which conferences are taking place related to the field in which your chosen topic fits and if possible view the presentations that have taken place.

In terms of reliability, the level of articles in \emph{conference proceedings} approaches that of scientific articles in journals. This applies less to professional conferences, because they are not preceded by a peer-review process. You can get to know the most important experts in a specific field and learn a lot about recent developments in the field. Please note that often researchers try to publish their research discussed at the conference in a more valued ``A1-journal'' later with a slightly different content.

\paragraph{Thesis}

Theses are also often interesting sources of information. The depth here largely depends on the study program for which the thesis was written: doctoral thesis (PhD thesis), master's thesis or bachelor's thesis. These texts are written under the supervision of a promoter who guarantees the quality of the content. If you find a published thesis, then in principle it has been proofread by an expert. In that respect, PhD theses are roughly at the level of scientific articles. Usually it is also the case that one or more parts of a doctoral thesis have also been published as articles in a scientific journal.

\paragraph{Book or manual}

Also with books it is important to know who the author is and to what extent he/she has the authority to write about a subject. After all, in principle anyone can publish a book, and there is no formal peer review, so no independent validation of the content.

% TODO: iets over handleidingen?

\paragraph{White paper}

A \emph{white paper} is a report on a particular topic that aims to give the reader enough background information on that topic to understand it, make decisions, or solve a problem. In our field, white papers are typically published by companies that sell a product related to the topic covered. For example, a company that sells antivirus software might issue white papers on how to secure computers, how to crack passwords, etc.

It is important to realize that white papers are usually not objective. The authors have something to sell, so it is important for them to present the topic in such a way that purchasing their products or services seems interesting. For example, a security software vendor may have an interest in making the cybercrime problem look worse than it actually is. The reader who becomes concerned about the situation is more likely to buy security software.

Read a white paper with a very critical eye and also try to find objective information from other sources.

\paragraph{Blog article}

A blog is usually (part of) a personal website where the author regularly publishes articles on a certain topic and shares his/her knowledge with others in the same field. There are thousands of blogs on ICT-related topics, and chances are that the most important experts within your chosen topic are writing one.

Again, it is important to find out who the author is and what authority he/she has on the subject. When you find an article on, for example, ``continuous delivery'' by Martin Fowler, a globally recognized expert and speaker on software development, this is a very useful resource. An article on the same subject by another author who, for example, after some searching on LinkedIn turns out to be a marketeer, should not be included in your reading list.

\subsection{Source quality judgement}
\label{sub:sourcequalityjudgement}

You should have noticed from the previous section that the quality of sources is not always easy to evaluate. Much depends on who the author is and what authority he has within the field.

A tool when assessing the quality of a resource is the \emph{CRAP-test} \autocite{Gratz2015}:

\begin{description}
   \item[Currency] or topicality: is the source sufficiently recent for the topic?
      \item[Reliability/Relevance]: is the content well substantiated? Are sources referenced? Is the content relevant to your research?
      \item[Authority]: is the author an authority on the subject? Is it about a person or an organization?
   \item[Point of view] or objectivity: what is the intention of the author? What does he/she want to achieve?
\end{description}

When assessing a source, it is necessary to find out who the author is and when it was written and published. Unfortunately, this is not possible on many websites and this information is not available. These kinds of sources do not belong in a bibliography!

\section{Summary}
\label{sec:literatureresearchsummary}

\begin{itemize}
   \item The aim of the literature study is to provide yourself and the reader of your bachelor's thesis with sufficient context to fully understand the subject.
   \item There are three types of sources: primary (results of own research), secondary (publications in scientific or professional literature) and tertiary (encyclopedias and search indexes).
   \item Only \emph{secondary} sources belong in a bibliography.
\end{itemize}
