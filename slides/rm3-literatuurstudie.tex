\documentclass[aspectratio=169]{beamer}
\usetheme{hogent}
\usecolortheme{hgwhite} % witte achtergrond, zwarte tekst

%% common.tex -- Code die in elk .tex-bestand terug komt

%% Packages

\usepackage[dutch]{babel}
\usepackage{graphicx}
\usepackage{comment,enumerate,hyperref}
\usepackage{amsmath,amsfonts,amssymb}
\usepackage{eurosym}
\usepackage{booktabs}
\usepackage{multicol,multirow}
\usepackage{listings}

\usepackage[outputdir=out]{minted}
%\usepackage{minted}

\usepackage[backend=biber,style=apa]{biblatex}
\DeclareLanguageMapping{dutch}{dutch-apa}

\usepackage{csquotes}

%% Variabelen, elk academiejaar aan te passen
\newcommand{\academicyear}{2023--2024 (revisie: \today)}
\newcommand{\lecturers}{Thomas Aelbrecht \and Thomas Parmentier \and Bert Van Vreckem}
\newcommand{\coursename}{Research Methods (IT)}

%% Macro's en commando's

%% \alertbox: een kader voor tekst die moet opvallen
\newcommand{\alertbox}[2][hgblue]{%
  \setbeamercolor{alertbox}{bg=#1,fg=white}
  \begin{beamercolorbox}[sep=2pt,center]{alertbox}
    \textbf{#2}
  \end{beamercolorbox}
}


%---------- Info over de presentatie ------------------------------------------

\title{Module 3. Een literatuurstudie uitvoeren.}
\subtitle{\coursename}
\author{\lecturers}   % Pas waarden aan in common.tex
\date{\academicyear}

\begin{document}

\begin{frame}
  \maketitle
\end{frame}

\begin{frame}
  \frametitle{Inhoud}

  \tableofcontents
\end{frame}

\section{Informatie opzoeken.}

\begin{frame}
  \frametitle{Soorten bronnen.}

  \begin{description}
    \item[Primaire] \textbf{Ruwe data} (zelf) verzameld tijdens onderzoek

      Datasets, enquêtes, interviews, \ldots

    \item[Secundaire] \textbf{Publicatie} van kennis, onderzoek, \ldots door anderen

      Artikel in wetenschappelijke journal of vaktijdschrift, presentatie op conferentie, boek, \ldots

    \item[Tertiaire] \textbf{Indexen}

      Zoekmachine, encyclopedie, databank bibliotheek, \ldots

  \end{description}

  \alertbox{Enkel \textcolor{hgyellow}{secundaire bronnen} zijn bruikbaar als referenties.}
\end{frame}

\begin{frame}
  \frametitle{Informatie opzoeken.}

  Start bij \alert{tertiare} bronnen:

  \begin{itemize}
    \item Google Scholar: \url{https://scholar.google.com/}
    \item Elicit: \url{https://elicit.org/}
    \item ScienceDirect: \url{https://www.sciencedirect.com/}
    \item Springer Online Journals: \url{https://link.springer.com/}
    \item Catalogus Bib: \url{https://www.hogent.be/student/bibliotheken/}
    \item Wikipedia
  \end{itemize}
\end{frame}

\begin{frame}
  \frametitle{Informatie opzoeken.}

  \alertbox{Let op: tertiaire bronnen, ihb.~Wikipedia, zijn zelf \textcolor{hgyellow}{niet} aanvaardbaar als referentie}

  {\pause}

  \begin{itemize}
    \item Geen garantie op juistheid
    \item Beweringen niet altijd aangetoond: [citation needed]
    \item \alert{Wél} een goed startpunt (bv.\ referenties onderaan artikel)
  \end{itemize}
\end{frame}


\begin{frame}
  \frametitle{Checklist kwaliteit bronnen.}
  \framesubtitle{Doe de CRAAP test!}

  \begin{description}
    \item[Current] Publicatiejaar? Is het voldoende recent? Is dit nog conform de \textbf{state-of-the-art}?
    \item[Reliable] Is het objectief? Gebalanceerd of eenzijdig?
      Bronvermeldingen?
    \item[Authoritative] Auteur? Is dit een erkend expert? Wordt er elders naar verwezen?
    \item[Purpose/Point of View] Opinie of feiten? Wil de auteur iets verkopen? Is het relevant voor je onderzoeksvraag?
  \end{description}

\end{frame}

\begin{frame}
  \frametitle{HOGENT tools.}

  \begin{itemize}
    \item<+-> Bezoek website HOGENT \textbf{bib}: \url{https://www.hogent.be/student/bibliotheken/}
      \begin{itemize}
        \item Handleidingen
        \item Kritisch denken en onderzoekscompetenties
        \item Zoeken op Bachelorproeven
      \end{itemize}
    \item<+-> Chamilo, tegel \textbf{Databanken} (via Academic Software)
      \begin{itemize}
        \item start zoekmachines vanop HOGENT
        \item Online journals en ebooks waar HOGENT een abonnement voor heeft (bv.~ScienceDirect, SpringerLink)
      \end{itemize}
    \item<+-> Zet \textbf{VPN} aan tijdens het opzoeken.
  \end{itemize}
\end{frame}

\begin{frame}
  \frametitle{Startpunten.}
  \framesubtitle{``Wetenschappelijke'' literatuur.}

  \begin{itemize}
    \item<+-> \textbf{Google Scholar}
      \begin{itemize}
        \item Gebruik vanop de campus, VPN of via Academic Software
        \item Kijk uit naar download-links aan de rechterkant: [PDF] of [fulltext@Hogent]
        \item Referentie in Bib{\TeX}-formaat verkrijgen (via instellingen)
        \item Gebruik zoekopties (bv.~beperken in tijd)
      \end{itemize}
    \item<+-> \textbf{Elicit}: AI assistent voor het vinden van wetenschappelijke literatuur
    \item<+-> \textbf{SpringerLink}
      \begin{itemize}
        \item e-boeken over computerwetenschappen
        \item journals, artikels
      \end{itemize}
    \item<+-> \textbf{Elsevier ScienceDirect}: journals
  \end{itemize}
\end{frame}

\begin{frame}
  \frametitle{Startpunten.}
  \framesubtitle{Vakliteratuur ict.}

  \begin{itemize}
    \item SpringerLink boeken (ook Apress)
    \item Presentaties \textbf{vakconferenties} (via Youtube, Vimeo, Slideshare, \dots)
          \begin{itemize}
            \item vb. Google IO, WWDC, FOSDEM, Velocity, \dots
            \item Zoeken via Lanyrd (\url{http://lanyrd.com/topics/})
          \end{itemize}
    \item<+-> Technische \textbf{portaalsites} voor ict-gerelateerde onderwerpen
      \begin{itemize}
        \item vb.~dzone.com, infoq.com, TechNet, enz.
      \end{itemize}
  \end{itemize}
\end{frame}

\begin{frame}
  \frametitle{Startpunten.}
  \framesubtitle{Vakliteratuur ict.}

  \begin{itemize}
    \item<+-> Wie zijn de belangrijkste namen in de \textbf{community}?
      \begin{itemize}
        \item Keynotes op conferenties, auteurs van standaardwerken, enz.
        \item Volg ze op Twitter
        \item Zoek hun blog
      \end{itemize}
    \item<+-> \textbf{Technische blogs} van bedrijven
      \begin{itemize}
        \item Google Developers Blog, Twitter Engineering/Developer Blog, Netflix Tech Blog, \dots
      \end{itemize}
  \end{itemize}
\end{frame}


\begin{frame}
  \frametitle{Bruikbare bronnen.}
  \framesubtitle{Voor een bachelorproef Informatica}

  \begin{description}
    \item[Journal article]<+-> in wetenschappelijk, peer reviewed tijdschrift
    \item[Conference proceedings]<+-> artikel gepresenteerd op wetenschappelijk, peer reviewed congres
    \item[Thesis]<+-> doctoraat (PhD), Master, Bachelor
    \item[Handleiding]<+-> bv.\ van gebruikte of besproken software
    \item[Boek]<+-> let op: iedereen kan een boek uitgeven. Controleer auteur, uitgeverij, doelpubliek (Springer vs.\ ``for dummies'')
    \item[Presentatie]<+-> door erkend vakexpert, bv.\ op vakconferentie (via Youtube, Vimeo, enz.)
    \item[Blogartikel]<+-> indien geschreven door erkend vakexpert
    \item[Vaktijdschrift]<+-> let op: geschreven door journalist\linebreak(is geen vakexpert)
  \end{description}
\end{frame}

\begin{frame}
  \frametitle{Onbruikbare bronnen.}

  \begin{itemize}
    \item Eender welk werk zonder auteur of publicatiejaar
    \item Wikipedia-artikel
    \item Blogartikel van iemand buiten het vakgebied
    \item Homepage van een besproken product of bedrijf
          \begin{itemize}
            \item Evt.~in de tekst zelf of als voetnoot
          \end{itemize}
    \item \dots
  \end{itemize}
\end{frame}

\section{Wat is een literatuurstudie?}

\begin{frame}
  \frametitle{Literatuurstudie.}

  \begin{itemize}
    \item Onderdeel van elk artikel, eindwerk
    \item Inleiding op het onderwerp
    \item Samenvatting van wat auteur gelezen heeft
    \item Verwijzingen naar vakliteratuur
  \end{itemize}

\end{frame}

\begin{frame}
  \frametitle{Doel van de literatuurstudie.}

  \begin{itemize}
    \item Wat is de huidige stand van zaken?
    \item Wat zeggen experts er over?
    \item Onderzoeksvragen verduidelijken, in context plaatsen
    \item Er is een probleem dat een oplossing vraagt
  \end{itemize}

  \bigskip

  \alertbox{\textcolor{hgyellow}{Elke bewering} in een literatuurstudie moet je bewijzen a.h.v.~referenties}
\end{frame}

\begin{frame}
  \frametitle{Vaak voorkomende fouten.}

  \begin{itemize}
    \item Onvolledige/geen referentielijst
    \item Enkel URLs
    \item Te weinig informatie in referentielijst
          \begin{itemize}
            \item \(\Rightarrow\) bronnen niet terug te vinden
          \end{itemize}
    \item Onaanvaardbare bronnen
    \item Verkeerde opmaak, bv.:
          \begin{itemize}
            \item URL's niet opgeruimd
            \item Google Books URL vermelden als ``bron''
          \end{itemize}
    \item Geen verwijzingen naar bronnen vanuit de tekst
    \item Opgedeeld per type (boek, web, enz.)
  \end{itemize}
\end{frame}


\begin{frame}[plain]
  \frametitle{Doel van de referentielijst.}

  Lezers toelaten:

  \begin{itemize}
    \item De gerefereerde bronnen op te zoeken
    \item Waarde bronnen zelf te beoordelen
  \end{itemize}

  {\pause}

  Strikte, vastgelegde vorm:

  \begin{itemize}
    \item Vastgelegde regels, afh.~publicatie (bv. IEEE, APA, Chicago Manual of Style, \ldots)
    \item Vaste volgorde (volgorde in de tekst of alfabetisch)
    \item Lijst met URL's is onvoldoende!
  \end{itemize}

  {\pause}

  \alertbox{Gebruik \textcolor{hgyellow}{referentie-software} om je referentielijst op te maken!}
\end{frame}

\begin{frame}
  \frametitle{Wanneer refereren naar de literatuur?}

  \begin{itemize}
    \item Definities, eerste vermelding vakterm
    \item Overnemen uit bron van letterlijk citaat, vertaling/parafrase, of afbeelding
          \begin{itemize}
            \item Geen referentie = \alert{plagiaat!}
          \end{itemize}
    \item Aanhalen resultaten vorig onderzoek
    \item Vrijwel elke bewering die je doet over het vakgebied
  \end{itemize}

  \bigskip

  \alertbox{Referenties geven \textcolor{hgyellow}{geloofwaardigheid} aan je literatuurstudie}
\end{frame}

\begin{frame}
  \frametitle{Conclusie}

  \begin{itemize}
    \item Een literatuurstudie vraagt veel werk!
    \item Kritisch lezen
    \item Nauwgezet bronnen bijhouden (zie volgende les)
    \item Losse bronnen structureren tot doorlopend verhaal
    \item \ldots{} maar is essentieel onderdeel van een scriptie!
  \end{itemize}

\end{frame}

\begin{frame}
  \frametitle{Opdracht}

  \begin{itemize}
    \item Gebruik de tips van deze les om info op te zoeken over het onderwerp dat je gekozen hebt
    \item Hou URL's bij! Later: opslaan in bibliografische databank
    \item Verscheidenheid aan soorten bronnen
          \begin{itemize}
            \item Boek, artikel in vak-/wet.\ tijdschrift, presentatie op conferentie, blog (door vakexpert!)
            \item Niet beperkt tot Nederlands, ook Engels!
          \end{itemize}
    \item Doe de CRAAP-test!
    \item Tip: verwerk in mindmap
  \end{itemize}

\end{frame}

\end{document}
