\documentclass[aspectratio=169]{beamer}
\usetheme{hogent}
\usecolortheme{hgwhite} % witte achtergrond, zwarte tekst

%% common.tex -- Code die in elk .tex-bestand terug komt

%% Packages

\usepackage[dutch]{babel}
\usepackage{graphicx}
\usepackage{comment,enumerate,hyperref}
\usepackage{amsmath,amsfonts,amssymb}
\usepackage{eurosym}
\usepackage{booktabs}
\usepackage{multicol,multirow}
\usepackage{listings}

\usepackage[outputdir=out]{minted}
%\usepackage{minted}

\usepackage[backend=biber,style=apa]{biblatex}
\DeclareLanguageMapping{dutch}{dutch-apa}

\usepackage{csquotes}

%% Variabelen, elk academiejaar aan te passen
\newcommand{\academicyear}{2023--2024 (revisie: \today)}
\newcommand{\lecturers}{Thomas Aelbrecht \and Thomas Parmentier \and Bert Van Vreckem}
\newcommand{\coursename}{Research Methods (IT)}

%% Macro's en commando's

%% \alertbox: een kader voor tekst die moet opvallen
\newcommand{\alertbox}[2][hgblue]{%
  \setbeamercolor{alertbox}{bg=#1,fg=white}
  \begin{beamercolorbox}[sep=2pt,center]{alertbox}
    \textbf{#2}
  \end{beamercolorbox}
}

\addbibresource{rm1-werken-met-latex.bib}

%---------- Info over de presentatie ------------------------------------------

\title{Module 4. Een bibliografische databank bijhouden.}
\subtitle{Research Methods}
\author{\lecturers}   % Pas waarden aan in common.tex
\date{\academicyear}

\begin{document}

\begin{frame}
  \maketitle
\end{frame}

\begin{frame}
  \frametitle{Inhoud}

  \tableofcontents
\end{frame}

\section{Literatuurstudie in {\LaTeX}.}

\begin{frame}[fragile]
  \frametitle{Bronvermelding en referentielijst in {\LaTeX}.}

  Bib{\LaTeX} en Biber

  \vspace{18pt}

  \verb|artikel.tex|: Hoofdtekst\\
  \verb|artikel.bib|: Bibliografische databank (bewerk met bv.~JabRef)

  \vspace{18pt}

  Preamble:

  \begin{verbatim}
  \usepackage[backend=biber,style=apa]{biblatex}
  \DeclareLanguageMapping{dutch}{dutch-apa}
  \addbibresource{artikel.bib}
  \end{verbatim}

\end{frame}

\subsection{Een bibliografische databank aanleggen.}

\begin{frame}
  \frametitle{Bibliografische gegevens in Jabref.}

  Info die \textbf{altijd} ingevuld moet worden:

  \begin{description}
    \item[Author] Familienaam, Voornaam and Familienaam, Voornaam and Familienaam, Voornaam\ldots
    \item[Title] v/h artikel, boek, \ldots
    \item[Year] of datum van publicatie
    \item[Bibtexkey] id van deze bron, gebruikt bij refereren (tip: klik op sleutel-icoon)
  \end{description}
\end{frame}

\begin{frame}[fragile]
  \frametitle{Bibliografische gegevens in Jabref.}
  \framesubtitle{Extra info voor Article}

  \begin{description}
    \item[Journal] Naam van het tijdschrift
    \item[Volume] Jaargang
    \item[Number] Nummer binnen de jaargang (optioneel)
    \item[Pages] \verb|mmm--nnn|
  \end{description}

  \bigskip

  \textbf{Voorbeeld:}

  \bigskip

  \fullcitebib{Anscombe1973}
\end{frame}

\begin{frame}[plain]
  \frametitle{Bibliografische gegevens in Jabref.}
  \framesubtitle{Extra info voor Electronic}

  \begin{description}
    \item[Url] Hyperlink naar de bron
    \item[Urldate] Datum van raadplegen
  \end{description}

  \bigskip

  \textbf{Voorbeeld:}

  \bigskip

  \fullcitebib{Lundin2020}

\end{frame}

\begin{frame}[plain]
  \frametitle{Bibliografische gegevens in Jabref.}
  \framesubtitle{Extra info voor InProceedings}

  \begin{description}
    \item[Booktitle] ``Proceedings of the [naam conferentie]''
    \item[Editor] Redacteur(s) (optioneel)
    \item[Pages] paginanummers (optioneel)
  \end{description}

  \medskip

  \textbf{Voorbeeld:}

  \fullcitebib{vanderLaanEtAl2015}
\end{frame}

\begin{frame}
  \frametitle{Bibliografische gegevens in Jabref.}

  Vul zoveel mogelijk info in (maakt zoeken gemakkelijker):

  \begin{description}
    \item[DOI] Digital Object Identifier: uniek ID voor artikel, hiermee kan je automatisch alle velden invullen
    \item[URL] ook al is het niet echt een Electronic bron
    \item[Keywords] Sleutelwoorden
    \item[File] PDF van de publicatie
    \item[Abstract] Samenvatting
    \item[Comments] Je eigen samenvatting/opmerkingen
  \end{description}

\end{frame}

\subsection{Refereren naar de literatuur.}

\begin{frame}[fragile]
  \frametitle{Bronvermelding en referentielijst in {\LaTeX}.}

  \begin{itemize}
    \item Verwijzingen in de tekst:

    \begin{itemize}
      \item \verb|\textcite{Knuth1998}| \(\Rightarrow\) Knuth (1998)
      \item \verb|\autocite{Knuth1998}| \(\Rightarrow\) (Knuth, 1998)
    \end{itemize}

    \item Literatuurlijst invoegen: \verb|\printbibliography|

    \item Compileren (in TexStudio):

    \begin{enumerate}
      \item Build/Compile (F5): bronnen worden nog niet toegevoegd, ``keys'' van bronnen in het vet aangeduid
      \item Bibliography (F8): selecteert de gerefereerde bronnen en maakt ze klaar
      \item Build/Compile (F5): effectief invoegen verwijzingen en literatuurlijst
    \end{enumerate}
  \end{itemize}

  Zie de voorziene sjablonen of de cursus voor voorbeelden!
\end{frame}

\begin{frame}
  \frametitle{En nu?}

  \begin{itemize}
    \item Verwerk wat je gelezen hebt tot een doorlopende tekst
    \item Stijl/structuur nabootsen van gelezen artikels!
    \item Structureren, bv. ahv Mind Map
      \begin{itemize}
        \item Xmind, Minder, FreeMind, Vym, \ldots
      \end{itemize}
  \end{itemize}
\end{frame}

\begin{frame}
  \frametitle{Meer info}

  De inhoud van deze les is verwerkt in een ``Praktische gids voor de bachelorproef''

  \vspace{12pt}

  \url{https://github.com/HoGentTIN/bachproef-gids/releases}

\end{frame}

%% TODO: aanvullen met info uit Bachelorproefgids

\end{document}
