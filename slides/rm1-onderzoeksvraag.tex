\documentclass[aspectratio=169]{beamer}
\usetheme{hogent}
\usecolortheme{hgwhite} % witte achtergrond, zwarte tekst

%% common.tex -- Code die in elk .tex-bestand terug komt

%% Packages

\usepackage[dutch]{babel}
\usepackage{graphicx}
\usepackage{comment,enumerate,hyperref}
\usepackage{amsmath,amsfonts,amssymb}
\usepackage{eurosym}
\usepackage{booktabs}
\usepackage{multicol,multirow}
\usepackage{listings}

\usepackage[outputdir=out]{minted}

\usepackage[backend=biber,style=apa]{biblatex}
\DeclareLanguageMapping{dutch}{dutch-apa}

\usepackage{csquotes}

%% Variabelen, elk academiejaar aan te passen
\newcommand{\academicyear}{2021--2022}
\newcommand{\lecturers}{Thomas Aelbrecht \and Lena De Mol \and Koen Mertens \and Jillisa Schittecatte \and Bert Van Vreckem}

%% Macro's en commando's

%% \alertbox: een kader voor tekst die moet opvallen
\newcommand{\alertbox}[2][hgblue]{%
  \setbeamercolor{alertbox}{bg=#1,fg=white}
  \begin{beamercolorbox}[sep=2pt,center]{alertbox}
    \textbf{#2}
  \end{beamercolorbox}
}


%---------- Info over de presentatie ------------------------------------------

\title{Module 1. Een onderzoeksvraag formuleren.}
\subtitle{Research Methods}
\author{\lecturers}   % Pas waarden aan in common.tex
\date{\academicyear}

\begin{document}

\begin{frame}
  \maketitle
\end{frame}

\begin{frame}
  \frametitle{Inhoud}

  \tableofcontents
\end{frame}

\section{Soorten onderzoek}

\begin{frame}
  \frametitle{Soorten onderzoek}
  \framesubtitle{Fundamenteel, academisch onderzoek}

  \begin{itemize}
    \item Typisch voor universiteit (PhD, postdoc)
    \item Uitbreiden kennis
    \item Niet noodzakelijk praktische toepassing!
    \item Voorbeelden:
          \begin{itemize}
            \item Nieuw algoritme ontwikkelen
            \item Programmeertaal ontwerpen/verbeteren
            \item Theoretische mogelijkheden van computers verkennen
          \end{itemize}
  \end{itemize}

\end{frame}

\begin{frame}
  \frametitle{Soorten onderzoek}
  \framesubtitle{Toegepast onderzoek}

  \begin{itemize}
    \item Typisch voor hogeschool
    \item Start van concrete vraag uit werkveld
    \item Op basis van state-of-the-art beste oplossing zoeken
    \item \textit{Valorisatie} is belangrijk
    \item Voorbeelden:
          \begin{itemize}
            \item Marktonderzoek
            \item Ontwerp architectuur v/e ict-oplossing
            \item Keuze maken tussen alternatieve producten/technologieën
            \item Nieuwe technologieën in de praktijk toepassen
          \end{itemize}
  \end{itemize}

\end{frame}

\begin{frame}
  \frametitle{Fundamenteel vs.\ toegepast}

  \begin{itemize}
    \item Toegepast onderzoek is NIET ondergeschikt/minderwaardig aan academisch onderzoek!
    \item Resultaten fundamenteel onderzoek niet altijd praktisch toepasbaar
          \begin{itemize}
            \item \textit{toy problems}
            \item model niet bruikbaar reële situatie
          \end{itemize}
    \item Professionele Bachelor: uniek gepositioneerd om als vakspecialist vertaalslag te maken tussen theorie en praktijk
  \end{itemize}

\end{frame}

\section{De bachelorproef informatica}

\begin{frame}{Doelstelling}

  \alertbox{Kunnen advies geven over toepassen nieuwe ict-technologieën in bedrijfscontext}

  \begin{itemize}
    \item informatie verzamelen en kritisch beschouwen
          \begin{itemize}
            \item literatuur
            \item belanghebbenden
            \item experimenten
          \end{itemize}
    \item structureren en analyseren
    \item proof-of-concept opzetten
    \item rapporteren
  \end{itemize}

\end{frame}

\begin{frame}
  \frametitle{Bachelorproef vs.\ stage}

  \centering
  \begin{tabular}{lcl}
    \toprule
    \textbf{Stage}               & $\leftrightarrow$ & \textbf{Bachelorproef}        \\
    \midrule
    Uitvoerend                   & $\leftrightarrow$ & Onderzoekend                  \\
    Onder supervisie mentor      & $\leftrightarrow$ & Zelfstandig                   \\
    Keuzes zijn gemaakt          & $\leftrightarrow$ & Zelf onderbouwde keuzes maken \\
    Resultaat: afgewerkt product & $\leftrightarrow$ & Resultaat: PoC, rapport       \\
    \bottomrule
  \end{tabular}
\end{frame}

\begin{frame}{Praktisch}

  \begin{itemize}
    \item 3e jaar modeltraject
    \item 1e semester: onderwerp kiezen en uitwerken
    \item 2e semester: realisatie
    \item 4 dagen stage; 1 dag BP
    \item indienen bij aanvang examenperiode
    \item presentatie + verdediging in juni
  \end{itemize}

  \bigskip

  Uitzonderingen: BP in 2e zit of 1e semester
\end{frame}

\begin{frame}{Opvolging}

  \begin{itemize}
    \item \textbf{Promotor:} \textbf{proces}opvolging
          \begin{itemize}
            \item Lector van de opleiding
            \item Voorzitter jury presentatie
            \item Kent examencijfer toe
          \end{itemize}
    \item \textbf{Co-promotor:} \textbf{inhoudelijke} opvolging
          \begin{itemize}
            \item Opdrachtgever
            \item Vakexpert
            \item Van buiten HoGent, (evt.\ een lector)
            \item Geen familielid (tot 3e graad)
          \end{itemize}
  \end{itemize}

  \bigskip

  Je dient zelf een co-promotor te zoeken!

\end{frame}

\section{Een onderwerp kiezen}

\begin{frame}{Op zoek naar een onderwerp}

  \begin{itemize}
    \item Via stagebedrijf/-mentor
    \item (Beperkt) aanbod onderwerpen via Chamilo
    \item Voorstel vanuit je eigen contacten in het werkveld
    \item Eigen idee/voorstel
  \end{itemize}

  \bigskip

  \alertbox{Wacht niet te lang met het zoeken naar een onderwerp!}
\end{frame}

\begin{frame}{Geschikte onderwerpen}

  \begin{itemize}
    \item Bachelorproef = \textbf{toegepast} onderzoek
    \item Doelpubliek = IT-professionals!
    \item Concreet probleem in bedrijfscontext
    \item Wat is de beste oplossing binnen huidige state-of-the-art?
  \end{itemize}

  \bigskip

  \alertbox{Start vanuit een concrete, \textcolor{hgyellow}{reële bedrijfscasus}}
\end{frame}

\begin{frame}{Ongeschikte onderwerpen}

  \begin{itemize}
    \item Onvoldoende technische diepgang (bv.\ enkel enquête)
    \item De toekomst van \ldots{} (niet speculeren!)
    \item Algemene vergelijking van frameworks/producten/\ldots
          \begin{itemize}
            \item Afhankelijk van specifieke requirements
            \item Koppel dit aan een reële bedrijfscasus
          \end{itemize}
    \item Te breed, vaag of vrijblijvend
    \item Te moeilijk voor een bacheloropleiding
    \item Enkel literatuurstudie
    \item Geen eigen bijdrage
  \end{itemize}

\end{frame}

\begin{frame}{Zelf een onderwerp zoeken}
  \framesubtitle{Kies een onderzoeksdomein}

  \begin{itemize}
    \item Keuzevak 3TI
    \item In welke job wil je starten?
    \item Met welke technologieën/platformen/\ldots wil je werken?
  \end{itemize}

\end{frame}

\begin{frame}{Zelf een onderwerp zoeken}
  \framesubtitle{Volg de actualiteit}

  \begin{itemize}
    \item Portaalsites (dzone, infoq, technet, \ldots)
    \item Relevante conferenties/meetups
    \item Lokale vakverenigingen (bv. OWASP Belgium)
    \item Belangrijkste namen ``community'' volgen

          (bv.\ via blog, Twitter)

    \item Newsletters
    \item \ldots
  \end{itemize}

\end{frame}

\begin{frame}{Een onderzoeksvraag formuleren}

  \begin{itemize}
    \item Wat zijn de actuele thema's in de ``community''?
          \begin{itemize}
            \item Onderwerpen op conferenties
            \item Discussies op fora, Twitter, \ldots
          \end{itemize}
    \item Op een onderzoeksvraag is nu nog geen antwoord!
    \item Zoek een co-promotor, vraag raad
  \end{itemize}

\end{frame}

\begin{frame}[plain]
  \frametitle{Vaak gestelde vraag}

  Is \ldots{} een goed onderwerp voor de bachelorproef?

  \begin{itemize}
    \item De cloud
    \item Blockchain
    \item AI
    \item Security
    \item \ldots
  \end{itemize}

  \bigskip

  \alertbox{Ja, vooropgesteld dat het onderwerp \textcolor{hgyellow}{ICT-gerelateerd is}, \textcolor{hgyellow}{technische diepgang heeft} en je een \textcolor{hgyellow}{concrete onderzoeksvraag} hebt geformuleerd vanuit een \textcolor{hgyellow}{reële bedrijfscasus}}
\end{frame}

\section{Het onderwerp uitschrijven}

\begin{frame}{Het onderwerp uitschrijven}

  \begin{itemize}
    \item Aan de hand van sjabloon
    \item Doel: zekerheid scheppen dat je voorstel \textbf{S.M.A.R.T.} is
          \begin{itemize}
            \item Specifiek, concreet
            \item Meetbare doelstellingen
            \item Acceptabel voor doelgroep
            \item Realistisch en haalbaar
            \item Tijdgebonden
          \end{itemize}
  \end{itemize}
\end{frame}

\begin{frame}
  \frametitle{Criteria beoordeling voorstel}

  \begin{itemize}
    \item Titel: duidelijk \& concreet
    \item Onderzoeksvraag: concreet, afgebakend
    \item Meerwaarde voor specifieke doelgroep
    \item Methodologie: voldoende concreet
    \item Toon je expertise aan!
    \item Eigen bijdrage!
  \end{itemize}

\end{frame}

\begin{frame}
  \frametitle{Oefening}
  \framesubtitle{Wat vind jij van deze onderwerpen?}

  \centering

  \url{https://forms.office.com/r/MG0aJfYv68}

  \bigskip

  \includegraphics[height=.5\textheight]{2/form-eval-onderwerp.png}

\end{frame}

\begin{frame}
  \frametitle{Opdracht}
  \framesubtitle{Zoek een onderwerp}

  \begin{itemize}
    \item Zie Github repo, instructies fase 2
    \item Kies een onderwerp, schrijf het (voorlopig) uit in 1 paragraaf
    \item Inspiratie nodig? Chamilo > Documenten > Voorbeelden bachelorproeftitels (xlsx)
  \end{itemize}

\end{frame}

\end{document}
