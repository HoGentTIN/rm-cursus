\documentclass[aspectratio=169]{beamer}
\usetheme{hogent}
\usecolortheme{hgwhite} % witte achtergrond, zwarte tekst

%% common.tex -- Code die in elk .tex-bestand terug komt

%% Packages

\usepackage[dutch]{babel}
\usepackage{graphicx}
\usepackage{comment,enumerate,hyperref}
\usepackage{amsmath,amsfonts,amssymb}
\usepackage{eurosym}
\usepackage{booktabs}
\usepackage{multicol,multirow}
\usepackage{listings}

\usepackage[outputdir=out]{minted}
%\usepackage{minted}

\usepackage[backend=biber,style=apa]{biblatex}
\DeclareLanguageMapping{dutch}{dutch-apa}

\usepackage{csquotes}

%% Variabelen, elk academiejaar aan te passen
\newcommand{\academicyear}{2023--2024 (revisie: \today)}
\newcommand{\lecturers}{Thomas Aelbrecht \and Thomas Parmentier \and Bert Van Vreckem}
\newcommand{\coursename}{Research Methods (IT)}

%% Macro's en commando's

%% \alertbox: een kader voor tekst die moet opvallen
\newcommand{\alertbox}[2][hgblue]{%
  \setbeamercolor{alertbox}{bg=#1,fg=white}
  \begin{beamercolorbox}[sep=2pt,center]{alertbox}
    \textbf{#2}
  \end{beamercolorbox}
}


%---------- Info over de presentatie ------------------------------------------

\title{Module 3. Het literatuuroverzicht.}
\subtitle{\coursename}
\author{\lecturers}   % Pas waarden aan in common.tex
\date{\academicyear}

\begin{document}

\begin{frame}
  \maketitle
\end{frame}

\begin{frame}
  \frametitle{Inhoud}

  \tableofcontents
\end{frame}


\section{Wat is het literatuuroverzicht?}

\begin{frame}
  \frametitle{literatuuroverzicht.}

  \begin{itemize}
    \item Onderdeel van elk artikel, eindwerk
    \item Inleiding op het onderwerp
    \item Samenvatting van wat auteur gelezen heeft
    \item Verwijzingen naar vakliteratuur
  \end{itemize}

  \bigskip

  \alertbox{Met het literatuuroverzicht heeft de lezer alle informatie om de inhoud van je tekst te begrijpen, zonder nog andere bronnen te moeten naslaan.}
\end{frame}

\begin{frame}
  \frametitle{Doel van het literatuuroverzicht.}

  \begin{itemize}
    \item Wat is de huidige stand van zaken?
    \item Wat zeggen experts er over?
    \item Is het probleem al eerder onderzocht?
    \item Onderzoeksvragen verduidelijken, in context plaatsen
    \item Er is een probleem dat een oplossing vraagt
  \end{itemize}

  \bigskip

  \alertbox{\textcolor{hgyellow}{Elke bewering} in een literatuuroverzicht moet je bewijzen a.h.v.~referenties}
\end{frame}

\begin{frame}
  \frametitle{Vaak voorkomende fouten.}

  \begin{itemize}
    \item Onvolledige/geen referentielijst
    \item Enkel URL's
    \item Te weinig informatie in referentielijst
          \begin{itemize}
            \item[$\Rightarrow$] bronnen niet terug te vinden
          \end{itemize}
    \item Onaanvaardbare bronnen
    \item Verkeerde opmaak, bv.:
          \begin{itemize}
            \item URL's niet opgekuist
            \item Google Books URL of zelfs illegale downloadsite vermelden als ``bron''
          \end{itemize}
    \item Geen verwijzingen naar bronnen vanuit de tekst
    \item Opgedeeld per type (boek, web \ldots)
  \end{itemize}
\end{frame}

\begin{frame}
  \frametitle{Wanneer refereren naar de literatuur?}

  \begin{itemize}
    \item Definities, eerste vermelding vakterm
    \item Overnemen uit bron van letterlijk citaat, vertaling/parafrase, of afbeelding
          \begin{itemize}
            \item Geen referentie = \alert{plagiaat!}
          \end{itemize}
    \item Aanhalen resultaten vorig onderzoek
    \item Vrijwel elke bewering die je doet over het vakgebied
  \end{itemize}

  \bigskip

  \alertbox{Referenties geven \textcolor{hgyellow}{geloofwaardigheid} aan je literatuuroverzicht}
\end{frame}

\section{Informatie opzoeken.}

\begin{frame}
  \frametitle{Checklist kwaliteit bronnen.}
  \framesubtitle{Doe de CRAAP test!}

  \begin{description}
    \item[Currency] Publicatiejaar? Is het voldoende recent? Is dit nog conform de \emph{state-of-the-art}?
    \item[Relevance] Relevant voor jouw onderzoek? Vakliteratuur of populariserend?
    \item[Authority] Auteur? Is dit een erkend expert? Wordt er elders naar verwezen?
    \item[Accuracy] Is de informatie correct? Verwijst de tekst naar bronnen? Kan de informatie worden geverifieerd? Is de informatie gereviewd? Is het taalgebruik professioneel? Is het objectief?
    \item[Purpose] Opinie of feiten? Neutraal of mening? Gebalanceerd of eenzijdig? Wil de auteur iets verkopen?
  \end{description}

\end{frame}

\begin{frame}
  \frametitle{Zoekhulpmiddelen en indexen.}

  \begin{itemize}
    \item Google Scholar: \url{https://scholar.google.com/}
    \item Elicit: \url{https://elicit.org/}
    \item ScienceDirect: \url{https://www.sciencedirect.com/}
    \item Springer Online Journals: \url{https://link.springer.com/}
    \item Catalogus Bib: \url{https://www.hogent.be/student/bibliotheken/}
    \item Wikipedia: \url{https://nl.wikipedia.org/}
  \end{itemize}
\end{frame}

\begin{frame}
  \frametitle{HOGENT tools.}

  \begin{itemize}
    \item<+-> Bezoek website HOGENT \textbf{bib}: \url{https://www.hogent.be/student/bibliotheken/}
    \begin{itemize}
        \item Handleidingen
        \item Kritisch denken en onderzoekscompetenties
        \item Zoeken op Bachelorproeven
    \end{itemize}
    \item<+-> Zet je HOGENT \textbf{VPN} aan tijdens het opzoeken, zo krijg je automatisch toegang tot journals en ebooks waarvoor HOGENT een abonnement heeft (bv.~ScienceDirect, SpringerLink). 
    \begin{itemize}
        \item Instructies voor configuratie van de VPN vind je op: \url{https://www.hogent.be/helpdesk/} (klik door naar VPN).
    \end{itemize}
    \item<+-> Kun je de VPN client niet installeren? Log in op Chamilo en klik op de tegel \textbf{Databanken} (via Academic Software)
    \begin{itemize}
        \item Start zoekmachines vanop HOGENT in je browser
    \end{itemize}
    
  \end{itemize}
\end{frame}

\begin{frame}
  \frametitle{Zoekhulpmiddelen en indexen.}
  \framesubtitle{``Wetenschappelijke'' literatuur.}

  \begin{itemize}
    \item<+-> \textbf{Google Scholar}
      \begin{itemize}
        \item Gebruik vanop de campus, VPN of via Academic Software
        \item Kijk uit naar download-links aan de rechterkant: [PDF] of [fulltext@Hogent]
        \item Referentie in Bib{\TeX}-formaat verkrijgen (via instellingen)
        \item Gebruik zoekopties (bv.~beperken in tijd)
      \end{itemize}
    \item<+-> \textbf{Elicit}: AI assistent voor het vinden van wetenschappelijke literatuur
    \item<+-> \textbf{SpringerLink}
      \begin{itemize}
        \item e-boeken over IT
        \item journals, artikels
      \end{itemize}
    \item<+-> \textbf{Elsevier ScienceDirect}: journals
  \end{itemize}
\end{frame}

\begin{frame}
  \frametitle{Zoekhulpmiddelen en indexen.}
  \framesubtitle{Vakliteratuur ict.}

  \begin{itemize}
    \item SpringerLink boeken (ook Apress)
    \item EBSCOhost (Packt Publishing)
    \item Presentaties \textbf{vakconferenties} (via Youtube, Vimeo, Slideshare, \dots)
          \begin{itemize}
            \item vb. Google IO, WWDC, FOSDEM, Velocity, \dots
          \end{itemize}
    \item<+-> Vaktechnische \textbf{portaalsites} voor ict-gerelateerde onderwerpen
      \begin{itemize}
        \item vb.~dzone.com, infoq.com, TechNet, enz.
      \end{itemize}
  \end{itemize}
\end{frame}

\begin{frame}
  \frametitle{Zoekhulpmiddelen en indexen.}
  \framesubtitle{Vakliteratuur ict.}

  \begin{itemize}
    \item<+-> Wie zijn de belangrijkste namen in de \textbf{community}?
      \begin{itemize}
        \item Keynotes op conferenties, auteurs van standaardwerken, enz.
        \item Volg ze op Twitter
        \item Zoek hun blog
      \end{itemize}
    \item<+-> \textbf{Technische blogs} van bedrijven
      \begin{itemize}
        \item Google Developers Blog, Twitter Engineering/Developer Blog, Netflix Tech Blog, \dots
      \end{itemize}
  \end{itemize}
\end{frame}

\begin{frame}
  \frametitle{Publicatievormen.}

  \begin{description}
    \item[Journal article]<+-> in wetenschappelijk, peer reviewed tijdschrift
    \item[Vaktijdschrift]<+-> let op: soms geschreven door journalist\linebreak(is geen vakexpert)
    \item[Kranten- of tijdschriftartikel] <+-> let op: meestal geschreven door journalist. Evt.~om te verwijzen naar gebeurtenis in de actualiteit.
      \item[Conference proceedings]<+-> artikel gepresenteerd op wetenschappelijk, peer reviewed congres
      \item[Presentatie]<+-> door erkend vakexpert, bv.\ op vakconferentie (via Youtube, Vimeo, enz.)
  \end{description}
\end{frame}

\begin{frame}
  \frametitle{Publicatievormen (vervolg).}

  \begin{description}
    \item[Thesis]<+-> doctoraat (PhD), Master, Bachelor
    \item[Boek]<+-> let op: iedereen kan een boek uitgeven. Controleer auteur, uitgeverij, doelpubliek (Springer vs.\ ``for dummies'')
    \item[Handleiding]<+-> bv.\ van gebruikte of besproken software
    \item[White paper]<+-> door bedrijf, meestal met commercieel doel
    \item[Blogartikel]<+-> indien geschreven door erkend vakexpert
  \end{description}
\end{frame}


\begin{frame}
  \frametitle{Onbruikbare bronnen.}

  \alertbox{Let op: Wikipedia is \textcolor{hgyellow}{niet} aanvaardbaar als bron!}

  {\pause}

  \begin{itemize}
    \item Geen garantie op juistheid
    \item Beweringen niet altijd aangetoond: [citation needed]
    \item \alert{Wél} een goed startpunt (bv.\ referenties onderaan artikel)
  \end{itemize}
\end{frame}


\begin{frame}
  \frametitle{Onbruikbare bronnen.}

  \begin{itemize}
    \item Eender welk werk zonder auteur of publicatiejaar
    \item Andere *pedia websites (investopedia, \dots)
    \item Blogartikel van niet-expert
    \item Homepage van een besproken product of bedrijf
          \begin{itemize}
            \item Evt.~in de tekst zelf of als voetnoot
          \end{itemize}
    \item Interne communicatie
    \item (Niet gepubliceerd) cursusmateriaal
  \end{itemize}

  \alertbox{Je kan enkel verwijzen naar \textcolor{hgyellow}{publicaties}!}
\end{frame}

\begin{frame}
  \frametitle{Te vermijden:}

  \begin{itemize}
    \item Googlen naar ``Wat is <vakterm>'' en zoekresultaten als bron gebruiken
    \item Populariserende bronnen (zoek naar vakliteratuur!)
    \item Krantenartikel over wetenschappelijk onderzoek (zoek originele artikel!)
    \item Eenzijdige selectie (bv.~enkel websites/blogs)
    \item Illegale downloadsites (bv.~``gratis'' PDF's van boeken)
  \end{itemize}
\end{frame}

\begin{frame}
  \frametitle{Conclusie}

  \begin{itemize}
    \item Een literatuuroverzicht schrijven vraagt veel werk!
    \item Kritisch lezen
    \item Nauwgezet bronnen bijhouden (zie volgende les)
    \item Losse bronnen structureren tot doorlopend verhaal
    \item \ldots{} maar is essentieel onderdeel van een scriptie!
  \end{itemize}

\end{frame}

\begin{frame}
  \frametitle{Opdracht}

  \begin{itemize}
    \item Gebruik de tips van deze les om info op te zoeken over het onderwerp dat je gekozen hebt
    \item Hou URL's/PDF's bij! Later: opslaan in bibliografische databank
    \item Verscheidenheid aan soorten bronnen
          \begin{itemize}
            \item Boek, artikel in vak-/wet.\ tijdschrift, presentatie op conferentie, blog (door vakexpert!)
            \item Niet beperkt tot Nederlands, ook Engels!
          \end{itemize}
    \item Doe de CRAAP-test!
    \item Tip: verwerk in mindmap
  \end{itemize}

\end{frame}

\end{document}
