\documentclass[aspectratio=169]{beamer}
\usetheme{hogent}
\usecolortheme{hgwhite} % witte achtergrond, zwarte tekst

%% common.tex -- Code die in elk .tex-bestand terug komt

%% Packages

\usepackage[dutch]{babel}
\usepackage{graphicx}
\usepackage{comment,enumerate,hyperref}
\usepackage{amsmath,amsfonts,amssymb}
\usepackage{eurosym}
\usepackage{booktabs}
\usepackage{multicol,multirow}
\usepackage{listings}

\usepackage[outputdir=out]{minted}

\usepackage[backend=biber,style=apa]{biblatex}
\DeclareLanguageMapping{dutch}{dutch-apa}

\usepackage{csquotes}

%% Variabelen, elk academiejaar aan te passen
\newcommand{\academicyear}{2021--2022}
\newcommand{\lecturers}{Thomas Aelbrecht \and Lena De Mol \and Koen Mertens \and Jillisa Schittecatte \and Bert Van Vreckem}

%% Macro's en commando's

%% \alertbox: een kader voor tekst die moet opvallen
\newcommand{\alertbox}[2][hgblue]{%
  \setbeamercolor{alertbox}{bg=#1,fg=white}
  \begin{beamercolorbox}[sep=2pt,center]{alertbox}
    \textbf{#2}
  \end{beamercolorbox}
}


%---------- Info over de presentatie ------------------------------------------

\title{Module 2. Een onderzoeksvraag formuleren.}
\subtitle{Research Methods}
\author{\lecturers}   % Pas waarden aan in common.tex
\date{\academicyear}

\begin{document}

\begin{frame}
  \maketitle
\end{frame}

\begin{frame}
  \frametitle{Inhoud}

  \tableofcontents
\end{frame}


\section{Soorten onderzoek}

% Soorten onderzoek: fundamenteel vs toegepast
% Bachelor vs master
% Typische soorten bachelorproeven
% - vergelijkende studie: wat is de beste oplossing voor een bepaald specifiek probleem in het werkveld?
% - vooronderzoek voor product: marktanalyse, architectuur
% - toepassen van nieuwe technologie in bedrijfscontext (casus)
% Doelpubliek = IT'ers!
% In BP moet je tonen dat je IT'er bent, je moet dus IT'er zijn om ze te kunnen schrijven
% Criteria voor een goed onderwerp
% Opdracht: verzin zelf een onderwerp, schrijf het (voorlopig) uit in 1 paragraaf (later verder uitwerken!)

\section{De bachelorproef informatica}

\begin{frame}{Doelstelling}
  
  \alertbox{Kunnen advies geven over toepassen nieuwe ict-technologieën in bedrijfscontext}
  
  \begin{itemize}
    \item informatie verzamelen en kritisch beschouwen
    \begin{itemize}
      \item literatuur
      \item belanghebbenden
      \item experimenten
    \end{itemize}
    \item structureren en analyseren
    \item proof-of-concept opzetten
    \item rapporteren
  \end{itemize}
  
\end{frame}

\begin{frame}{Bachelorproef vs stage}
  
  \begin{multicols}{2}
    \textbf{Stage}
    
    \begin{itemize}
      \item Uitvoerend
      \item Onder begeleiding mentor
      \item Resultaat kan afgewerkt product zijn
    \end{itemize}
    
    \columnbreak
    
    \textbf{Bachelorproef}
    
    \begin{itemize}
      \item Onderzoekend
      \item Zelfstandig
      \item Resultaat is rapport, proof-of-concept, \textbf{geen} product
    \end{itemize}
  \end{multicols}
  
  Stage en bachelorproef mogen over zelfde onderwerp gaan, maar moeten gescheiden zijn en onafhankelijk van elkaar.
  
\end{frame}

\begin{frame}{Praktisch}
  
  \begin{itemize}
    \item 3e modeltraject
    \item 1e semester: onderwerp kiezen en uitwerken
    \item 2e semester: realisatie
    \item 4 dagen stage; 1 dag BP
    \item indienen bij aanvang examenperiode
    \item presentatie + verdediging in juni
  \end{itemize}
  
  Uitzonderingen: BP in 2e zit of 1e semester
\end{frame}

\begin{frame}{Opvolging}
  
  \begin{itemize}
    \item \textbf{Promotor:} \textbf{proces}opvolging
    \begin{itemize}
      \item Lector van de opleiding
      \item Voorzitter jury presentatie
      \item Kent examencijfer toe
    \end{itemize}
    \item \textbf{Co-promotor:} \textbf{inhoudelijke} opvolging
    \begin{itemize}
      \item Opdrachtgever
      \item Vakexpert
      \item Van buiten HoGent, (evt. een lector)
      \item Geen familielid (tot 3e graad)
    \end{itemize}
  \end{itemize}
  
  Je dient zelf een co-promotor te zoeken!
  
\end{frame}

\section{Een onderwerp kiezen}

\begin{frame}{Op zoek naar een onderwerp}
  
  \begin{itemize}
    \item Via stagebedrijf/-mentor
    \item (Beperkt) aanbod onderwerpen via Chamilo
    \item Voorstel vanuit je eigen contacten in het werkveld
    \item Eigen idee/voorstel
  \end{itemize}
  
  \alertbox{Wacht niet te lang met het zoeken naar een onderwerp!}
\end{frame}

\begin{frame}{Geschikte onderwerpen}
  
  Bachelorproef = \textbf{toegepast} onderzoek
  
  \begin{itemize}
    \item Concreet probleem in bedrijfscontext
    \item Wat is de beste oplossing binnen huidige state-of-the-art?
  \end{itemize}
  
  \alertbox{Start vanuit een concrete, \textcolor{hgyellow}{reële bedrijfscasus}}
\end{frame}

\begin{frame}{Ongeschikte onderwerpen}
  
  \begin{itemize}
    \item Onvoldoende technische diepgang (bv. enkel enquête)
    \item De toekomst van \ldots (niet speculeren!)
    \item Algemene vergelijking van frameworks/producten/\ldots
    \begin{itemize}
      \item Afhankelijk van specifieke requirements
      \item Koppel dit aan een reële bedrijfscasus
    \end{itemize}
    \item Te breed, vaag of vrijblijvend
    \item Enkel literatuurstudie
    \item Geen eigen bijdrage
  \end{itemize}
  
\end{frame}

\begin{frame}{Zelf een onderwerp zoeken}
  \framesubtitle{Kies een onderzoeksdomein}
  
  \begin{itemize}
    \item Keuzevak 3TI
    \item In welke job wil je starten?
    \item Met welke technologieën/platformen/\ldots wil je werken?
  \end{itemize}
  
\end{frame}

\begin{frame}{Zelf een onderwerp zoeken}
  \framesubtitle{Volg de actualiteit}
  
  \begin{itemize}
    \item Portaalsites (dzone, infoq, technet, \ldots)
    \item Relevante conferenties/meetups
    \item Lokale vakverenigingen (bv. OWASP Belgium)
    \item Belangrijkste namen ``community'' volgen
    
    (bv. via blog, Twitter)
    
    \item Newsletters
    \item \ldots
  \end{itemize}
  
  \alertbox{Start hier nu al mee!}
\end{frame}

\begin{frame}{Een onderzoeksvraag formuleren}
  
  \begin{itemize}
    \item Wat zijn de actuele thema's in de ``community''?
    \begin{itemize}
      \item Onderwerpen op conferenties
      \item Discussies op fora, Twitter, \ldots
    \end{itemize}
    \item Op een onderzoeksvraag is nu nog geen antwoord!
    \item Zoek een co-promotor, vraag raad
  \end{itemize}
  
\end{frame}

\begin{frame}{Vaak gestelde vraag}
  
  Is \ldots een goed onderwerp voor de bachelorproef?
  
  \begin{itemize}
    \item De cloud
    \item Blockchain
    \item AI
    \item Security
    \item \ldots
  \end{itemize}
  
  \alertbox{Ja, vooropgesteld dat het onderwerp \textcolor{hgyellow}{ICT-gerelateerd is}, \textcolor{hgyellow}{technische diepgang heeft} en je een \textcolor{hgyellow}{concrete onderzoeksvraag} hebt geformuleerd vanuit een \textcolor{hgyellow}{reële bedrijfscasus}}
\end{frame}

\section{Het onderwerp uitschrijven}

\begin{frame}{Het onderwerp uitschrijven}
  
  \begin{itemize}
    \item Aan de hand van sjabloon
    
    \url{https://github.com/HoGentTIN/bachproef-latex-sjabloon/tree/master/voorstel}
    
    \item Doel: zekerheid scheppen dat je voorstel \textbf{S.M.A.R.T.} is
    \begin{itemize}
      \item Specifiek, concreet
      \item Meetbare doelstellingen
      \item Acceptabel voor doelgroep
      \item Realistisch en haalbaar
      \item Tijdgebonden
    \end{itemize}
  \end{itemize}
\end{frame}

\begin{frame}{Procedure}
  
  \begin{itemize}
    \item Indienen op Chamilo (cursus Bachelorproef) vóór deadline (half december)
    \item Feedback door één van de lectoren, dit wordt je promotor
    \item Herwerk zo nodig voorstel
    \item Aan de slag!
  \end{itemize}
\end{frame}

\end{document}
