\documentclass[aspectratio=169]{beamer}
\usetheme{hogent}
\usecolortheme{hgwhite} % witte achtergrond, zwarte tekst

%% common.tex -- Code die in elk .tex-bestand terug komt

%% Packages

\usepackage[dutch]{babel}
\usepackage{graphicx}
\usepackage{comment,enumerate,hyperref}
\usepackage{amsmath,amsfonts,amssymb}
\usepackage{eurosym}
\usepackage{booktabs}
\usepackage{multicol,multirow}
\usepackage{listings}

\usepackage[outputdir=out]{minted}
%\usepackage{minted}

\usepackage[backend=biber,style=apa]{biblatex}
\DeclareLanguageMapping{dutch}{dutch-apa}

\usepackage{csquotes}

%% Variabelen, elk academiejaar aan te passen
\newcommand{\academicyear}{2023--2024 (revisie: \today)}
\newcommand{\lecturers}{Thomas Aelbrecht \and Thomas Parmentier \and Bert Van Vreckem}
\newcommand{\coursename}{Research Methods (IT)}

%% Macro's en commando's

%% \alertbox: een kader voor tekst die moet opvallen
\newcommand{\alertbox}[2][hgblue]{%
  \setbeamercolor{alertbox}{bg=#1,fg=white}
  \begin{beamercolorbox}[sep=2pt,center]{alertbox}
    \textbf{#2}
  \end{beamercolorbox}
}

\addbibresource{rm4-bibliografie.bib}

%---------- Info over de presentatie ------------------------------------------

\title{Module 4. Keeping a bibliographic database.}
\subtitle{Research Methods}
\author{\lecturers}   % Pas waarden aan in common.tex
\date{\academicyear}

\begin{document}

\begin{frame}
  \maketitle
\end{frame}

\begin{frame}
  \frametitle{Content}

  \tableofcontents
\end{frame}

\section{Literature study in {\LaTeX}.}

\begin{frame}[fragile]
  \frametitle{Source citation and reference list in {\LaTeX}.}

  Bib{\LaTeX} and Biber

  \vspace{18pt}

  \verb|article.tex|: Main text\\
  \verb|article.bib|: Bibliografic database (edit using e.g.~JabRef)

  \vspace{18pt}

  Preamble:

  \begin{verbatim}
  \usepackage[backend=biber,style=apa]{biblatex}
  \DeclareLanguageMapping{dutch}{dutch-apa}
  \addbibresource{article.bib}
  \end{verbatim}

\end{frame}

\subsection{Making a bibliographic database.}

\begin{frame}
  \frametitle{Bibliografic data in Jabref.}

  fields that \textbf{always} need to be filled:

  \begin{description}
    \item[Author] Surname, firstname and surname, firstname and surname, firstname\ldots
    \item[Title] of the article, book, \ldots
    \item[Year] or publication date
    \item[Bibtexkey] id of this source, used when referencing, (tip: click on the key icoon)
  \end{description}
\end{frame}

\begin{frame}[fragile]
  \frametitle{Bibliografic data in Jabref.}
  \framesubtitle{Extra fields for Article}

  \begin{description}
    \item[Journal] Name of the journal
    \item[Volume] Volume
    \item[Number] Number within the volume (optional)
    \item[Pages] \verb|mmm--nnn|
  \end{description}

  \bigskip

  \textbf{Example:}

  \bigskip

  \fullcitebib{Anscombe1973}
\end{frame}

\begin{frame}[plain]
  \frametitle{Bibliografic data in Jabref.}
  \framesubtitle{Extra fields for Electronic}

  \begin{description}
    \item[Url] Hyperlink to source
    \item[Urldate] Visit date
  \end{description}

  \bigskip

  \textbf{Example:}

  \bigskip

  \fullcitebib{Lundin2020}

\end{frame}

\begin{frame}[plain]
  \frametitle{Bibliografic data in Jabref.}
  \framesubtitle{Extra fields for InProceedings}

  \begin{description}
    \item[Booktitle] ``Proceedings of the [name conference]''
    \item[Editor] Editor(s) (optional)
    \item[Pages] Page numbers (optional)
  \end{description}

  \medskip

  \textbf{Voorbeeld:}

  \fullcitebib{vanderLaanEtAl2015}
\end{frame}

\begin{frame}
  \frametitle{Bibliografic data in Jabref.}

  Fill out as much information as possible (facilitates searching):

  \begin{description}
    \item[DOI] Digital Object Identifier: unique ID for article, automatically fill in every field
    \item[URL] even if it is not a real Electronic source
    \item[Keywords] Keywords
    \item[File] PDF of the publication
    \item[Abstract] Summary
    \item[Comments] Your own summary/remarks
  \end{description}

\end{frame}

\subsection{Referencing literature.}

\begin{frame}[fragile]
  \frametitle{Source citation and reference list in {\LaTeX}.}

  \begin{itemize}
    \item References in the text:

    \begin{itemize}
      \item \verb|\textcite{Knuth1998}| \(\Rightarrow\) Knuth (1998)
      \item \verb|\autocite{Knuth1998}| \(\Rightarrow\) (Knuth, 1998)
    \end{itemize}

    \item Insert reference list: \verb|\printbibliography|

    \item Compile (in TexStudio):

    \begin{enumerate}
      \item Build/Compile (F5): sources are not added yet, ``keys'' of sources are marked in bold
      \item Bibliography (F8): selects the referenced sources and prepares them 
      \item Build/Compile (F5): inserts references in the text and generates the reference list
    \end{enumerate}
  \end{itemize}

  Checkout the provided templates or the course syllabus for examples!
\end{frame}

\begin{frame}
  \frametitle{What now?}

  \begin{itemize}
    \item Process what you read to a easily readable text
    \item Mimic the style/structure of the articles that you read!
    \item Structure, e.g. using a mind map
      \begin{itemize}
        \item Xmind, Minder, FreeMind, Vym, \ldots
      \end{itemize}
  \end{itemize}
\end{frame}

\begin{frame}
  \frametitle{More information}

  The content of this class is compiled into a ``Praktische gids voor de bachelorproef''

  \vspace{12pt}

  \url{https://github.com/HoGentTIN/bachproef-gids/releases}

\end{frame}

%% TODO: aanvullen met info uit Bachelorproefgids

\end{document}
