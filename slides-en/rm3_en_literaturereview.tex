\documentclass[aspectratio=169]{beamer}
\usetheme{hogent}
\usecolortheme{hgwhite} % witte achtergrond, zwarte tekst

%% common.tex -- Code die in elk .tex-bestand terug komt

%% Packages

\usepackage[dutch]{babel}
\usepackage{graphicx}
\usepackage{comment,enumerate,hyperref}
\usepackage{amsmath,amsfonts,amssymb}
\usepackage{eurosym}
\usepackage{booktabs}
\usepackage{multicol,multirow}
\usepackage{listings}

\usepackage[outputdir=out]{minted}
%\usepackage{minted}

\usepackage[backend=biber,style=apa]{biblatex}
\DeclareLanguageMapping{dutch}{dutch-apa}

\usepackage{csquotes}

%% Variabelen, elk academiejaar aan te passen
\newcommand{\academicyear}{2023--2024 (revisie: \today)}
\newcommand{\lecturers}{Thomas Aelbrecht \and Thomas Parmentier \and Bert Van Vreckem}
\newcommand{\coursename}{Research Methods (IT)}

%% Macro's en commando's

%% \alertbox: een kader voor tekst die moet opvallen
\newcommand{\alertbox}[2][hgblue]{%
  \setbeamercolor{alertbox}{bg=#1,fg=white}
  \begin{beamercolorbox}[sep=2pt,center]{alertbox}
    \textbf{#2}
  \end{beamercolorbox}
}


%---------- Info over de presentatie ------------------------------------------

\title{Module 3. Performing a literature review.}
\subtitle{\coursename}
\author{\lecturers}   % Pas waarden aan in common.tex
\date{\academicyear}

\begin{document}

\begin{frame}
    \maketitle
\end{frame}

\begin{frame}
    \frametitle{Content}

    \tableofcontents
\end{frame}

\section{Looking up information.}

\begin{frame}
    \frametitle{Source types.}

    \begin{description}
        \item[Primary] \textbf{Raw data} gathered (yourself) during research

            Data sets, surveys, interviews, \ldots

        \item[Secondary] \textbf{Publication} of knowledge, research, \ldots by others

            Article in scientific or professional journal, presentation at conference, book, \ldots

        \item[Tertiary] \textbf{Indexes}

            Search engine, encyclopedia, database library, \ldots

    \end{description}

    \alertbox{Only \textcolor{hgyellow}{secondary sources} are useful as references.}
\end{frame}

\begin{frame}
    \frametitle{Looking up information.}

    Start with \alert{tertiary} sources:

    \begin{itemize}
        \item Google Scholar: \url{https://scholar.google.com/}
        \item Elicit: \url{https://elicit.org/}
        \item ScienceDirect: \url{https://www.sciencedirect.com/}
        \item Springer Online Journals: \url{https://link.springer.com/}
        \item Catalogus Bib: \url{https://www.hogent.be/student/bibliotheken/}
        \item Wikipedia
    \end{itemize}
\end{frame}

\begin{frame}
    \frametitle{Looking up information.}

    \alertbox{Watch out: tertiary sources, in particular Wikipedia, are \textcolor{hgyellow}{not} accepted as reference}

    {\pause}

    \begin{itemize}
        \item No correctness guarantee
        \item Statements are not always proven: [citation needed]
        \item \alert{Though} a good starting point (e.g.\ references below article)
    \end{itemize}
\end{frame}

\begin{frame}
    \frametitle{HOGENT tools.}

    \begin{itemize}
        \item<+-> visit HOGENT \textbf{library} website : \url{https://www.hogent.be/student/bibliotheken/}
            \begin{itemize}
                \item Manuals
                \item Critical mindset and research competences
                \item Search for Bachelor theses
            \end{itemize}
        \item<+-> Chamilo, tile \textbf{Databases} (via Academic Software)
            \begin{itemize}
                \item Start search engines from HoGent
                \item Online journals and ebooks where HoGent has a subscription (e.g.~ScienceDirect, SpringerLink)
            \end{itemize}
        \item<+-> Enable \textbf{VPN} while searching.
    \end{itemize}
\end{frame}

\begin{frame}
    \frametitle{Starting points.}
    \framesubtitle{``Scientific'' literature.}

    \begin{itemize}
        \item<+-> \textbf{Google Scholar}
            \begin{itemize}
                \item Browse on-campus, using VPN or via Academic Software
                \item Look out for download-links at the right hand side: [PDF] or [fulltext@Hogent]
                \item Get reference in Bib{\TeX}-format (via settings)
                \item Use search options (e.g.~limit in time)
            \end{itemize}
        \item<+-> \textbf{Elicit}: AI assistant for searching scientific literature
        \item<+-> \textbf{SpringerLink}
            \begin{itemize}
                \item E-books on computer sciences
                \item Journals, articles
            \end{itemize}
        \item<+-> \textbf{Elsevier ScienceDirect}: journals
    \end{itemize}
\end{frame}

\begin{frame}
    \frametitle{Starting points.}
    \framesubtitle{Professional ICT literature.}

    \begin{itemize}
        \item SpringerLink books (also Apress)
        \item Presentations of \textbf{academic or professional conferences} (via Youtube, Vimeo, Slideshare, \dots)
              \begin{itemize}
                  \item e.g. Google IO, WWDC, FOSDEM, Velocity, \dots
                  \item Search via Lanyrd (\url{http://lanyrd.com/topics/}): no longer active, acquired by Eventbrite
              \end{itemize}
        \item<+-> Technical \textbf{portal sites} for ICT related subjects
            \begin{itemize}
                \item vb.~dzone.com, infoq.com, TechNet, enz.
            \end{itemize}
    \end{itemize}
\end{frame}

\begin{frame}
    \frametitle{Starting points.}
    \framesubtitle{Professional ICT literature.}

    \begin{itemize}
        \item<+-> Who are the most important names in the \textbf{community}?
            \begin{itemize}
                \item Keynotes at conferences, authors of reference works, etc.
                \item Follow them on Twitter
                \item Find their blog
            \end{itemize}
        \item<+-> \textbf{Technical company blogs}
            \begin{itemize}
                \item Google Developer's Blog, Twitter Engineering/Developer Blog, Netflix Tech Blog, \dots
            \end{itemize}
    \end{itemize}
\end{frame}


\begin{frame}
    \frametitle{Useful sources.}
    \framesubtitle{For a bachelor thesis in computer sciences}

    \begin{description}
        \item[Journal article]<+-> in scientific, peer reviewed journal
        \item[Conference proceedings]<+-> article presented at scientific, peer reviewed conference
        \item[Thesis]<+-> doctoral thesis (PhD), Master, Bachelor
        \item[Manual]<+-> e.g.\ of common or famous software
        \item[Book]<+-> watch out: anyone can publish a book. Check the author, publisher, target audience (Springer vs.\ ``for dummies'')
        \item[Presentation]<+-> by known expert, e.g.\ at conference (via Youtube, Vimeo, enz.)
        \item[Blog article]<+-> if authored by known expert
        \item[Professional journal]<+-> watch out: written by journalist \\ (is NOT an expert)
    \end{description}
\end{frame}

\begin{frame}
    \frametitle{Unuseful sources.}

    \begin{itemize}
        \item Any work without author or year of publication
        \item Wikipedia article
        \item Blog article of non-professional
        \item Homepage of product or company
              \begin{itemize}
                  \item sometimes in text or as a footer
              \end{itemize}
        \item \dots
    \end{itemize}
\end{frame}

\begin{frame}
    \frametitle{Checklist source quality.}
    \framesubtitle{CRAAP test!}

    \begin{description}
        \item[Current] Year of publication? Recent enough? Is it still \textbf{state-of-the-art}?
        \item[Reliable] Is it objective? Balanced or biased? Sources?
        \item[Authoritative] Author? Is it a known expert? Any references?
        \item[Accuracy] Is the information correct? Does it contain references to other work? Can the information be verified? Is the information peer reviewed? Is the language professional?
        \item[Purpose/Point of View] Opinion or facts? Does the author wish to sell? Is it relevant enough for your research question?
    \end{description}

\end{frame}

%% TODO: opdracht = Gebruik de tips van deze les om info op te zoeken over het onderwerp dat je gekozen hebt
%% - Hou URLs bij! Later in bibliografische databank bijhouden
%% - verscheidenheid aan soorten bronnen (evt. journal article, boek, presentatie op een vakconferentie, vaktijdschrift, blog (door expert!))
%% - NL, maar ook EN!
%% - Doe de CRAP test!


\section{What is a literature review?}

\begin{frame}
    \frametitle{Literature review.}

    \begin{itemize}
        \item Part of every article, thesis
        \item Introduction to the subject
        \item Summary of what author has read
        \item References to professional literature
    \end{itemize}

\end{frame}

\begin{frame}
    \frametitle{Goal of the literature review.}

    \begin{itemize}
        \item What is the current state of the art?
        \item What do experts say?
        \item Clarify research questions, place in context
        \item There is a problem that demands a solution
    \end{itemize}

    \bigskip

    \alertbox{\textcolor{hgyellow}{Every statement} in a literature review needs to be backed up by references}
\end{frame}

\begin{frame}
    \frametitle{Common mistakes.}

    \begin{itemize}
        \item Incomplete/no bibliography
        \item Only URLs
        \item Missing information in bibliography
              \begin{itemize}
                  \item \(\Rightarrow\) sources untrackable
              \end{itemize}
        \item Unacceptable sources
        \item Wrong format
        \item No references to sources from text
        \item Grouped per type (book, web, etc.)
    \end{itemize}
\end{frame}


\begin{frame}[plain]
    \frametitle{Bibliography goal.}

    Allow readers to:

    \begin{itemize}
        \item Search for the referenced sources
        \item Judge the value of sources
    \end{itemize}

    {\pause}

    Strict, fixed form:

    \begin{itemize}
        \item Fixed set of rules, depending on publicaton (bv. IEEE, APA, Chicago Manual of Style, \ldots)
        \item Fixed order (order in text or alphabetically)
        \item List of URLs is insufficient!
    \end{itemize}

    {\pause}

    \alertbox{Use \textcolor{hgyellow}{reference-software} to generate your bibliography!}
\end{frame}

\begin{frame}
    \frametitle{When to reference?}

    \begin{itemize}
        \item Definitions, first reference to professional term
        \item Copy from source(citation), translate/paraphrase, or picture
              \begin{itemize}
                  \item No reference = \alert{plagiarism!}
              \end{itemize}
        \item Refer to results of previous research
        \item Almost every statement on the subject
    \end{itemize}

    \bigskip

    \alertbox{References generate \textcolor{hgyellow}{credibility} to your literature review}
\end{frame}

\begin{frame}
    \frametitle{Conclusion}

    \begin{itemize}
        \item A literature review demands great efforts!
        \item Critical mindset
        \item Meticulously manage/track your sources (see next session)
        \item Structure loose sources into coherent story
        \item \ldots{} but is an essential part of a thesis!
    \end{itemize}

\end{frame}

\end{document}
