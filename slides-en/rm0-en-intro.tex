\documentclass[aspectratio=169]{beamer}
\usetheme{hogent}
\usecolortheme{hgwhite} % witte achtergrond, zwarte tekst

%% common.tex -- Code die in elk .tex-bestand terug komt

%% Packages

\usepackage[dutch]{babel}
\usepackage{graphicx}
\usepackage{comment,enumerate,hyperref}
\usepackage{amsmath,amsfonts,amssymb}
\usepackage{eurosym}
\usepackage{booktabs}
\usepackage{multicol,multirow}
\usepackage{listings}

\usepackage[outputdir=out]{minted}

\usepackage[backend=biber,style=apa]{biblatex}
\DeclareLanguageMapping{dutch}{dutch-apa}

\usepackage{csquotes}

%% Variabelen, elk academiejaar aan te passen
\newcommand{\academicyear}{2021--2022}
\newcommand{\lecturers}{Thomas Aelbrecht \and Lena De Mol \and Koen Mertens \and Jillisa Schittecatte \and Bert Van Vreckem}

%% Macro's en commando's

%% \alertbox: een kader voor tekst die moet opvallen
\newcommand{\alertbox}[2][hgblue]{%
  \setbeamercolor{alertbox}{bg=#1,fg=white}
  \begin{beamercolorbox}[sep=2pt,center]{alertbox}
    \textbf{#2}
  \end{beamercolorbox}
}


%---------- Info over de presentatie ------------------------------------------

\title{Module 0. Course intro, software installation.}
\subtitle{Research Methods}
\author{\lecturers}   % Pas waarden aan in common.tex
\date{\academicyear}

\begin{document}

\begin{frame}
  \maketitle
\end{frame}

\begin{frame}
  \frametitle{Contents}

  \tableofcontents
\end{frame}

\section{Studiewijzer}

\begin{frame}
  \frametitle{Doel van deze cursus}

  De bachelorproef blijkt een uitdaging!

  \begin{itemize}
    \item Onderzoekend i.p.v.\ uitvoerend
    \item Onderwerp uitputtend behandelen
    \item Een lange tekst schrijven
  \end{itemize}

\end{frame}

\begin{frame}
  \frametitle{Doel van deze cursus}

  Research Methods = hulp bij voorbereiding op bachelorproef!

  \begin{itemize}
    \item Wat is een geschikt onderwerp?
    \item Op zoek naar informatie
    \item Correct refereren naar vakliteratuur
    \item Structuur van een artikel, scriptie
    \item Helder communiceren
    \item Correct taalgebruik en passende schrijfstijl
  \end{itemize}
\end{frame}

\begin{frame}
  \frametitle{Leerdoelen}

  Zie Syllabus IT-luik en \href{https://bamaflexweb.hogent.be/BMFUIDetailxOLOD.aspx?a=134982&b=5&c=1}{studiefiche}!
\end{frame}

\begin{frame}
  \frametitle{Leerinhoud: IT-luik}

  \begin{enumerate}
    \item Gebruik van {\LaTeX}
    \item Een onderwerp kiezen
    \item Literatuuronderzoek
    \item Refereren en bibliografie
    \item Onderzoeksmethoden
    \item Schrijven van de bachelorproef
  \end{enumerate}

\end{frame}

\begin{frame}
  \frametitle{Leerinhoud: Taalluik}

  \begin{enumerate}
    \item Structuur v/e professionele tekst
    \item Een professionele schrijfstijl
    \item Focus op typische taalfouten
    \item Bronnen kritisch benaderen
    \item Een samenvatting schrijven
    \item Een goede presentatie geven
  \end{enumerate}

\end{frame}

\begin{frame}
  \frametitle{Leermateriaal}

  Al het leermateriaal wordt op Chamilo gepubliceerd, o.a.

  \begin{itemize}
    \item Syllabi
    \item Leerpad
    \item Lesopnames
    \item Oefeningen/Opdrachten
    \item \ldots
  \end{itemize}

\end{frame}

\begin{frame}
  \frametitle{Werkvormen}

  \begin{itemize}
    \item 6 lessen IT-luik
    \item 6 lessen taalluik
    \item Doorlopende opdracht, resultaat = paper (zie verder)
    \item Oefeningen
  \end{itemize}

\end{frame}

\begin{frame}
  \frametitle{Werk- en leeraanwijzingen}

  \begin{itemize}
    \item Deadline = Week 13
    \item Werk tijdens het semester!
    \item Zwaartepunt: ca.\ week 7--9
  \end{itemize}

\end{frame}

\begin{frame}
  \frametitle{Planning}

  \begin{table}
    \centering
    \small
    \begin{tabular}{cll}
      \toprule
      \textbf{Wk} & \textbf{IT-luik}                    & \textbf{Taalluik}                   \\
      \midrule
      1--2        & Voorbereiding, {\LaTeX}             & Structuur van professionele teksten \\
      3--4        & Een onderzoeksvraag formuleren      & Professionele schrijfstijl          \\
      5--6        & Literatuurstudie                    & Typische taalfouten                 \\
      7--8        & Correct refereren                   & Bronnen kritisch benaderen          \\
      9--10       & Onderzoeksmethoden                  & Een samenvatting schrijven          \\
      11--12      & Rapporteren                         & Een goede presentatie geven         \\
      13          & \multicolumn{2}{c}{Deadline paper!}                                       \\
      \bottomrule
    \end{tabular}
  \end{table}
\end{frame}

\begin{frame}
  \frametitle{Lectoren}

  \textbf{IT-luik:}

  \begin{itemize}
    \item Thomas Aelbrecht (Gent)
    \item Koen Mertens (Aalst, IC, TIAO, VC)
    \item Bert Van Vreckem (Titularis, Gent)
  \end{itemize}

  \bigskip
  \textbf{Taalluik:}

  \begin{itemize}
    \item Lena De Mol (IC, TIAO, VC)
    \item Jillisa Schittecatte (Aalst, Gent)
  \end{itemize}
\end{frame}

\begin{frame}
  \frametitle{Evaluatie: 100\% paper}

  \begin{itemize}
    \item Schrijven voorstel onderwerp bachelorproef
    \item Stap voor stap!
    \item Beoordeeld door:
          \begin{itemize}
            \item lector IT: inhoudelijk
            \item taallector: schrijfstijl, structuur, taal
          \end{itemize}
    \item Rubrics (worden z.s.m.\ gepubliceerd)
  \end{itemize}
\end{frame}

\begin{frame}
  \frametitle{Vragen?}

  \begin{itemize}
    \item Ik deed vorig jaar Onderzoekstechnieken (4SP), waarom moet ik dit vak volgen?
      \begin{itemize}
        \item Je kreeg vrijstelling voor Data Science \& AI (4SP), 90\% overlap inhoud
        \item Voor het onderdeel dat nu Research Methods (4SP) geworden is was er binnen OZT onvoldoende tijd
        \item We willen je helpen om een betere BP te schrijven!
      \end{itemize}
    \item Ik neem ook BP op, heb al onderwerp uitgeschreven!
      \begin{itemize}
        \item Je mag dat voorstel gebruiken als basis voor de paper!
        \item Gebruik wat je hier leert om je voorstel te verbeteren en maak er gebruik van in het werk voor je BP.
      \end{itemize}
  \end{itemize}
\end{frame}

\begin{frame}
  \frametitle{Vragen?}

  \begin{itemize}
    \item Moet ik dit onderwerp behouden voor mijn BP?
          \begin{itemize}
            \item Nee, je mag dan een nieuw onderwerp kiezen!
            \item bv.\ via stagebedrijf, keuzevak 3TI.
          \end{itemize}
    \item Krijgen we tussentijdse feedback?
          \begin{itemize}
            \item Met het grote aantal studenten kunnen we dit niet garanderen\ldots
          \end{itemize}
  \end{itemize}
\end{frame}

\section{Aan de slag!}

\begin{frame}
  \frametitle{Opzetten werkomgeving}

  \begin{itemize}
    \item Maak repo aan voor opdracht paper (via link Chamilo)
    \item Volg instructies fase 1: Opzetten werkomgeving
          \begin{itemize}
            \item Software installeren
            \item Configuratie Git, LaTeX
          \end{itemize}
  \end{itemize}
\end{frame}

\begin{frame}
  \frametitle{Installatie Software}

  \begin{itemize}
    \item Gebruik package manager!
    \item Cfr.\ System Engineering Lab (1TI)!
    \item Windows: Chocolatey
    \item MacOS X: Homebrew
    \item Linux: ingebouwd\ldots
  \end{itemize}
\end{frame}

\begin{frame}
  \frametitle{Configuratie}

  \begin{itemize}
    \item Git basisconfig!
    \item {\TeX}studio, Jabref: ``moderne'' compiler
  \end{itemize}
\end{frame}

\begin{frame}
  \frametitle{Onderhoud Mik{\TeX}}

  \begin{itemize}
    \item Open Mik{\TeX} console als Administrator
    \item Settings:
          \begin{itemize}
            \item Always install missing packages on-the-fly
            \item Default paper format: A4
          \end{itemize}
    \item Updates: Check for updates
          \begin{itemize}
            \item In geval van fouten: installeer updates
          \end{itemize}
    \item Zorg dat Mik{\TeX} niet geblokkeerd wordt door firewall/antivirus
  \end{itemize}
\end{frame}

\begin{frame}[fragile]
  \frametitle{Lettertypes}

  HOGENT huisstijl:

  \begin{itemize}
    \item Montserrat Regular, ExtraBold: \url{https://fonts.google.com/specimen/Montserrat}
    \item Code Pro Black: \url{https://www.dafontfree.net/freefonts-code-pro-black-f62435.htm}
  \end{itemize}

  Code-font met ligaturen (e.g. \({\leftarrow}\) i.p.v. \verb|<-|):

  \begin{itemize}
    \item Fira Code: \texttt{choco install firacode}
  \end{itemize}
\end{frame}

\begin{frame}
  \frametitle{Configuratie TeXstudio}

  \begin{itemize}
    \item Options > Configure TeXstudio
    \item Commands:
          \begin{itemize}
            \item XeLaTeX\@: \texttt{xelatex -synctex=1 -interaction=nonstopmode -shell-escape \%.tex}
          \end{itemize}
    \item Build:
          \begin{itemize}
            \item Default Compiler: XeLaTeX
            \item Default Bibliography tool: Biber
          \end{itemize}
    \item Editor, enz.\ naar eigen voorkeur
  \end{itemize}
\end{frame}

\begin{frame}
  \frametitle{Een alternatief: VS Code}

  \begin{itemize}
    \item VS Code kan {\TeX}studio, Jupyter Notebook en Markdown editor vervangen
    \item Configuratie is wat complexer
  \end{itemize}
\end{frame}

\begin{frame}
  \frametitle{Installatie}

  \begin{itemize}
    \item \texttt{choco install -y texlive}  (in plaats van Mik{\TeX})
    \item \texttt{choco install -y vscode}
    \item Extensies:
          \begin{itemize}
            \item {\LaTeX} Workshop (James Yu)
            \item GitLens (GitKraken)
            \item Jupyter (Keymap, Notebook Renderers) (Microsoft)
            \item Python, Pylance (Microsoft)
          \end{itemize}
  \end{itemize}
\end{frame}

\begin{frame}
  \frametitle{Installatie testen}

  \begin{itemize}
    \item Compileer sjabloon voor je paper!
    \item Krijg je een PDF?
    \item Bevat de PDF ook een bibliografie?
  \end{itemize}
\end{frame}

\end{document}
