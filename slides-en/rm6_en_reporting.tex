\documentclass[aspectratio=169]{beamer}
\usetheme{hogent}
\usecolortheme{hgwhite} % witte achtergrond, zwarte tekst

%% common.tex -- Code die in elk .tex-bestand terug komt

%% Packages

\usepackage[dutch]{babel}
\usepackage{graphicx}
\usepackage{comment,enumerate,hyperref}
\usepackage{amsmath,amsfonts,amssymb}
\usepackage{eurosym}
\usepackage{booktabs}
\usepackage{multicol,multirow}
\usepackage{listings}

\usepackage[outputdir=out]{minted}
%\usepackage{minted}

\usepackage[backend=biber,style=apa]{biblatex}
\DeclareLanguageMapping{dutch}{dutch-apa}

\usepackage{csquotes}

%% Variabelen, elk academiejaar aan te passen
\newcommand{\academicyear}{2023--2024 (revisie: \today)}
\newcommand{\lecturers}{Thomas Aelbrecht \and Thomas Parmentier \and Bert Van Vreckem}
\newcommand{\coursename}{Research Methods (IT)}

%% Macro's en commando's

%% \alertbox: een kader voor tekst die moet opvallen
\newcommand{\alertbox}[2][hgblue]{%
  \setbeamercolor{alertbox}{bg=#1,fg=white}
  \begin{beamercolorbox}[sep=2pt,center]{alertbox}
    \textbf{#2}
  \end{beamercolorbox}
}


%---------- Info over de presentatie ------------------------------------------

\title{Module 6. Rapporteren over onderzoek in \LaTeX{}.}
\subtitle{Research Methods}
\author{\lecturers}   % Pas waarden aan in common.tex
\date{\academicyear}

\begin{document}

\begin{frame}
  \maketitle
\end{frame}

\begin{frame}
  \frametitle{Inhoud}

  \tableofcontents
\end{frame}

% - LaTeX tips & tricks:
%   - afbeeldingen
%   - ...
% - Opdracht: schrijf BP-onderwerp uit op basis van voorgaande werk

\section{Criteria voor beoordeling voorstel}

\begin{frame}{Criteria voor beoordeling voorstel}
  
  \alertbox{De \textbf{titel} is \textbf{concreet} en geeft een correct beeld van de onderzoeksdoelstelling}
  
  \begin{itemize}
    \item Niet enkel het onderzoeksdomein benoemen
    \item Geen vraag
    \item Geen ``Onderzoek naar''
    \item Hoeft niet definitief te zijn!
  \end{itemize}
  
  % TODO: voorbeelden geven: goed/slecht
  
  %\textcolor{OrangeRed}{\XBox} De cloud
  %
  %\textcolor{ForestGreen}{\CheckedBox} Selectie van een Platform-as-a-Service oplossing voor een 
  
\end{frame}

\begin{frame}{Criteria voor beoordeling voorstel}
  
  \alertbox{Er is een \textbf{samenvatting} met alle nodige elementen}
  \small
  \begin{description}
    \item[Context] Waarom is dit werk belangrijk?
    \item[Nood] Waarom moet dit onderzocht worden?
    \item[Taak] Wat ga je specifiek doen doen?
    \item[Object] Wat staat in dit document geschreven?
    \item[Resultaat] Welk concreet resultaat verwacht je van je onderzoek?
    \item[Conclusie] Wat verwacht je van van de conclusies?
    \item[Perspectief] Waarvoor zal het resultaat kunnen gebruikt worden? Wat is de meerwaarde van je werk?
  \end{description}
  
  \alertbox{Een samenvatting is geen inleiding!}
  
\end{frame}

\begin{frame}{Criteria voor beoordeling voorstel}
  
  \alertbox{Het voorstel is \textbf{vernieuwend} en heeft een duidelijke \textbf{meerwaarde} voor een specifieke doelgroep uit het werkveld}
  
  \begin{itemize}
    \item Op een onderzoeksvraag is nu nog geen antwoord!
    \item Eigen bijdrage moet duidelijk zijn!
    \item Baken domein af naar specifieke doelgroep toe (bv. een bedrijf)
    \item Geen ``algemeen'' onderwerp
  \end{itemize}
  
\end{frame}

\begin{frame}{Criteria voor beoordeling voorstel}
  
  \alertbox{De methodologie is duidelijk verantwoord, onderzoekstechnieken zijn geschikt voor beantwoorden onderzoeksvraag}
  
  \begin{itemize}
    \item Concreet plan van aanpak!
    \item Uit welke fasen bestaat jouw onderzoek?
    \item Welk doel en resultaat heeft elke fase?
    \item Wees specifiek! Niet: ``onderzoek doen naar\ldots''
    \begin{itemize}
      \item Literatuurstudie
      \item Interview (bv. requirements verzamelen)
      \item Experiment (bv. performantiemeting)
      \item Vergelijkende studie
      \item Risico-analyse (bv. in kader van security)
      \item Proof-of-concept opzetten
      \item \ldots
    \end{itemize}
  \end{itemize}
  
\end{frame}

\begin{frame}{Criteria voor beoordeling voorstel}
  
  \alertbox{Er wordt op een correcte manier verwezen naar vakliteratuur}
  
  \begin{itemize}
    \item Voldoende referenties
    \item Elke bewering
    \item APA-stijl (volgens sjabloon)
  \end{itemize}
  
\end{frame}

\begin{frame}{Criteria voor beoordeling voorstel}
  
  \alertbox{Er wordt een \textbf{zakelijke schrijfstijl} gehanteerd}
  
  \begin{description}
    \item[Zakelijk] geen bloemrijk, informeel taalgebruik, spreektaal
    \item[Objectief] geen mening, enkel aantoonbare feiten
    \item[Onpersoonlijk] \textbf{Geen ``ik''}/wij
    \item[Precies] geen vage uitspraken, alles kwantificeren
    \item[Correct] taalgebruik: grammatica, spelling
  \end{description}
  
  Zie \url{http://www.taalwinkel.nl/een-wetenschappelijke-schrijfstijl/}
  
\end{frame}

\section{Uitdagingen}

\begin{frame}{Uitdagingen}
  
  \begin{itemize}
    \item Inwerken in een nieuw domein
    \item Iets \textbf{onderzoeken} is moeilijker dan concrete opdracht \textbf{uitvoeren}
    \item Uitstelgedrag!
    \item Gebruik stagevrije dag voor BP, niet voor andere activiteiten!
  \end{itemize}
  
\end{frame}

\section{Veel succes!}

\end{document}
